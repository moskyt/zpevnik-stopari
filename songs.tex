\pagestyle{empty}

\begin{song}{Amazonka}{Hop-trop}

\begin{xverse}{1.~}
Byly krásný naše [\large G]plány, [\large C|]{} [\large G|]{}
byla jsi můj celej svět, [\large Hmi|]{} [\large Bmi|]{}
[\large Ami]{}čas je vzal a nechal [\large G]rány,
[\large Ami]starší jsme jen o pár [\large D]let.
\end{xverse}

\begin{xverse}{2.~}
Tenkrát byly děti malý, ale život utíká,
už na ``táto'' slyší jinej, i když si tak neříká.
\end{xverse}

\begin{xverse}{R.~}
Nebe modrý zrcad[\large G]lí se
v [\large E7]{}řece, která všechno [\large Ami]ví,
stejnou barvu jako [\large G]měly
[\large Ami]tvoje oči džíno[\large D]vý.
\end{xverse}

\begin{xverse}{3.~}
Kluci tenkrát, co tě znali,  všude, kde jsem s tebou byl,
Amazonka říkávali, a já hrdě přisvědčil.
\end{xverse}

\begin{xverse}{4.~}
Tvoje strachy, že ti mládí pod rukama utíká
vedly k tomu, že ti nikdo Amazonka neříká.
\end{xverse}

\begin{xverse}{R.~}
Nebe modrý zrcadlí se ...
\end{xverse}

\begin{xverse}{5.~}
Zlatý kráse cingrlátek, jak sis časem myslela,
vadil možná trampskej šátek, nosit dál's ho nechtěla.
\end{xverse}

\begin{xverse}{R.~}
Teď jsi víla z paneláku, samá dečka, samej krám,
já si přál jen, abys byla pořád stejná, přísahám,
pořád [\large Hmi]stejná, přísa[\large A]hám.
\end{xverse}

\end{song}

\setcounter{page}{1}

\begin{song}{Anděl}{Karel Kryl}

\begin{xverse}{1.~}
[\large C]Z rozmláce[\large Ami]nýho kostela v [\large C]krabici s [\large G7]kusem mýdla
[\large C]přinesl [\large Ami]jsem si anděla, [\large C]poláma[\large G7]li mu [\large C]křídla,
díval se [\large Ami]na mě oddaně, [\large C]já měl jsem [\large G7]trochu trému,
[\large C]tak vtiskl [\large Ami]jsem mu do dlaně [\large C]lahvičku [\large G7]od par[\large C]fému.
\end{xverse}

\begin{xverse}{R.~}
[\large C]A proto, [\large Ami]prosím, věř mi, [\large C]chtěl jsem ho [\large G7]{žá}dat,
[\large C]aby mi [\large Ami]mezi dveřmi [\large C]pomohl [\large G7]hádat,
[\large C]co mě čeká [\large Ami|]{}   [\large G7]a nemi[\large C]ne, co mě čeká [\large Ami|]{}   [\large G7]a    nemi[\large C]ne.
\end{xverse}

\begin{xverse}{2.~}
Pak hlídali jsme oblohu, pozorujíce ptáky,
debatujíce o Bohu a hraní na vojáky,
do tváře jsem mu neviděl, pokoušel se ji schovat,
to asi ptákům záviděl, že mohou poletovat.
\end{xverse}

\begin{xverse}{R.~}
A proto, prosím, věř mi ...
\end{xverse}

\begin{xverse}{3.~}
Když novinky mi sděloval u okna do ložnice,
já křídla jsem mu ukoval z mosazný nábojnice,
a tak jsem pozbyl anděla, on oknem odletěl mi,
však přítel prý mi udělá novýho z mojí helmy.
\end{xverse}

\begin{xverse}{R.~}
A proto, prosím, věř mi ...
\end{xverse}

\end{song}

\begin{song}{Až vzlétnou ptáci}{Spirituál Kvintet}

%TODO predelat komplet akordy

\begin{xverse}{1.~}
[\large D]Na před[\large A]městí [\large D]stá[\large A]val [\large Hmi]dům, malý [\large A]chlapec [\large Hmi]si tam [\large A]hrával
[\large D]drak, co [\large A]vzlétal k [\large D]ob[\large A]la[\large Hmi]kům, všechna [\large A]tajná [\large Hmi]přání [\large A]znával.
[\large G]Draci totiž [\large A]vždycky [\large G]ví to, co je klukům [\large A]nejvíc [\large G]líto
[\large Hmi]když mu[\large A]sí jít [\large Hmi]večer [\large A]spát.
\end{xverse}

\begin{xverse}{2.~}
Jako víno dozrává, jako v mořích vlny hasnou
vzpomínka mi zůstává na tu smutnou zemi krásnou
na rybářské staré sítě, na draka a malé dítě,
které nemá si s kým hrát.
\end{xverse}

\begin{xverse}{3.~}
Na provázku slunce měl, oblohou se za ním vznášel,
jako vánek šel kam chtěl smutné lampy lidem zhášel,
pohádkovou dýchal vůni, mluvil řečí horských tůní,
průzračnou jak dětský smích.
\end{xverse}

\begin{xverse}{4.~}
Bílý koník běžel dál osamělou pustou plání,
na dlani sníh dětství tál i když padal bez ustání,
den začíná tichou flétnou, chvíli předtím, nežli vzlétnou
hejna ptáků ve větvích.
\end{xverse}

\begin{xverse}{5.~}
= 1.
\end{xverse}

\end{song}

\begin{song}{Banka (Make love Cosa Nostra)}{Hoboes}

\begin{xverse}{1.~}
Ze starejch [\large D]{}časáků a [\large E]fotek zažlout[\large A7]lejch
šklebí se [\large D]chlap, co už se [\large E]poldům dávno [\large A7]zdejch
gangster má na kahánku, když kouří [\large D]marijánku
miluje [\large A7]Paula Anku, je velkej [\large D]lump.
\end{xverse}

\begin{xverse}{2.~}
Tuctovej ksicht a je to přece velkej šéf
na Pátý avenue se třese každej sejf
kolťáky vycíděný, heroin a krásný ženy
dolary upocený nejsou náš džob.
\end{xverse}

\begin{xverse}{R.~}
[\large G]Jedeme přepadnout [\large C]veli[\large D]kánskou [\large G]banku
[\large G]ve vokýnku kulomet a [\large C]dvě stě [\large D]litrů v [\large G]tanku
[\large A7]ukradneme mraky dolarů
[\large D]sejdeme se večer u báru a tam je [\large D7]prochlastáme
[\large G]Zastřelíme poldu, [\large C]co tu [\large D]banku [\large G]chrání
[\large G]naši velký loupeži už [\large C]nikdo [\large D]neza[\large G]brání
[\large A7]soustředíme všechno úsilí
[\large D]poldové už shání posily a my je [\large D7]zastřelíme
[\large G]do břicha sekerou. [\large A7|]{} [\large D|]{}
\end{xverse}


\begin{xverse}{3.~}
Nacpeme kolťáky do pouzder podpažních
namažem mechanismy zbraní trofejních
naskáčem do chrysleru kašleme na aféru
rozdělíme si sféru odkud až kam.
\end{xverse}


\begin{xverse}{4.~}
Bezdýmej prach a prvotřídní auťáky
falešný dolary a whisky hekťáky
mafie naše máma decentně hejbe s náma
you are my sugar baby make love get back.
\end{xverse}


\end{song}

\begin{song}{Bedna vod whisky}{Hoboes}

\begin{xverse}{1.~}
[\large Ami]Dneska už mně [\large C]fóry ňák [\large Ami]nejdou přes pys[\large E]ky,
[\large Ami]stojím s dlouhou [\large C]kravatou na [\large Ami]bedně [\large E]vod whis[\large Ami]ky,
stojím s dlouhým [\large C]vobojkem [\large Ami]jak stájovej [\large E]pinč,
tu [\large Ami]kravatu, co [\large C]nosím, mi [\large Ami]navlík' [\large E]soudce [\large A]Lynč.
\end{xverse}


\begin{xverse}{R.~}
Tak [\large A]kopni do tý [\large D]bedny, ať [\large E]panstvo neče[\large A]ká,
jsou dlouhý schody [\large D]do nebe a [\large E]{}štreka dale[\large A]ká
do nebeskýho [\large D]baru, já [\large E]sucho v krku [\large A]mám,
tak kopni do tý [\large D]bedny, ať [\large E]na cestu se [\large A]dám. [\large Ami]{}
\end{xverse}


\begin{xverse}{2.~}
Mít tak všechny bedny od whisky vypitý,
postavil bych malej dům na louce ukrytý,
postavil bych malej dům a z vokna koukal ven
a chlastal bych tam s Billem a chlastal by tam Ben.
\end{xverse}


\begin{xverse}{R.~}
Tak kopni do tý bedny ...
\end{xverse}


\begin{xverse}{3.~}
Kdyby jsi se, hochu, jen porád nechtěl rvát,
nemusel jsi dneska na týhle bedně stát,
moh' jsi někde v suchu tu svoji whisku pít,
nemusel jsi dneska na krku laso mít.
\end{xverse}


\begin{xverse}{R.~}
Tak kopni do tý bedny ...
\end{xverse}


\begin{xverse}{4.~}
Až kopneš do tý bedny, jak se to dělává,
do krku mi zvostane jen dírka mrňavá,
jenom dírka mrňavá a k smrti jenom krok,
má to smutnej konec, a whisky ani lok.
\end{xverse}


% \begin{xverse}{R.~}
% Tak kopni do tý bedny...
% \end{xverse}


\end{song}

\begin{song}{Blátivá cesta}{Pacifik}

\begin{xverse}{R.~}
[\large C]Blátivou, [\large Emi]blátivou cestou [\large F]dál nechceš [\large C]jít,
kde jen [\large F]máš touhu [\large C]blázni[\large Emi]vou,
kde jen [\large Dmi]máš, co chtěl jsi [\large G]mít, chtěl jsi [\large C]mít.
\end{xverse}

\begin{xverse}{1.~}
[\large Ami]A tak se [\large G]koukáš, jak si [\large Ami]kolem hrajou děti,
[\large C]ve slunci [\large G]kotě [\large Ami]usíná,
[\large F]a jak si [\large G]před hospodou [\large C]vyprávějí [\large Ami]kmeti,
[\large D7]{}život prej stále začí[\large G]ná.
\end{xverse}

\begin{xverse}{2.~}
Z města tě vyhánějí ocelový stíny,
jak dříve šel bys asi rád,
z bejvalejch cest ti zbyly potrhaný džíny,
čas běží, je to ale znát.
\end{xverse}

\begin{xverse}{R.~}
Blátivou, blátivou ...
\end{xverse}

\begin{xverse}{3.~}
Na poli pokoseným přepočítáš snopy,
do trávy hlavu položíš,
zdá se ti o holkách, co oči vždycky klopí,
po jiným ani netoužíš.
\end{xverse}

\begin{xverse}{4.~}
Měkký jsou stíny, dole zrcadlí se řeka,
nad jezem kolébá se prám,
kolem je ticho, že i vlastní hlas tě leká,
a přesto necítíš se sám.
\end{xverse}

\begin{xverse}{R.~}
Blátivou, blátivou ...
\end{xverse}

\end{song}

\begin{song}{Blues folsomské věznice}{Greenhorns}

\begin{xverse}{1.~}
Můj [\large G]děda bejval blázen, texaskej ahasver,
a na půdě nám po něm zůstal [\large G7]ošoupanej kvér,
ten [\large C]kvér obdivovali všichni kámoši z oko[\large G]lí
a [\large D7]máma mi říkala: ``Nehraj si s tou pisto[\large G]lí!''
\end{xverse}

\begin{xverse}{2.~}
Jenže i já byl blázen, tak zralej pro malér,
a ze  zdi jsem sundával tenhleten dědečkův kvér,
pak s kapsou vyboulenou chtěl jsem bejt chlap all right
a s holkou vykutálenou hrál jsem si na Bonnie and Clyde.
\end{xverse}

\begin{xverse}{3.~}
Ale udělat banku, to není žádnej žert,
sotva jsem do ní vlítnul, hned zas vylít' jsem jak čert,
místo jako kočka já utíkám jak slon,
takže za chvíli mě veze policejní anton.
\end{xverse}

\begin{xverse}{4.~}
Teď okno mřížovaný mi říká, že je šlus,
proto tu ve věznici zpívám tohle Folsom Blues.
pravdu měla máma, radila: ``Nechoď s tou holkou!'',
a taky mi říkala: ``Nehraj si s tou pistolkou!''
\end{xverse}

\end{song}

\begin{song}{Betty}{Hop trop}

\begin{xverse}{1.~}
Já na [\large Gmi]plachtu svýho vozu jako [\large F]každej jsem si [\large D]psal:
``Hrab a [\large Gmi]dři anebo umři!'' a pak [\large F]na západ se [\large D]hnal,
z plachty [\large A]dávno jsou už cáry, ale [\large G]heslo platí [\large D]dál,
mě [\large G]vítá Kali[\large D]fornie, tak [\large A7]nač bych umí[\large D]ral. [\large D7]{}
\end{xverse}

\begin{xverse}{R.~}
Betty, [\large G]vyndej z bedny soudek, rozžvejkám a spolknu [\large Ami]{}špunt,
upíchnem se právě tady, v ruce [\large D]{}žmoulám slibnej [\large G]grunt,
doufám, že to s nima zmáknu, vodsaď dál už nepu[\large Ami]dem,
navěky snad přece smůla nebu[\large D]de mým osu[\large G]dem. [\large Gmi|]{}
\end{xverse}

\begin{xverse}{2.~}
Zarazíme první kolík, druhej támhle musí bejt,
za potok dej ty dva další, budeme v něm zlato mejt,
jedu sehnat ňakej ouřad, Betty, pojď mě vobejmout,
tebe přiklepli mi tenkrát, teď i dílec přiklepnou.
\end{xverse}

\begin{xverse}{R.~}
Betty, vyndej z bedny soudek ...
\end{xverse}

\end{song}

\begin{song}{Divoké koně}{Jarek Nohavica}

\begin{xverse}{1.~}
/: [\large Ami]Já viděl [\large Dmi]divoké [\large Ami]koně, [\large C]běželi [\large Dmi]soumra[\large Ami]kem, :/
[\large Dmi]vzduch [\large Ami]těžký [\large Dmi]byl a divně [\large Ami]voněl [\large Ddim]{\,\,}tabá[\large F]kem,
[\large Dmi]vzduch [\large Ami]těžký [\large Dmi]byl a divně [\large Ami]voněl [\large E7]tabá[\large Ami]kem.
\end{xverse}

\begin{xverse}{2.~}
Běželi, běželi bez uzdy a sedla krajinou řek a hor,
sper to čert, jaká touha je to vedla za obzor?
\end{xverse}

\begin{xverse}{3.~}
Snad vesmír nad vesmírem, snad lístek na věčnost,
naše touho, ještě neumírej, sil máme dost.
\end{xverse}

\begin{xverse}{4.~}
V nozdrách sládne zápach klisen na břehu jezera,
milování je divoká píseň večera.
\end{xverse}

\begin{xverse}{5.~}
Stébla trávy sklání hlavu, staví se do šiku,
král s dvořany přijíždí na popravu zbojníků.
\end{xverse}

\begin{xverse}{6.~}
Chtěl bych jak divoký kůň běžet, běžet, nemyslet na návrat,
s koňskými handlíři vyrazit dveře, to bych rád.
\end{xverse}

\begin{xverse}{7.~}
Já viděl divoké koně ...
\end{xverse}

\end{song}
\chords{\chordAdim}

\begin{song}{Dokud se zpívá}{Jarek Nohavica}

\begin{xverse}{1.~}
Z [\large C]Těšína [\large Emi]vyjíždí [\large Dmi7]vlaky co [\large F]{}čtvrthodi[\large C]nu, [\large Emi]{\qquad} [\large Dmi7]{\qquad} [\large G]{\qquad}
[\large C]včera jsem [\large Emi]nespal a [\large Dmi7]ani dnes [\large F]nespoči[\large C]nu,  [\large Emi]{\qquad} [\large Dmi7]{\qquad} [\large G]{\qquad}
[\large F]svatý Me[\large G]dard, můj pa[\large C]tron, ťuká [\large Ami]si na če[\large G]lo,
ale [\large F]dokud se [\large G]zpívá, [\large F]ještě se [\large G]neumře[\large C]lo.  [\large Emi]{\qquad} [\large Dmi7]{\qquad} [\large G]{\qquad}
\end{xverse}

\begin{xverse}{2.~}
Ve stánku koupím si housku a slané tyčky,
srdce mám pro lásku a hlavu pro písničky,
ze školy dobře vím, co by se dělat mělo,
ale dokud se zpívá, ještě se neumřelo.
\end{xverse}

\begin{xverse}{3.~}
Do alba jízdenek lepím si další jednu,
vyjel jsem před chvílí, konec je v nedohlednu,
za oknem míhá se život jak leporelo,
ale dokud se zpívá, ještě se neumřelo.
\end{xverse}

\begin{xverse}{4.~}
Stokrát jsem prohloupil a stokrát platil draze,
houpe to, houpe to na housenkové dráze,
i kdyby supi se slítali na mé tělo,
tak dokud se zpívá, ještě se neumřelo.
\end{xverse}

\begin{xverse}{5.~}
Z Těšína vyjíždí vlaky až na kraj světa,
zvedl jsem telefon a ptám se:"Lidi, jste tam?"
A z veliké dálky do uší mi zaznělo,
že dokud se zpívá, ještě se neumřelo,
že dokud se zpívá ještě se neumřelo
\end{xverse}

\end{song}
\chords{\chordDmiSeven}

\begin{song}{Dům U vycházejícího slunce}{}

\begin{xverse}{1.~}
Snad [\large Ami]znáš ten [\large C]dům za [\large D]New Or[\large F]leans,
ve [\large Ami]{}štítu [\large C]znak slunce [\large E]má,
je to [\large Ami]dům, kde [\large C]lká sto [\large D]chlapců ubo[\large F]hejch
a [\large Ami]kde jsem [\large E]zkejs' i [\large Ami]já. \ \  [\large C|]{} [\large D|]{} [\large F|]{} [\large Ami|]{} [\large E|]{} [\large Ami|]{} [\large E|]{}
\end{xverse}

\begin{xverse}{2.~}
Mý mámě Bůh dal věnem svatebním
jen prát a šít blue jeans,
táta můj se flákal jen
sám po New Orleans.
\end{xverse}

\begin{xverse}{3.~}
Bankrotář se zhroutil před hernou,
jenom bídu svou měl a chlast,
k putykám pak táh' tu pouť mizernou
a znal jen pít a krást.
\end{xverse}

\begin{xverse}{4.~}
Být matkou, dám svým synům
lepší dům, než má kdo z vás,
ten dům, kde spím, má emblém sluneční,
ale je v něm sníh a mráz.
\end{xverse}

\begin{xverse}{5.~}
Kdybych směl se hnout z těch kleští,
pěstí vytrhnout tu mříž,
já jak v snách bych šel do New Orleans
a měl tam k slunci blíž.
\end{xverse}

\begin{xverse}{6.~}
=\ 1.
\end{xverse}

\end{song}

\begin{song}{Fi-li-mi}{Spirituál Kvintet}

\begin{xverse}{1.~}
[\large Emi]{Čert} aby vzal už tuhle trať, kdo [\large G]hledáš práci, tak se ztrať,
[\large Emi]{že} nemáš prachy, no tak ať, jó, tak se [\large Hmi]na to [\large Emi]dívám!
\end{xverse}

\begin{xverse}{R.~}
[\large Emi]Fi-li-mi-jo-ri-jú-ri-ej,  [\large G]fi-li-mi-jo-ri-jú-ri-ej,
[\large Emi]fi-li-mi-jo-ri-jú-ri-ej, vo tom [\large Hmi]si teď [\large Emi]zpívám.
\end{xverse}

\begin{xverse}{2.~}
Jen pražec chop a kolej suň, chyť lano, táhni jako kůň,
pod tíhou jako medvěd fuň, jó, tak se na to dívám!
\end{xverse}

\begin{xverse}{R.~}
Fi-li-mi ...
\end{xverse}

\begin{xverse}{3.~}
Z kůže se loupeš jako had, je vedro, že by jeden pad',
na vodu smíš jen vzpomínat, jó, tak se na to dívám!
\end{xverse}

\begin{xverse}{R.~}
Fi-li-mi ...
\end{xverse}

\begin{xverse}{4.~}
Když konečně máš vody dost, určitě přes ni stavíš most,
kláda ti ráda zlomí kost, jó, tak se na to dívám!
\end{xverse}

\begin{xverse}{R.~}
Fi-li-mi ...
\end{xverse}

\begin{xverse}{5.~}
Na rukách už jsem potěžkal většinu těch okolních skal,
ještě to cejtí každej sval, jó, vo tom si teď zpívám!
\end{xverse}

\begin{xverse}{R.~}
Fi-li-mi ...
\end{xverse}

\begin{xverse}{6.~}
Slunce už dělá z trávy troud,
jen kdybych se směl vodsaď hnout,
na tuhle trať zapomenout, jó, tak se na to dívám!
\end{xverse}

\begin{xverse}{R.~}
Fi-li-mi ...
\end{xverse}

\begin{xverse}{6.~}
Jen Bůh mi víru zachovej a nasednout mi sílu dej,
můj [\large Emi]vagón bude [\large C]pérovej,
jó, [\large Ami]vo tom [\large Hmi]si teď [\large Emi]zpívám!
\end{xverse}

\begin{xverse}{R.~}
Fi-li-mi ...
\end{xverse}

\end{song}

\begin{song}{Fram}{Wabi Daněk}

\begin{xverse}{1.~}
[\large Ami]Zas mě to táhne o kus [\large Ami6|]{dál,} [\large Ami]{zas} nemám doma nikde [\large D]stání,
desítky důvodů si [\large Dmi]sháním, [\large E7]už abych na cestu se [\large Ami]dal. [\large Ami6]{}
\end{xverse}

\begin{xverse}{2.~}
Pelikán křídly zamával, vítr je příhodný a stálý,
za námi slunce mosty pálí, tak proč bych ještě vyčkával.
\end{xverse}

\begin{xverse}{R.~}
[\large Ami]Klenotník měsíc zavřel krám, [\large Ami/G#|]{}
[\large Ami/G]{z vý}kladu [\large Ami/F#]svoje šperky [\large Emi]sklízí,
[\large Dmi]obrysy domů v dálce [\large Emi]mizí,
[\large Dmi]tak naposled ti zamá[\large E7]vám
z paluby lodi jménem [\large Ami]Fram.
\end{xverse}

\begin{xverse}{3.~}
Dávno už vyvětral se dým mých věčných cigaret a dýmek,
ty žiješ jenom ze vzpomínek, a já se stále nevracím.
\end{xverse}

\begin{xverse}{4.~}
Námořní mapy pokryl prach, mé knihy nikdo neutírá,
nevíš, zda právě neumírám tam někde na ledových krách.
\end{xverse}

\begin{xverse}{R.~}
Klenotník měsíc zavřel krám, z výkladu svoje šperky sbírá,
chlap jen tak lehce neumírá, na modré lodi jménem Fram
tě za pár roků vyhledám.
\end{xverse}

\end{song}
\chords{ \chordAmiSix \chordAmiGis \chordAmiG \chordAmiFis }

\begin{song}{Frankie Dlouhán}{}

\begin{xverse}{1.~}
[\large C]Kolik je smutného, když [\large F]mraky černé [\large C]jdou
[\large C]lidem nad hla[\large G]vou, [\large F]smutnou dála[\large C]vou,
[\large C]já slyšel příběh, který [\large F]velkou pravdu [\large C]měl,
za čas odle[\large G]těl, [\large F]každý zapom[\large C]něl.
\end{xverse}

\begin{xverse}{R.~}
Měl kapsu [\large G]prázdnou Frankie Dlouhán,
po státech [\large F]toulal se jen [\large C]sám,
a že byl [\large F]veselej, tak [\large C]každej měl ho [\large G]rád.
Tam ruce [\large F]k dílu mlčky přiloží a [\large C]zase jede [\large Ami]dál,
a [\large F]každej kdo s ním [\large G]chvilku byl,
tak [\large F]dlouho [\large G]se pak [\large C]smál.
\end{xverse}

\begin{xverse}{2.~}
Tam kde byl pláč, tam Frankie hezkou píseň měl,
slzy neměl rád, chtěl se jenom smát.
A když pak ranče večer tiše usínaj,
Frankův zpěv jde dál, nocí s písní dál.
\end{xverse}

\begin{xverse}{3.~}
Tak Frankieho vám jednou našli, přestal žít,
jeho srdce spí, tiše klidně spí.
Bůh ví jak,za co, tenhle smíšek konec měl,
farář píseň pěl, umíráček zněl.
\end{xverse}

\end{song}

% \begin{song}{František}{Buty}
%
% \begin{xverse}{1.~}
% [\large G]Na hladinu rybníka svítí sluníč[\large C]ko
% [\large Emi]a kolem stojí v hustém kruhu [\large G]topoly,
% [\large Ami]které tam zasadil jeden hodný [\large Hmi]{čl}ověk,
% [\large Ami]jmenoval se František [\large D]Dobrota.
% \end{xverse}
%
%
% \begin{xverse}{2.~}
% František Dobrota, rodák z blízké vesnice,
% měl hodně dětí a jednu starou babičku,
% která když umírala, tak mu řekla: Františku,
% teď dobře poslouchej, co máš všechno udělat.
% \end{xverse}
%
%
% \begin{xverse}{R.~}
% [\large C]Balabambam, balabambam, [\large D|]{} [\large C|]{}
% balabambam, balabambam,  [\large D|]{} [\large C|]{}
% balabambam, balabambam,  [\large D|]{} [\large C|]{}
% [\large Ami]a kolem rybníka nahusto nasázet [\large D]topoly
% \end{xverse}
%
%
% \begin{xverse}{3.~}
% František udělal všechno, co mu řekla,
% balabambam, balabambam,
% a po snídani poslal děti do školy,
% žebriňák s nářadím dotáhl od chalupy k rybníku,
% vykopal díry a zasadil topoly.
% \end{xverse}
%
%
% \begin{xverse}{4.~}
% Od té doby vítr na hladinu nefouká,
% takže je klidná jako velké zrcadlo,
% sluníčko tam svítí vždycky rádo,
% protože v něm vidí Františkovu babičku.
% \end{xverse}
% \end{song}

\begin{song}{Hajnej Hruška}{Hop trop}

\begin{xverse}{1.~}
[\large Emi]Na pařez já [\large C7]used[\large H7]nu si v [\large Emi]lesním polo[\large H7_]{še}[\large Emi]{ru}
a na hajnýho [\large C7]vzpome[\large H7]nu si, [\large Emi]jenž má hezkou [\large H7]dce[\large Emi]ru,
[\large D7]na hajnýho [\large G]Hrušku a [\large D7]jeho dceru - [\large G]samej skvost,
[\large D7]jenže von má [\large G]pušku a s [\large F#]puškou střeží [\large H7]dcery ctnost.
\end{xverse}

\begin{xverse}{2.~}
Na pařezu přemejšlím, a dá to velkou fušku,
jak bych vyzrál na hajnýho, na hajnýho Hrušku,
na Hrušku a jeho zbraň a křepeláka Azora,
kterej hlídá jako saň vchod do hájovny ze dvora.
\end{xverse}

\begin{xverse}{R.~}
A [\large Emi]{ště}ká přitom [\large C7]na srnce, [\large Emi]na datly i [\large C7]{žlů}vy,
[\large Emi]na ťuhýka [\large C7]na trnce, [\large Emi]na vejry i [\large C7]sůvy,
[\large Ami]na chudáka [\large Emi]vandráka von [\large Ami]{ště}ká ponej[\large Emi]více,
[\large Ami]vidí ve mně [\large Emi]pytláka, co [\large F#]líčí na za[\large H7]jíce.
\end{xverse}

\begin{xverse}{*.~}
[\large Emi]Vrr haf, vrr haf, [\large C7]sypej [\large H7]si to, [\large Emi]vrr haf, vrr haf, [\large H7]fuj fuj [\large Emi]fuj,
[\large Emi]vrr haf, vrr haf, [\large C7]sypej [\large H7]si to, [\large Emi]padej pryč a [\large H7]upa[\large Emi]luj!
\end{xverse}

\begin{xverse}{3.~}
Hruška zbystří sluch i zrak a vzkřikne:"Namouvěru,
zas je tu ten darebák, co zprznit mi chce dceru,
zas je tu ten chuligán, co slídí, kde je dcerka,
jenže já se do něj dám a proženu mu perka!"
\end{xverse}

\begin{xverse}{4.~}
A už běží, v hubě pěnu, dělá dlouhý kroky,
pušku k palbě připravenu, má v ní srnčí broky,
letí, letí jeko blesk ze světnice na dvorek,
ve vočích má divnej lesk i jeho pejsek Azorek.
\end{xverse}

\begin{xverse}{R.~}
Ten, kterej má rád štvanice na lišky i kance,
když vypukne pranice, vždycky v ní má šance,
on si troufne na zvíře tak, jako medvěd velký,
a milýho trempíře chce kousnout do pr...avý ruky.
\end{xverse}

\begin{xverse}{5.~}
Vím, jak vyzrát na hajnýho, ba i na Azora:
vyštuduju na vrchního lesů revizora,
až přijedu na kontrolu se služební volhou,
postavím je do pozoru, což je mojí touhou.
\end{xverse}

\begin{xverse}{6.~}
Počkej, hajnej, povím na tě, že jsi prodal jedli,
žes' ji střelil nastojatě, a on bude zbledlý,
``smilujou se, revizore, vždyť mám doma dceru,''
jenže, Hruško, na tvou dceru já už dávno ... hej, beru!
\end{xverse}

\begin{xverse}{Rec.~}
Tak teda, tatínku, do smrti dobrý, ne?
\end{xverse}

\begin{xverse}{*.~}
A to bude asi všecko, na zdi visí puška,
pod ní kolíbá mi děcko můj tchán - hajnej Hruška ...
\end{xverse}

\end{song}

\begin{song}{Hlídač krav}{Jarek Nohavica}

\begin{xverse}{1.~}
[\large D]Když jsem byl malý, říkali mi naši:
``Dobře se uč a jez chytrou kaši,
[\large G]až jednou vyrosteš, [\large A]budeš doktorem [\large D]práv,
takový doktor sedí pěkně v suchu,
bere velký peníze a škrábe se v uchu,''
[\large G]já jim ale na to řek': ``[\large A]Chci být hlídačem [\large D]krav.''
\end{xverse}

\begin{xverse}{R.~}
Já chci [\large D]mít čapku s bambulí nahoře,
jíst kaštany, mýt se v lavoře,
[\large G]od rána po celý [\large A]den zpívat si [\large D]jen,
zpívat si: pam pam pa dam ...
\end{xverse}

\begin{xverse}{2.~}
K vánocům mi kupovali hromady knih,
co jsem ale vědět chtěl, to nevyčet' jsem z nich:
nikde jsem se nedozvěděl, jak se hlídají krávy,
ptal jsem se starších a ptal jsem se všech,
každý na mě hleděl jako na pytel blech,
každý se mě opatrně tázal na moje zdraví.
\end{xverse}

\begin{xverse}{R.~}
Já chci ...
\end{xverse}

\begin{xverse}{3.~}
Dnes už jsem starší a vím, co vím,
mnohé věci nemůžu a mnohé smím,
a když je mi velmi smutno, lehnu si do mokré trávy,
s nohama křížem a s rukama za hlavou
koukám nahoru na oblohu modravou,
kde se mezi mraky honí moje strakaté krávy.
\end{xverse}

\begin{xverse}{R.~}
Chtěl bych mít ...
\end{xverse}

\end{song}

\begin{song}{Hotel Hillary}{Poutníci}

\begin{xverse}{1.~}
Tvař se [\large Ami]trochu nostalgicky, už tě nikdy nepotkám, [\large Emi|]{}
[\large F]máš to jistý [\large G]provždycky, nastav [\large Ami]uši vzpomínkám,
jak tě znám, i v tuhle chvíli měl bys řeči peprný, [\large Emi|]{}
jak [\large F]tenkrát, když nám [\large G]tvrdili, že je [\large Ami]vítr stříbrný.
\end{xverse}

\begin{xverse}{R.~}
A [\large F]tváře měli kožený, my jim zdrhli z průvodu,
zaho[\large Dmi]dili lampióny a [\large D]našli hospodu,
ale [\large F]taky Jacquese Brela a s ním smutek z cizích vin
a [\large Dmi]{žádo}stivost těla a pak [\large D]radost z volovin,
a ta nám [\large Ami]zbejvá.
\end{xverse}

\begin{xverse}{2.~}
Po večerech pro diváky dělali jsme kašpary,
pak na zemi dva spacáky - náš Hotel Hillary,
slavný sliby jsme už znali, i to, jak se neplní,
a cenzoři nám kázali o správným umění.
\end{xverse}


\begin{xverse}{R.~}
A tváře měli kožený ...
\end{xverse}


\begin{xverse}{3.~}
A tak válčím s nostalgií, bují ve mně jako mech,
a pořád všechno slibují starý hesla na domech,
ty jsi splatil všechny dluhy, i za Hotel Hillary,
a já vyhážu ty černý stuhy funebrákům navzdory.
\end{xverse}


\begin{xverse}{R.~}
Vždyť mají tváře kožený, my jim zdrhnem z průvodu,
zahodíme lampióny a najdem hospodu,
a tam svýho Jacquese Brela a s ním smutek z cizích vin
a žádostivost těla a pak radost z volovin,
/: a ta nám zbejvá. :/
\end{xverse}
\end{song}

\begin{song}{Hudsonský šífy}{Wabi Daněk}

\begin{xverse}{1.~}
Ten, kdo [\large Ami]nezná hukot vody lopat[\large C]kama vířený
jako [\large G]já, jó, jako [\large Ami]já,
kdo hudsonský slapy nezná sírou [\large G]pekla sířený,
ať se [\large Ami]na hudsonský [\large G]{}šífy najmout [\large Ami_]{dá}, [\large G]joho[\large Ami]ho.
\end{xverse}

\begin{xverse}{2.~}
Ten, kdo nepřikládal uhlí, šíf když na mělčinu vjel,
málo zná, málo zná,
ten, kdo neměl tělo ztuhlý, až se nočním chladem chvěl,
ať se na hudsonský šífy najmout dá, johoho.
\end{xverse}

\begin{xverse}{R.~}
A[\large F]hoj, páru tam [\large Ami]hoď,
ať [\large G]do pekla se dříve dohra[\large Ami]bem,
[\large G]joho[\large Ami]ho,  [\large G]joho[\large Ami]ho.
\end{xverse}

\begin{xverse}{3.~}
Ten, kdo nezná noční zpěvy zarostenejch lodníků
jako já, jó, jako já,
ten, kdo cejtí se bejt chlapem, umí dělat rotyku,
ať se na hudsonský šífy najmout dá, johoho.
\end{xverse}

\begin{xverse}{4.~}
Ten, kdo má na bradě mlíko, kdo se rumem neopil,
málo zná, málo zná,
kdo necejtil hrůzu z vody, kde se málem utopil,
ať se na hudsonský šífy najmout dá, johoho.
\end{xverse}

\begin{xverse}{R.~}
Ahoj, páru tam hoď ...
\end{xverse}

\begin{xverse}{5.~}
Kdo má roztrhaný boty, kdo má pořád jenom hlad
jako já, jó, jako já,
kdo chce celý noci čuchat pekelnýho vohně smrad,
ať se na hudsonský šífy najmout dá, johoho.
\end{xverse}

\begin{xverse}{6.~}
Kdo chce zhebnout třeba zejtra, komu je to všechno fuk,
kdo je sám, jó, jako já,
kdo má srdce v správným místě, kdo je prostě príma kluk,
ať se na hudsonský šífy najmout dá, johoho.
\end{xverse}

\end{song}

\begin{song}{Jarní tání}{Brontosauři}

\begin{xverse}{1.~}
Když první [\large Ami]tání [\large Dmi]cestu sněhu [\large C]zkříží
[\large F]a nad [\large Dmi]ledem se [\large E]voda obje[\large Ami]ví,
voňavá zem se [\large Dmi]sněhem tiše [\large C]plíží,
[\large F]tak nějak [\large Dmi]líp si [\large E]balím, proč, Bůh [\large Ami]ví.
\end{xverse}

\begin{xverse}{R.~}
Přišel čas [\large F]slunce, zrození a [\large C]tratí,
na kterejch [\large F]potkáš kluky ze všech [\large C]stran, [\large E]{}
/: Hubenej [\large Ami]Joe, Čára, Ušoun se ti [\large Dmi]vrátí,
oživne [\large F]kemp, [\large E]jaro, vítej k [\large Ami]nám. :/
\end{xverse}

\begin{xverse}{2.~}
Kdo ví, jak voní země, když se budí,
pocit má vždy, jak zrodil by se sám,
jaro je lék na řeči, co nás nudí,
na lidi, co chtěj' zkazit život nám.
\end{xverse}

\begin{xverse}{R.~}
Přišel čas slunce ...
\end{xverse}

\begin{xverse}{3.~}
Zmrznout by měla, kéž by se tak stalo,
srdce těch pánů, co je jim vše fuk,
pak bych měl naději, že i příští jaro
bude má země zdravá jako buk.
\end{xverse}

\begin{xverse}{R.~}
Přišel čas slunce ...
+ oživne [\large F]kemp, [\large E]jaro, vítej [\large A]k nám ...
\end{xverse}

\end{song}

\begin{song}{Jdem zpátky do lesů}{Žalman}

\begin{xverse}{1.~}
[\large Ami7]Sedím na kolejích, [\large D]které nikam neve[\large G]dou, [\large C|]{} [\large G|]{}
[\large Ami7]koukám na kopretinu, jak [\large D]miluje se s lebe[\large G]dou, [\large C|]{} [\large G|]{}
[\large Ami7]mraky vzaly slunce [\large D]zase pod svou ochra[\large G]nu, [\large Emi|]{}
[\large Ami7]jen ty nejdeš, holka zlatá, [\large D]kdypak já tě dosta[\large G]nu? [\large D|]{}
\end{xverse}

\begin{xverse}{R.~}
Z [\large G]ráje, my vyhnaní z [\large Emi]ráje,
kde není už [\large Ami7]místa, [\large C7]prej něco se [\large G]chystá, [\large D|]{}
z [\large G]ráje nablýskaných [\large Emi]plesů
jdem zpátky do [\large Ami7|]{lesů} [\large C7]{}za nějaký [\large G]{}čas.
\end{xverse}

\begin{xverse}{2.~}
Vlak nám včera ujel ze stanice do nebe,
málo jsi se snažil, málo šel jsi do sebe,
šel jsi vlastní cestou, a to se zrovna nenosí,
i pes, kterej chce přízeň, napřed svýho pána poprosí.
\end{xverse}

\begin{xverse}{R.~}
Z ráje...
\end{xverse}

\begin{xverse}{3.~}
Už tě vidím z dálky, jak máváš na mě korunou,
a jestli nám to bude stačit, zatleskáme na druhou,
zabalíme všechny, co si dávaj' rande za branou,
v ráji není místa, možná v pekle se nás zastanou.
\end{xverse}

\begin{xverse}{R.~}
Z ráje...
\end{xverse}

\end{song}
% \chords{ \chordAmiSeven }

% \begin{song}{Jednou mi fotr povídá}{Ivan Hlas}
%
% \begin{xverse}{1.~}
% [\large A7]Jednou mi fotr povídá, [\large D7]zůstali jsme už sami dva,
% že [\large E7]si chce začít taky trochu [\large A7]{žít},
% nech si to projít palicí a nevracej se s vopicí,
% snaž se mě hochu trochu pochopit.
% \end{xverse}
%
%
% \begin{xverse}{R.~}
% Já [\large E7]{šel}, šel dál, baby, [\large A7]kam mě Pánbůh zval,
% já [\large E7]{šel}, šel dál, baby, a [\large D7]furt jen tancoval,
% [\large A7]na každý divný hranici, [\large D7]na policejní stanici
% [\large E7]hrál jsem jenom rock'n'roll for [\large A7]you.
% \end{xverse}
%
%
% \begin{xverse}{2.~}
% Přiletěl se mnou černej čáp, zobákem dělal klapy klap
% a nad kolíbkou Elvis Presley stál,
% obrovskej bourák v ulici, po boku krásnou slepici
% a lidi šeptaj: přijel ňákej král.
% \end{xverse}
%
%
% \begin{xverse}{R.~}
% Já šel, šel dál, baby, kam mě Pánbůh zval, ...
% \end{xverse}
%
%
% \begin{xverse}{3.~}
% Pořád tak ňák nemohu, chytit štěstí za nohu
% a nemůžu si najít klidnej kout,
% bláznivý ptáci začnou řvát a nový ráno šacovat
% a do mě pustí vždycky silnej proud.
% \end{xverse}
%
%
% \begin{xverse}{R.~}
% Já šel, šel dál, baby, kam mě Pánbůh zval, ...
% \end{xverse}
%
% \end{song}

\begin{song}{Kdysi a kdesi}{Šlitr/Suchý}

\begin{xverse}{1.~}
[\large C]Kdysi a kdesi [\large F]bylo nebylo,
[\large G]minomety metaly a [\large C]dělo pálilo,
pan velitel roty na to [\large F]nebral ohledy,
[\large G]{řek'}, abych si obul boty [\large C]{a šel} na zvědy.
\end{xverse}


\begin{xverse}{R.~}
[\large C]Vyfasujem kvér a flašku džinu,
skrze tmu si [\large G]tunel vydla[\large C]bem,
přes Waterloo za Hercegovinu,
podél Mississippi až do [\large G]{Ústí} nad La[\large C]bem.
\end{xverse}


\begin{xverse}{2.~}
I vyšel jsem za malou chvíli směrem k severu,
aby Turci netušili, že je nežeru,
že mám bodák na bodání, pažbu k bušení,
Taliáni nemaj' zdání ani tušení.
\end{xverse}


\begin{xverse}{R.~}
Vyfasujem kvér a flašku džinu ...
\end{xverse}


\begin{xverse}{3.~}
V zákopech si Němci tiše seděli,
aniž tu neděli o mě něco věděli,
času bylo málo a mě to hnalo tam,
kde se zdálo, že Tatarům hlavu zamotám.
\end{xverse}


\begin{xverse}{R.~}
Vyfasujem kvér a flašku džinu ...
\end{xverse}

\begin{xverse}{4.~}
Švédové si právě pekli vepřový,
když tu jsem na ně náhle vyběh' ze křoví,
jejich jediná mě střela minula,
a tak jsem tu bitvu v Kentu vyhrál tři-nula.
\end{xverse}

% \begin{xverse}{R.~}
% Vyfasujem kvér a flašku džinu ...
% \end{xverse}

\end{song}

\begin{song}{Kláda}{Hop trop}

\begin{xverse}{1.~}
[\large Hmi]Celý roky prachy jsem si skládal,
[\large D]nikdy svýho [\large A]floka nepro[\large Hmi]pil,
vod lopaty měl vohnutý záda,
[\large D]paty od baráku [\large A]jsem neodle[\large Hmi]pil,
[\large A]nikdo neví, do čeho se [\large Hmi]někdy zamotá,
tohle [\large D]já už [\large A]dávno pocho[\large Hmi]pil.
\end{xverse}


\begin{xverse}{2.~}
Taky kdysi vydělat jsem toužil,
brácha řek' mi, že by se mnou šel,
tak jsem háky, lana, klíny koupil,
a sekyru jsme svoji každej doma měl,
a plány veliký, jak fajn budem se mít,
nikdo z nás pro holku nebrečel.
\end{xverse}


\begin{xverse}{R.~}
[\large G]Duní [\large D]kláda kory[\large Emi]tem, bacha [\large Hmi]dej, [\large A]hej, bacha [\large Hmi]dej!
S [\large G]tou si, [\large D]bráško, nety[\large Emi]kej, nety[\large F#]kej !
\end{xverse}


\begin{xverse}{3.~}
Dřevo dostat k pile, kde ho koupí,
není těžký, vždyť jsme fikaný,
ten rok bylo jaro ale skoupý,
a teď jsme na dně my i vory svázaný,
a k tomu můžem říct jen, že nemáme nic,
jen kus práce nedodělaný.
\end{xverse}


\begin{xverse}{R.~}
Duní kláda ...
\end{xverse}
\end{song}

\begin{song}{Kluziště}{Karel Plíhal}

\begin{xverse}{1.~}
[\large C]Strejček [\large Emi7/H|]{kovář} [\large Ami7]chytil k[\large C/G]leště,[\large Fmaj7] uštíp' z [\large C]noční [\large Fmaj7_]{oblo}[\large G]{hy}
[\large C]jednu [\large Emi7/H|]{malou} [\large Ami7]kapku [\large C/G]deště, [\large Fmaj7]ta mu sp[\large C]adla [\large Fmaj7]pod no[\large G]{hy,}
[\large C]nejdřív [\large Emi7/H|]{ale} [\large Ami7|]{chytil} [\large C/G]slinu, [\large Fmaj7]pak šáh' [\large C]kamsi [\large Fmaj7]pro pi[\large G]{vo,}
[\large C]pak při[\large Emi7/H|]{táhl} [\large Ami7]kovad[\large C/G]linu [\large Fmaj7|]{}a ob[\large C]rovský [\large Fmaj7_]{kladi}[\large G]{vo.}
\end{xverse}

\begin{xverse}{R.~}
Zatím [\large C]tři bílé [\large Emi7/H]vrány pě[\large Ami7]kně za se[\large C/G]bou
kolem [\large Fmaj7]jdou, někam [\large C]jdou, do rytm[\large D7]u se kýva[\large G]jí,
tyhle [\large C]tři bílé [\large Emi7/H]{vrány} pěk[\large Ami7]ně za seb[\large C/G]ou
kolem [\large Fmaj7]jdou, někam [\large C]jdou, nedojd[\large Fmaj7]ou, nedo[\large C]jdou.
\end{xverse}

\begin{xverse}{2.~}
Vydal z hrdla mocný pokřik ztichlým letním večerem,
pak tu kapku všude rozstřík' jedním mocným úderem,
celej svět byl náhle v kapce a vysoko nad námi
na obrovské mucholapce visí nebe s hvězdami.
\end{xverse}

\begin{xverse}{R.~}
Zatím tři bílé vrány ...
\end{xverse}

\begin{xverse}{3.~}
Zpod víček mi vytrysk' pramen na zmačkané polštáře,
kdosi mě vzal kolem ramen a políbil na tváře,
kdesi v dálce rozmazaně strejda kovář odchází,
do kalhot si čistí ruce umazané od sazí.
\end{xverse}
\end{song}
\chords{ \chordAmiSeven \chordFmajSeven }

\begin{song}{Kometa}{Jarek Nohavica}

\begin{xverse}{1.~}
[\large Ami]Spatřil jsem kometu, oblohou letěla,
chtěl jsem jí zazpívat, ona mi zmizela,
[\large Dmi]zmizela jako laň [\large G7]u lesa v remízku,
[\large C]v očích mi zbylo jen [\large E7]pár žlutých penízků.
\end{xverse}

\begin{xverse}{2.~}
Penízky ukryl jsem do hlíny pod dubem,
až příště přiletí, my už tu nebudem,
my už tu nebudem, ach, pýcho marnivá,
spatřil jsem kometu, chtěl jsem jí zazpívat.
\end{xverse}

\begin{xverse}{R.~}
[\large Ami]O vodě, o trávě, [\large Dmi]o lese,
[\large G7]o smrti, se kterou smířit [\large C]nejde se,
[\large Ami]o lásce, o zradě, [\large Dmi]o světě
[\large E]a o všech lidech, co [\large E7]kdy žili na téhle [\large Ami]planetě.
\end{xverse}

\begin{xverse}{3.~}
Na hvězdném nádraží cinkají vagóny,
pan Kepler rozepsal nebeské zákony,
hledal, až nalezl v hvězdářských triedrech
tajemství, která teď neseme na bedrech.
\end{xverse}

\begin{xverse}{4.~}
Velká a odvěká tajemství přírody,
že jenom z člověka člověk se narodí,
že kořen s větvemi ve strom se spojuje
a krev našich nadějí vesmírem putuje.
\end{xverse}

\begin{xverse}{R.~}
Na na na ...
\end{xverse}


\begin{xverse}{5.~}
Spatřil jsem kometu, byla jak reliéf
zpod rukou umělce, který už nežije,
šplhal jsem do nebe, chtěl jsem ji osahat,
marnost mne vysvlékla celého donaha.
\end{xverse}


\begin{xverse}{6.~}
Jak socha Davida z bílého mramoru
stál jsem a hleděl jsem, hleděl jsem nahoru,
až příště přiletí, ach, pýcho marnivá,
my už tu nebudem, ale jiný jí zazpívá.
\end{xverse}

\begin{xverse}{R.~}
O vodě, o trávě, o lese,
o smrti, se kterou smířit nejde se,
o lásce, o zradě, o světě,
bude to písnička o nás a kometě ...
\end{xverse}

\end{song}

% \begin{song}{Král a klaun}{Karel Kryl}
%
% \begin{xverse}{1.~}
% [\large D]Král [\large C]do boje [\large G]táh',[\large C][\large G] do [\large C]veliké [\large G]dálky,[\large C|]{} [\large G|]{}
% a s [\large C]ním do té [\large G]války [\large D7]jel na mezku [\large G]klaun,
% [\large D]než [\large C]hledí si [\large G]stáh' [\large C] [\large G] , tak z [\large C]výrazu [\large G]tváře [\large C|]{} [\large G|]{}
% [\large C]bys nepoznal [\large G]lháře, [\large D7]co zakrývá [\large G]strach.
% [\large D7]Tiše šeptal při té hrůze: "[\large G]Inter arma silent musae,"
% [\large A]místo zvonku cinkal brně[\large D7]ním, [\large C#7|]{} [\large D7|]{}
% [\large C]král do boje [\large G]táh' [\large C] [\large G] , do [\large C]veliké [\large G]dálky, [\large C|]{} [\large G|]{}
% a s [\large C]ním do té [\large G]války [\large D7]jel na mezku [\large G]klaun. [\large H|]{} [\large C|]{} [\large G|]{} [\large A7|]{}
% \end{xverse}
%
% \begin{xverse}{2.~}
% Král do boje táh', a sotva se vzdálil,
% tak vesnice pálil a dobýval měst,
% klaun v očích měl hněv, když sledoval žháře,
% jak smývali v páře prach z rukou a krev.
% Tiše šeptal při té hrůze:"Inter arma silent musae,"
% místo loutny držel v ruce meč,
% král do boje táh', a sotva se vzdálil,
% tak vesnice pálil a dobýval měst.
% \end{xverse}
%
% \begin{xverse}{3.~}
% Král do boje táh', s tou vraždící lůzou
% klaun třásl se hrůzou a odvetu kul,
% když v noci byl klid, tak oklamal stráže
% a, nemaje páže, sám burcoval lid.
% Všude křičel do té hrůzy, ve válce že mlčí Múzy,
% muži by však mlčet neměli,
% král do boje táh', s tou vraždící lůzou
% klaun třásl se hrůzou a odvetu kul.
% \end{xverse}
%
% \begin{xverse}{4.~}
% Král do boje táh', a v červáncích vlídných
% zřel, na čele bídných jak vstříc jde mu klaun,
% když západ pak vzplál, tok potoků temněl,
% klaun tušení neměl jak zahynul král:
% kdekdo křičel při té hrůze:"Inter arma silent musae,"
% krále z toho strachu trefil šlak,
% klaun tiše se smál a zem žila dále
% a neměla krále, klaun na loutnu hrál,
% [\large D7]klaun na loutnu [\large G]hrál, [\large D7]klaun na loutnu [\large G]hrál ...
% \end{xverse}
%
% \end{song}
% \chords{\chordCisSeven}

\begin{song}{Krutá válka}{Spirituál kvintet}

\begin{xverse}{1.~}
Tmou [\large E]zní zvony [\large C#mi]z dálky, o [\large F#mi]{}čem to, milý, [\large G#mi]sníš,
[\large G#]hoří [\large A]dál plamen [\large F#mi]války a [\large E]rá[\large A6]no je [\large F#mi]blíž,
[\large H7]chci [\large E]být stále s [\large C#mi]tebou, až [\large F#mi]trubka začne [\large G#mi]znít,
[\large G#]lásko [\large A]má, vem mě s [\large F#mi]sebou! [\large E]Ne, to [\large A]nesmí [\large E]být!
\end{xverse}

\begin{xverse}{2.~}
Můj šál skryje proud vlasů, na pás pak připnu nůž,
poznáš jen podle hlasu, že já nejsem muž,
tvůj kapitán tě čeká, pojď, musíme už jít,
noc už svůj kabát svléká ... Ne, to nesmí být!
\end{xverse}

\begin{xverse}{3.~}
Až dým vítr stočí, tvář změní pot a prach,
do mých dívej se očí, tam není strach,
když výstřel tě raní, kdo dával by ti pít,
hlavu vzal do svých dlaní ... Ne, to nesmí být!
\end{xverse}

\begin{xverse}{4.~}
Ach, má lásko sladká, jak mám ti to jen říct,
každá chvíle je krátká a já nemám víc,
já mám jenom tebe, můj dech jenom tvůj zná,
nech mě jít vedle sebe ... Pojď, lásko má!
\end{xverse}

\end{song}
\chords{\chordASix}

\begin{song}{Krysař}{Pacifik}

\begin{xverse}{1.~}
[\large Emi]Bylo to v dobách [\large C]osvícených, před [\large D]branou krysař [\large Emi]stál,
[\large Emi]městem šlo jako [\large Ami]pohlazení, když [\large D]na píšťalu [\large Emi]hrál.
[\large G]Viděl jak brány [\large C]otvírají, [\large Ami]jak každý šel mu [\large D]vstříc,
[\large Emi]{že} lidé jeho [\large Ami]píseň znají, [\large D]každý chtěl slyšet [\large Emi]víc.
[\large Emi|]{ } [\large C|]{ } [\large G|]{ } [\large D|]{ } [\large Emi|]{ }
\end{xverse}


\begin{xverse}{2.~}
Bylo to v dobách osvícených, před branou krysař stál,
až jednou se svým doprovodem přišel i sám pan král,
prodej mi flétnu, chlapče milý, já všechno zlato ti za ni dám,
ten nápěv tolik roztomilý, dávno v srdci mám.
\end{xverse}


\begin{xverse}{3.~}
Povídá krysař: pane králi, ať píšťalka je tvou,
ať v kámen nikdy nepromění písničku nevinnou.
Vyjdi s tou písní mezi lidi, na každou ze všech cest,
ať slepý rázem krásu vidí, otvírej brány měst.
\end{xverse}


\begin{xverse}{4.~}
Otvírej srdce zatvrzelá, tulákům lámej hůl,
ať s tebou zpívá země celá, dej království všem půl.
Krejčíkům plátno, rybářům síť a včelám květů pyl,
já zase musím svou cestou jít, ty zpívej ze všech sil.
\end{xverse}


\begin{xverse}{5.~}
Bylo to v dobách osvícených, před branou krysař stál,
městem šlo jako pohlazení, když na píšťalu hrál.
Kde jsou ty doby osvícené, zatímco svět šel dál,
kde jsou ty písně zanícené, kam zmizel ten, co hrál.
\end{xverse}
\end{song}

\begin{song}{Kulatý vobdélníky}{Hop trop}

\begin{xverse}{R.~}
/: Kulatý [\large D]obdélníky, kulatý obdélníky,
fialovej [\large A7]les a žlutá [\large D]voda. :/
\end{xverse}

\begin{xverse}{1.~}
[\large D]Pojď se mnou, ty moje poupě,
já [\large A7]ukážu ti opiový [\large D]doupě,
tam v těžkým dýmu omamnejch jedů
uvidíš [\large A7]fialovej les a žlutou [\large D]vodu.
\end{xverse}

\begin{xverse}{R.~}
Kulatý obdélníky ...
\end{xverse}

\begin{xverse}{2.~}
Ležím si na břiše, na zádech bednu kytu,
v kapse hrst hašiše, žiju si v blahobytu,
dva kufry algeny dostal jsem za chatu
a potom za auťák LSD lopatu.
\end{xverse}

\begin{xverse}{R.~}
Kulatý obdélníky ...
\end{xverse}

\begin{xverse}{3.~}
Fenmetrák posvačím, čuchnu si čikuli,
mám z toho čistidla frňák jak bambuli,
konečně v kómatu rysy mi přituhly,
sako a kravatu dají mi do truhly.
\end{xverse}

\begin{xverse}{R.~}
Kulatý obdélníky ...
\end{xverse}

\end{song}

\begin{song}{Lodníkův lament}{Hop trop}

\begin{xverse}{1.~}
[\large Emi]Já snad [\large G]hned, když jsem se [\large D]narodil,
na [\large G]bludnej [\large D]kámen [\large G]{šláp'},
a do školy moc [\large D]nechodil, i [\large Emi]tak je [\large D]ze mě [\large Emi]chlap,
velký [\large G]dusno, který [\large D]nad hlavou mi [\large G]doma [\large D]vise[\large G]lo,
drsnýmu chlapu [\large D]nesvědčí,
já [\large Emi]{ťuk'} si [\large D]na če[\large Emi]lo.
\end{xverse}


\begin{xverse}{R.~}
[\large D|]{} [\large G|]{} [\large D|]{} [\large G|]{} [\large C|]{} [\large G|]{}
Má[\large G]ma mě doma držela a [\large D]táta na mě dřel,
já moh' jsem jít hned študovat, kdy[\large G]bych jen trochu chtěl,
voženit se, vzít si ňákou [\large D]trajdu copatou
a za její lásku platit [\large G]celou vejplatou, hó [\large Emi]hou.
\end{xverse}


\begin{xverse}{2.~}
Potom do knajpy jsem zašel a tam uslyšel ten žvást,
že na lodích je veselo a fasujou tam chlast,
a tak honem jsem se nalodil na starej vratkej křáp,
kde kapitán byl kořala a řval na nás jak dráb.
\end{xverse}


\begin{xverse}{3.~}
Vlny s kocábkou si házely a každej dostal strach
a my lodníci se vsázeli, kdo přežije ten krach,
všechny krysy z lodi zmizely a v dálce maják zhas'
a první byl hned kapitán, kdo měl korkovej pás.
\end{xverse}


\begin{xverse}{4.~}
Kolem zubama už cvakali žraloci hladoví,
moc nikomu se nechtělo do vody ledový,
k ránu bouře trochu ustala, já mořskou nemoc měl,
všem, co můžou chodit po zemi, jsem tolik záviděl.
\end{xverse}

\begin{xverse}{5.~}
Jako zázrakem jsme dojeli, byl každý živ a zdráv
a všichni byli veselí, jen já jsem rukou máv',
na loď nikdy víc už nevlezu, to nesmí nikdo chtít,
teď lituju a vzpomínám, jak jen jsem se moh' mít.
\end{xverse}

\end{song}

\begin{song}{Louisiana}{Hop trop}

\begin{xverse}{1.~}
Ten, [\large Emi]kdo by jednou chtěl bejt vopravdickej chlap
a na [\large G]{šífu} křížit [\large D]svět ho nele[\large Emi]ká,
teď příležitost má a stačí, aby se jí drap',
ať [\large G]na tu chvíli [\large D]dlouho neče[\large Emi]ká.
\end{xverse}


\begin{xverse}{R.~}
Louisi[\large G_]{a}[\large D]{na}, Louisi[\large G_]a[\large D]na [\large G]zná už [\large D]dálky modra[\large A]vý, [\large Emi|]{}
[\large G]bí[\large D]lá Louisi[\large G_]a[\large D]na, jako [\large G]víra pevná [\large D]loď,
podepiš a s náma [\large Emi]pojď, taky hned si z bečky [\large Hmi]nahni na zdra[\large Emi]ví.
\end{xverse}


\begin{xverse}{2.~}
Jó, tady každej z nás má ruku k ruce blíž,
když to musí bejt, i do vohně ji dá,
proti nám je pracháč i kostelní myš,
nám stačí dejchat volně akorát.
\end{xverse}


\begin{xverse}{R.~}
Louisiana, Louisiana ...
\end{xverse}


\begin{xverse}{3.~}
Až budem někde dál, kde není vidět zem,
dvě hnáty křížem vzhůru vyletí,
zas bude Černej Jack smát se nad mořem,
co je hrobem jeho obětí.
\end{xverse}

\begin{xverse}{R.~}
Louisiana, Louisiana ...
\end{xverse}

\end{song}

\begin{song}{Malý velký muž}{Pacifik}


\begin{xverse}{1.~}
Dokud [\large Emi]tráva bude růst
Řeky potečou a [\large C]stoupat bude dým
Léta utečou a [\large D]kam padne tvůj stín
Země tvá bude [\large Emi]tvou

Dokud [\large Ami]noci střídá den
vítr bude vát a [\large G]mraky poplujou
Slunce bude plát a [\large H7]tak jak léta jdou
Země tvá bude [\large Emi]tvou
\end{xverse}


\begin{xverse}{R.~}
Jen [\large G]malý velký muž
tolik dobře věděl [\large Emi]co je vostrej nůž
smutek prázdnych sedel
[\large C]malý velký muž čekal svý zname[\large D]ní
Jen [\large G]malý velký muž
žehnal ohni sílu, [\large Emi]z rudých kamenů
vítal dýmku míru, [\large C]přesto pohřbil sen
velký [\large D]sen u Wounded [\large Emi]Knee
\end{xverse}


\begin{xverse}{2.~}
Dokud tráva bude růst
ruce špinavý až v plání vztyčí kříž
řeky zastaví se, plakat uslyšíš
Slunce zář krvavou

Dokud noci střídá den
slova neplatí, a co je vlastně jen
ono prokletí, co padlo na tvou zem
na tvou zem ztracenou
\end{xverse}


\begin{xverse}{R.~}
Jen malý ...
\end{xverse}

\begin{xverse}{3.~}
Dokud tráva bude růst
rány nezhojí a neopláchne déšť
řeky nespojí se v jeden silný proud
silný proud nadějí ..

Dokud noci střídá den
srdce zlomená, a jejich dávný sen
skalní ozvěna už nevrátí tvou zem
tu tvou zem ztracenou
\end{xverse}

\begin{xverse}{R.~}
Jen malý ...
\end{xverse}
\end{song}

\begin{song}{Mississippi blues}{Pacifik}


\begin{xverse}{1.~}
[\large Ami]{Ří}kali mu Charlie a [\large Dmi]jako každej kluk
[\large Ami]kalhoty si [\large G]o plot potr[\large Ami]hal,
říkali mu Charlie a [\large Dmi]byl to Toma vnuk,
[\large Ami]na plácku rád [\large G]košíkovou [\large Ami]hrál,
[\large C]křídou kreslil po ohradách [\large F]plány dětskejch snů,
[\large Dmi]až mu jednou ze tmy řekli: [\large E]konec je tvejch dnů,
[\large Ami]někdo střelil zezadu a [\large Dmi]vrub do pažby vryl,
nikdo [\large Ami]neplakal a [\large G]nikdo nepro[\large Ami]sil.
\end{xverse}


\begin{xverse}{R.~}
Missis[\large C]sippi, Missis[\large Ami]sippi, [\large F]{čer}ný tělo [\large G]nese říční [\large C]proud,
Mississippi, Missis[\large Ami]sippi, [\large F]po ní bude [\large G]jeho duše [\large C]plout. [\large Ami|]{}
\end{xverse}


\begin{xverse}{2.~}
Říkali mu Charlie a jako každej kluk
na trubku chtěl ve smokingu hrát,
v kapse nosil kudlu a knoflíkovej pluk,
uměl se i policajtům smát,
odmalička dobře věděl, kam se nesmí jít,
který věci jinejm patří a co sám může mít,
že si do něj někdo střelí jak do hejna hus,
netušil, a teď mu řeka zpívá blues.

\end{xverse}

\begin{xverse}{R.~}
Mississippi, Mississippi ...
\end{xverse}


\begin{xverse}{3.~}
Chlapec jménem Charlie, a jemu patří blues,
ve kterým mu táta sbohem dal,
chlapec jménem Charlie snad ušel cesty kus,
jako slepý na kolejích stál,
nepochopí jeho oči, jak se může stát,
jeden že má ležet v blátě, druhej klidně spát,
jeho blues se naposledy řekou rozletí,
kdo vyléčí rány, smaže prokletí.
\end{xverse}

\begin{xverse}{R.~}
Mississippi, Mississippi ...
\end{xverse}

\end{song}

% \begin{song}{Mladičká básnířka}{Jarek Nohavica}
%
% \begin{xverse}{1.~}
% [\large G]Mladičká básnířka s [\large Hmi]korálky nad kotníky [\large Emi|]{} [\large D|]{}
% [\large G]bouchala na dvířka [\large Hmi]paláce poetiky,  [\large Emi|]{} [\large D|]{}
% s někým se [\large G]vyspala, někomu [\large Hmi]nedala,láska jako [\large Emi]hobby,
% [\large Cmi]pak o tom napsala [\large D]blues na čtyři [\large G]doby. [\large Hmi|]{} [\large Emi|]{} [\large D|]{}
% \end{xverse}
%
% \begin{xverse}{2.~}
% Své srdce skloňovala podle vzoru Ferlinghetti,
% ve vzduchu nechávala viset vždy jen půlku věty,
% plná tragiky, plná mystiky, plná spleenu,
% pak jí to otiskli v jednom magazínu, ho ho hó.
% \end{xverse}
%
% \begin{xverse}{3.~}
% Bývala viděna v malém baru u rozhlasu,
% od sebe kolena a cizí ruka kolem pasu,
% trochu se napila, trochu se opila na účet redaktora
% a týden nato byla hvězdou Mikrofóra, ho ho hó.
% \end{xverse}
%
% \begin{xverse}{4.~}
% Pod paží nosila rozepsané rukopisy,
% ráno se budila vedle záchodové mísy,
% životem potřísněná, můzou políbená, plná zázraků
% a pak ji vyhodili z gymplu a hned nato i z baráku, ho ho hó.
% \end{xverse}
%
% \begin{xverse}{5.~}
% Ve třetím měsíci dostala chuť na jahody,
% ale básníci-tatíci nepomýšlej' na rozvody,
% cítila u srdce, jak po ní přešla železná bota,
% tak o tom napsala sonet, a ten byl ze života.
% \end{xverse}
%
% \end{song}

\begin{song}{Mlýny}{Spirituál kvintet}

\begin{xverse}{R.~}
[\large G]Slyším mlýnský kámen, jak se otáčí,
[\large C]slyším mlýnský kámen, jak se otá[\large G]{čí},
já slyším mlýnský kámen, [\large H7]jak se otá[\large Emi]{čí},
[\large C_]o[\large D]tá[\large G]{čí}, otá[\large D]{čí}, otá[\large C]{čí}.
\end{xverse}

\begin{xverse}{1.~}
Ty mlýny [\large G]melou celou [\large C]noc a melou [\large G]celý den,
melou [\large C7]bez výhod a melou [\large G]stejně všem,
melou doleva [\large C]jen a melou [\large G]doprava,
melou [\large A]pravdu i lež, když zrovna [\large D]vyhrává,
melou [\large G]otro[\large C]káře, melou [\large G]otroky,
melou [\large C]na minuty, na hodiny, [\large G]na roky,
melou [\large H7]pomalu a jistě, ale [\large Emi]melou [\large C]včas,
já už [\large G]slyším [\large D7]jejich [\large G]hlas.
\end{xverse}

\begin{xverse}{R.~}
Slyším mlýnský kámen ...
\end{xverse}

\begin{xverse}{2.~}
Ó, já, chtěl bych aspoň na chvíli být mlynářem,
pane, já bych mlel, až by se chvěla zem,
to mi věřte, uměl bych dobře mlít,
já bych věděl komu ubrat, komu přitlačit,
ty mlýny čekají někde za námi, až zdola zazní naše volání,
až zazní jeden lidský hlas: no tak už melte, je čas!
\end{xverse}

\begin{xverse}{R.~}
Slyším mlýnský kámen ...
\end{xverse}

\end{song}

\begin{song}{Mrtvej vlak}{Hoboes}

\begin{xverse}{1.~}
[\large Ami]Znáš tu trať, co jezdit po ní [\large Dmi]je tak zrovna k zbláznění[\large Ami],
v semaforu místo lampy [\large Dmi]svítěj' kosti zkříže[\large E7]ný,
[\large Dmi]pták tam zpívat zapomněl a [\large F]vítr jenom v drátech [\large E7]zní,
[\large Ami]jednou za rok touhle tratí [\large Dmi]zaduní vlak pohřební[\large Ami].
\end{xverse}

\begin{xverse}{2.~}
Po koleji rezavý, tam, kde jsou mosty zřícený,
bez páry a bez píšťaly, kotle dávno studený,
nikdo lístky neprohlíží, s brzdou je to zrovna tak,
s pavučinou místo kouře jede nocí mrtvej vlak.
\end{xverse}

\begin{xverse}{R.~}
Mrtvej [\large Dmi]vlak, mrtvej [\large F+]vlak nedr[\large Ami]{}ží jízdní [\large Ddim]{}řád,
dálku [\large Ami]máš přece [\large Ddim|]{rád,} [\large E7]nase[\large Ami]dat,
neměj [\large Dmi]strach, ve ska[\large F+]lách zadu[\large Ami]ní mrtvej [\large Ddim]vlak,
chceš mít [\large Ami]klid, máš ho [\large Ddim]mít, už jede [\large Ami]vlak.
\end{xverse}

\begin{xverse}{3.~}
V životě jsi neměl prachy, zato jsi měl řádnej pech,
kamarádi pochcípali v sakra nízkejch tunelech,
že jsi zůstal sám a že jsi jenom hobo ubohej,
zasloužíš si za to všechno aspoň funus fajnovej.
\end{xverse}

\begin{xverse}{4.~}
Jednou vlezeš pod vagón a budeš to mít hotový,
kam jsi tímhle vlakem odjel, nikdo už se nedoví,
slunce tady nevychází, cesty zpátky nevedou,
ďábel veksl přehodí a stáhne šraňky za tebou.
\end{xverse}

\begin{xverse}{R.~}
Mrtvej vlak ...
už jede [\large Ami]vlak, už jede vlak ...
\end{xverse}

\end{song}
% \chords{ \chordDdim \chordFplus }

\begin{song}{Na cestě - On the Road}{Wabi Daněk}


\begin{xverse}{1.~}
[\large G]Kdysi u silnice [\large D]stával, deku do půl [\large G]zad,
ať si mával, jak si [\large D]mával, nechtěli ho [\large G]brát,
[\large C]nikdy [\large G/H]Kerouaca [\large Ami]nečet' a [\large E]neznal třetí [\large Ami]proud,
[\large C]přesto [\large G]býval [\large D]spolu s [\large G]Deanem [\large D]každej [\large G]víkend [\large D]on the [\large G]road.
\end{xverse}

\begin{xverse}{2.~}
Nikdy neměl ani zdání, jak se hrával bop,
měl jen slinu na toulání a překážel mu strop,
životem na plný pecky a neubírat plyn,
tuhle víru na svý pouti vždycky vzýval Sal i Dean.
\end{xverse}

\begin{xverse}{R.~}
[\large C]Tak mi [\large G]{}řekni, [\large C]na co vlastně [\large G]mám
[\large C]moudrosti [\large G]vyčtený z [\large Ami]knížek,
[\large C]co je [\large G]dobrý, [\large C]na to přijdu [\large G]sám,
[\large C]co je [\large G]{}špatný, za tím [\large F]křížek udě[\large D]lám.
\end{xverse}

\begin{xverse}{3.~}
Tohle na cestě mi říkal, já ho jednou vzal,
potom zavolal jen ``díky'' a já frčel dál,
od těch dob jsem vždycky hlídal, ať kamkoliv jsem jel,
nestojí-li u patníku se svou vírou Dean a Sal.
\end{xverse}

\begin{xverse}{4.~}
Vždycky u silnice stával, vlasy do půl zad,
ať si mával, jak si mával, nechtěli ho brát,
nikdy tuhle knížku nečet' a neznal třetí proud,
přesto býval spolu s Deanem každej víkend on the road,
[\large D]on the [\large G]road ...
\end{xverse}

\end{song}
% \chords{ \chordGH }

\begin{song}{Nebeští jezdci}{Waldemar Matuška}

\begin{xverse}{1.~}
Po [\large Ami]zasmušilé pustině jel [\large C]starý honec krav,
den [\large Ami]temný byl a vítr ševe[\large C]lil ve stéblech trav,
tu [\large Ami]honák k nebi pohleděl a v hrůze zůstal stát,
když z [\large F]rozedraných [\large Ami]oblaků uviděl stádo krav se hnát.
\end{xverse}

\begin{xverse}{R.~}
Jipija [\large C]hej, jipija [\large Ami]hou,to [\large F]přízraky [\large Dmi]táhnou [\large Ami]tmou.
\end{xverse}

\begin{xverse}{2.~}
Ten skot měl nohy z ocele a oči krvavý
a na bocích mu plápolaly cejchy řeřavý.
A oblohou se neslo jeho kopyt dunění
a za ním jeli honáci až k smrti znavení.
\end{xverse}

\begin{xverse}{R.~}
Jipija hej ...
\end{xverse}


\begin{xverse}{3.~}
Ti muži byli sinalí a kalný měli zrak
a marně stádo stíhali,jak mračno stíhá mrak.
A proudy potu máčely jim cáry košilí
a starý honák uslyšel ten jekot kvílivý.
\end{xverse}

\begin{xverse}{R.~}
Jipija hej ...
\end{xverse}

\begin{xverse}{4.~}
Tu jeden z jezdců zastavil a pravil: ``Pozor dej'',
svou duši hříchu vyvaruj a ďáblu odpírej,
bys nemusel se po smrti na věky věků štvát
a nekonečnou oblohou to stádo s náma hnát.
\end{xverse}

\begin{xverse}{R.~}
Jipija hej ...
\end{xverse}

\end{song}

\begin{song}{Nehrálo se o ceny}{Hop trop}

\begin{xverse}{1.~}
[\large Emi]Měli jsme bundy zele[\large Ami]ný, [\large C]někomu občas lezly [\large G]krkem,
[\large Ami]kdekdo si o nás myslel [\large D7]svý, [\large Ami]jako by nikdy nebyl [\large D7]klukem.
\end{xverse}

\begin{xverse}{2.~}
Vod lidí pohled kyselej a kam jet, to nám bylo volný,
každej už hrozně dospělej, i když to věkem bylo sporný.
\end{xverse}

\begin{xverse}{R.~}
Když [\large C]na nádraží při pátku nám čekání se kdysi zdálo [\large G]dlouhý,
víc [\large C]než milión v prasátku bylo nabídnutí cigarety [\large G]pouhý,
\end{xverse}

\begin{xverse}{}
tam [\large Ami]vo zábradlí vopřený, dvě kytary a syrovej sbor [\large D7]hlasů,
tam [\large Ami]nehrálo se o ceny, ale pro radost a ukrácení [\large D7]{ča}su.
\end{xverse}

\begin{xverse}{3.~}
Jméno si každej vysloužil a bral ho stejně jako pravý,
vždyť na tom, jakej kdo z nás byl, stálo, jak bude přiléhavý.
\end{xverse}

\begin{xverse}{R.~}
Když na nádraží při pátku ...
\end{xverse}

\begin{xverse}{4.~}
Přesto, že každej jinam šel životem úspěchů i pádů,
/: těžko by asi zapomněl na partu dobrejch kama[\large (G)]rádů. :/
\end{xverse}

\end{song}

\begin{song}{Nejdelší vlak}{Spirituál Kvintet}

\begin{xverse}{1.~}
[\large C]Proud řeky ví, kdy kámen pohla[\large C7]dí,
[\large F]ví, kdy ho zastaví [\large C]hráz, [\large C7|]{}
[\large F]ví, voda ví, kdy ji [\large C]noc [\large C/H]ochla[\large Ami]dí,
a [\large C]zná, jak [\large G]pálí [\large C]mráz.
\end{xverse}


\begin{xverse}{2.~}
Plout s vlnou výš a znát, kde je břeh tvůj,
slůvka tři prostá ti říct,
tam někde v dálce je návěstí ``stůj!'',
ráda tě mám, nic víc.
\end{xverse}


\begin{xverse}{R.~}
[\large C]Ví[\large G]tr [\large Emi]stín [\large F]tvůj [\large G]svál,
[\large F]nejdelší vlak jel [\large Cmaj7]dál,
[\large F]{šest}náct vagónů [\large C]měl, [\large C/H]tenkrát [\large Ami]měl,
v [\large Dmi]posledním z [\large G]nich jsi [\large C]stál.
\end{xverse}


\begin{xverse}{3.~}
Kouř zprávu hlásí: nejdelší vlak tvůj
vrátí se, půjdu mu vstříc,
blízko mých očí je návěstí ``stůj!'',
ráda tě mám, nic víc.
\end{xverse}


\begin{xverse}{R.~}
Vítr stín tvůj svál ...
\end{xverse}

\end{song}
\chords{\chordCmajSeven \chordCH}

\begin{song}{Ohradník}{Hop trop}

\begin{xverse}{1.~}
Už [\large Dmi]sníh se ztrácí [\large G]ze strání a [\large Bb]zem začíná [\large F]{žít,}
jenom [\large Dmi]blejsklo slunce [\large G]do louží, už [\large Bb]parťák na nás [\large F]vlít',
toho flákání prej [\large Bb]po farmě má [\large F]právě ako[\large Bb]rát,
proto: ``[\large Dmi]Chlapi, skočte [\large F]do vozů a [\large G]natáhneme [\large Dmi]drát.''
\end{xverse}


\begin{xverse}{2.~}
Ty auťáky maj shnilej rám a rozrachtanej plech,
dva džípy z války poslední jen stěží chytaj' dech,
kolikrát mi ten můj nechtěl jet, kolikrát bych do něj kop',
ale ohradníky stavím rád, je to náš jarní džob.
\end{xverse}


\begin{xverse}{R.~}
[\large Dmi]Stovky ran [\large Bb]palicí a [\large F]kůly budou [\large C]stát
[\large Bb]pro míle [\large F]dlouhý vede[\large C]ní,
[\large Bb]dvě stopy [\large F]nad zemí pak [\large C]izolátor [\large G]dát,
[\large Dmi]stáda nám [\large F]ohlídají [\large Ami]dráty mědě[\large Dmi]ný.
\end{xverse}


\begin{xverse}{3.~}
Až pak za struny drátěný ten ohradníku drát
bolavý ruce vymění a večer začnou hrát,
budeme si všichni zpívat a spánek okradem,
ale ráno, až se rozední, tak zase dál se hnem.
\end{xverse}

\begin{xverse}{R.~}
Stovky ran palicí a kůly budou stát ...
\end{xverse}

\end{song}

\begin{song}{Oregon / Touha žít}{Pacifik}

\begin{xverse}{1.~}
Kdo [\large Emi]vyhnal tě tam na cestu dalekou --
touha [\large G]{}žít, touha [\large Emi]{}žít,
kdo zboural ti dům a pravdu staletou --
touha [\large G]{}žít, touha [\large Emi]{}žít,
[\large Ami]těžko se ti dejchá v [\large Emi]těsným ovzduší,
[\large Ami]{}že máš hlad a žízeň, to [\large H7]nikdo netuší,
kdo [\large Emi]přes pláně hnal tvůj osamělej vůz --
touha [\large G]{}žít, touha [\large Emi]{}žít.
\end{xverse}

\begin{xverse}{R.~}
Ore[\large C]gon, Ore[\large D]gon, slyšíš, jak v [\large G]dálce bije [\large Emi]zvon,
Ore[\large C]gon, Ore[\large D]gon, slyšíš ho [\large Emi]znít.
\end{xverse}

\begin{xverse}{2.~}
Kdo pár cajků tvých pod plachty naložil --
 touha žít, touha žít,
kdo studenou zbraň ti k líci přiložil --
 touha žít, touha žít,
na týhletý cestě jen dvě možnosti máš:
buďto někde chcípnout, anebo držet stráž,
kdo vysnil ti zem a odvahu ti dal - touha žít, touha žít.
\end{xverse}

\begin{xverse}{R.~}
Oregon, Oregon, slyšíš, jak v dálce bije zvon ...
\end{xverse}

\begin{xverse}{3.~}
Kdo vést bude pluh, až půdu zakrojí --
 touha žít, touha žít,
kdo zavolá den a úly vyrojí --
 touha žít, touha žít,
člověk se drápe až někam k nebi blíž,
dostává rány, a přesto leze výš,
hledej svůj sen, ať sílu neztratí touha žít, touha žít.
\end{xverse}

\begin{xverse}{R.~}
Oregon, Oregon, slyšíš, jak v dálce bije zvon ...
\end{xverse}

\begin{xverse}{R.~}
Oregon, Oregon, nesmíš tu stát jak uschlej strom,
Oregon, Oregon, dál musíš jít.
\end{xverse}

\end{song}

\begin{song}{Outsider waltz}{Wabi Daněk}

\begin{xverse}{1.~}
Dnes [\large G]ráno, když bylo půl, při [\large Hmi]pravidelný hygieně
[\large Ami]poklesls' hodně v ceně, když jsi [\large C]zahlíd' svůj [\large D]zjev,
už [\large Ami]nejsi, co jsi [\large D]býval, tu [\large G]tvář nespraví ti [\large Emi]masáž,
[\large Ami]marně se, hochu, [\large D]kasáš, už nejsi [\large G]lev a velkej [\large D]{}šéf.
\end{xverse}

\begin{xverse}{R.~}
[\large G]Máš svůj svět a [\large Emi]ten se ti hroutí,
to [\large G]dávno znám, já [\large E]prožil to sám,
[\large Ami]kráčíš [\large D]dál a [\large Ami]cesta se [\large D]kroutí,
až [\large Ami]potkáš nás [\large Hmi]na ní, tak   [\large D]přidej se k [\large G]nám.
\end{xverse}

\begin{xverse}{2.~}
Jsi z vojny doma čtrnáct dnů, a na radnici velká sláva,
to se ti holka vdává, cos' jí dva roky psal,
ulicí tiše krouží ten blbej motiv z Lohengrina,
není ta - bude jiná, dopisy spal a jde se dál.
\end{xverse}

\begin{xverse}{R.~}
Máš svůj svět a ten se ti hroutí ...
\end{xverse}

\begin{xverse}{3.~}
Za sebou máš třicet let a zejtra ráno třetí stání
a nemáš ani zdání, jak to potáhneš dál,
ten, komus' kdysi hrával, se znenadání někam ztratil,
už nemáš, čím bys platil, no tak se sbal a šlapej dál.
\end{xverse}

\begin{xverse}{R.~}
Máš svůj svět a ten se ti hroutí ...
\end{xverse}

\end{song}

\begin{song}{Píseň, co mě učil listopad}{Wabi Daněk}

\begin{xverse}{1.~}
[\large G]Málo jím a [\large C]málo spím a [\large G]málokdy tě [\large C]vídám,
[\large G]málokdy si [\large Hmi]nechám něco [\large Ami]zdát, [\large D7]{}
[\large C]doma nemám [\large G]stání [\large G/F#|]{už} [\large Emi]od jarního [\large C]tání,
[\large F]cítím, že se blíží listo[\large G]pad.
\end{xverse}

\begin{xverse}{R.~}
Listopado[\large F]vý písně [\large C]od léta už [\large G]slýchám,
vítr ledo[\large Ami|]{vý} [\large C]přinesl je k [\large G]nám,
tak mě neče[\large F]kej, dneska [\large C]nikam nepos[\large G]píchám,
listopado[\large Ami|]{vý} [\large C]písni naslou[\large G]chám.
\end{xverse}

\begin{xverse}{2.~}
Chvíli stát a poslouchat, jak vítr větve čistí,
k zemi padá zlatý vodopád,
pod nohama cinká to poztrácené listí,
vím, že právě zpívá listopad.
\end{xverse}

\begin{xverse}{R.~}
Listopadový písně ...
\end{xverse}

\begin{xverse}{3.~}
Dál a dál tou záplavou, co pod nohou se blýská,
co mě nutí do zpěvu se dát,
tak si chvíli zpívám a potom radši pískám
píseň, co mě učil listopad.
\end{xverse}

\begin{xverse}{R.~}
Listopadový písně ...
\end{xverse}

\end{song}

% \chords{ \chordGFis }
%
\begin{song}{Pískající cikán}{Spirituál Kvintet}

\begin{xverse}{1.~}
[\large C]Dívka [\large G]loudá se [\large C]vini[\large G|]{cí,} [\large C]tam, kde [\large G]zídka je [\large C_]{níz}[\large G]{ká},
[\large C]tam, kde [\large G]stráň končí [\large C]voní[\large F]cí, si [\large C]písnič[\large F]ku někdo [\large CF_]{pí}[\large G]{ská}.
\end{xverse}

\begin{xverse}{2.~}
Ohlédne se a ``propána!'', v stínu, kde stojí líska,
švarného vidí cikána, jak leží, písničku píská.
\end{xverse}


\begin{xverse}{3.~}
Chvíli tam stojí potichu, písnička si jí získá,
domů jdou spolu ve smíchu, je slyšet cikán, jak píská.
\end{xverse}


\begin{xverse}{4.~}
Jenže tatík, jak vidí cikána, pěstí do stolu tříská,
``ať táhne pryč, vesta odraná, groš nemá, něco ti spíská.''
\end{xverse}


\begin{xverse}{5.~}
Teď smutnou dceru má u vrátek, jen Bůh ví, jak se jí stýská,
``kéž vrátí se mi zas nazpátek ten, který v dálce si píská.''
\end{xverse}


\begin{xverse}{6.~}
Pár šídel honí se po louce, v trávě rosa se blýská,
cikán, rozmarýn v klobouce, jde dál a písničku píská.
\end{xverse}


\begin{xverse}{7.~}
Na závěr zbývá už jenom říct, v čem je ten kousek štístka:
peníze často nejsou nic, má víc, kdo po svém si píská ...
\end{xverse}

\end{song}

\begin{song}{Pochod marodů}{Jarek Nohavica}

\begin{xverse}{1.~}
[\large Ami]Krabička cigaret a [\large C]do kafe [\large G]rum, [\large F]rum, [\large Ami]rum,
dvě vodky a fernet a teď, [\large C]doktore, [\large G]{čum}, [\large F]{čum}, [\large Ami]{čum},
chra[\large Dmi]pot v hrud[\large F]ním ko[\large Ami]{ši}, no [\large Dmi]to je [\large F]záži[\large E]tek,
[\large Ami]my jsme kámoši řidi[\large C]{čů} sani[\large G]tek, -[\large F]tek, -[\large Ami]tek.
\end{xverse}


\begin{xverse}{2.~}
Měli jsme ledviny, ale už jsou nadranc, -dranc, -dranc,
i tělní dutiny už ztratily glanc, glanc, glanc,
u srdce divný zvuk, co je to, nemám šajn,
je to vlastně fuk, žijem fajn, žijem fajn, fajn, fajn.
\end{xverse}


\begin{xverse}{R.~}
[\large Ami]Cirhóza, [\large C]trombóza, [\large G]dávivý [\large C]kašel,
[\large Dmi]tuberku[\large Ami]lóza - [\large E]jó, to je [\large Ami]naše!
neuróza, [\large C]skleróza, [\large G]ohnutá [\large C]záda,
[\large Dmi]paraden[\large Ami]tóza, no [\large E]to je pa[\large Ami]ráda!
Jsme [\large Dmi]slabí na tě[\large Ami]le, ale [\large G]silní na du[\large C]chu,
[\large Dmi]{ži}jem vese[\large Ami]le, [\large E]juchuchuchu[\large Ami]chu!
\end{xverse}


\begin{xverse}{3.~}
Už kolem nás chodí pepka mrtvice, -ce, -ce,
tak pozor, marodi, je zlá velice, -ce, -ce,
zná naše adresy a je to čiperka,
koho chce, najde si, ten natáhne perka, -rka, -rka.
\end{xverse}


\begin{xverse}{4.~}
Zítra nás odvezou, bude veselo, -lo, -lo,
doktoři vylezou na naše tělo, -lo, -lo,
budou nám řezati ty naše vnitřnosti
a přitom zpívati ze samé radosti, -sti, -sti.
\end{xverse}


\begin{xverse}{R.~}
Zpívati: cirhóza, trombóza, dávivý kašel,
tuberkulóza, hele, já jsem to našel!
Neuróza, skleróza, křivičná záda,
paradentóza, no to je paráda!
Byli slabí na těle, ale silní na duchu,
žili vesele, než měli poruchu.
\end{xverse}

\end{song}

\begin{song}{Pole s bavlnou}{Rangers/Plavci}

\begin{xverse}{1.~}
Pane [\large C]můj, co v nebi je [\large C7]tvůj dům,
má máma můj [\large F]{ži}vot dala [\large C]katům,
katům mým v polích s bavl[\large G7]nou.
Pane [\large C]můj, co v nebi je [\large C7]tvůj dům,
má máma můj [\large F]{ži}vot dala [\large C]katům,
katům mým v [\large G7]polích s bavl[\large C]nou. [\large F|]{} [\large C|]{}
\end{xverse}

\begin{xverse}{R.~}
[\large C7]Den za dnem [\large F]kůže zná bič katů, vidíš [\large C]jen černý záda bratrů,
jak tam dřou v polích s bavl[\large G7]nou,
to, co [\large C]znáš ty v Lousi[\large C7]aně, černý záda [\large F]znaj' i v Texa[\large C]caně,
i tam jsou v [\large G7]polích s bavl[\large C]nou. [\large F|]{} [\large C|]{}
\end{xverse}

\begin{xverse}{2.~}
Já vím, brzy musí přijít soud, 
černý záda práva na něm vyhrajou,
Boží soud v polích s bavlnou.
\end{xverse}

\begin{xverse}{R.~}
Den za dnem ...
\end{xverse}

\begin{xverse}{3.~}
Chtěl bych jít na potem zvlhlý lány, 
tak řekni, Pane můj, černý zvoňte hrany
katům mým v polích s bavlnou.
\end{xverse}

\begin{xverse}{R.~}
Den za dnem ...
\end{xverse}

\begin{xverse}{4.~}
Dnes měj, Pane, co v nebi je tvůj dům, 
mou duši, když život patří katům,
katům mým v polích s bavlnou.
\end{xverse}

% \begin{xverse}{R.~}
% Den za dnem ...
% \end{xverse}
%
\end{song}

\begin{song}{Pošťák}{Hop trop}

\begin{xverse}{1.~}
[\large Ami]Psal jsem ti, [\large G]brácho, a [\large Ami]na ouřad [\large G]psaní šel [\large F]dát,
psaní vo [\large C]tom, že jsem [\large Dmi]{čer}nej, že z [\large Ami]fleku bych [\large E7]krad',
z [\large Ami]váčku jsem [\large G]lovil pár [\large Ami]centů a [\large G]{šéf} mi vtom [\large F]{řek':}
pošťák se [\large C]nevrátil, [\large Dmi]jestli bych [\large Ami]vzal po něm [\large E7]flek,
[\large Ami]sáně mi [\large G]dal, tresky v [\large Ami]balíku [\large G]pro psy a [\large F]kvér,
brašnu, v ní [\large C]lejstra, a [\large Dmi]po zádech [\large Ami]plác' mě: ``Buď [\large E7]fér!''
\end{xverse}

\begin{xverse}{R.~}
[\large A]Pošťák se má, za [\large D]známky nepla[\large A]tí,
[\large D]hlavně když se s [\large C]pytlem prachů [\large E7]někam neztra[\large A]tí,
pošťák se má, a [\large D]když se neztra[\large A]tí,
[\large D]za pět roků [\large C]utopený [\large E7]sáně zapla[\large A]tí.
\end{xverse}

\begin{xverse}{2.~}
Ten rok byl divnej, i slunce si přišlo ňák dřív,
led se měl hnout, a když ne, tak to stal by se div,
místo jsem našel, kde předjíždět Kobuk měl jít,
dál předák nechtěl a já nerad musel psy bít,
zázrakem živej pak dostal se na druhej břeh
bez psů a sání, a proklínal zbytečnej spěch.
\end{xverse}

\begin{xverse}{R.~}
Pošťák se má ...
\end{xverse}

\end{song}

% \begin{song}{Pozor, tunel!}{Kamelot}
%
% \begin{xverse}{1.~}
% [\large Ami]Slyšíš rachot kol a k rozednění půlhodina [\large Fmaj7]schází,
% Sára [\large Dmi7]protře rosou oči a [\large Emi]prázdný kapsy obrá[\large Ami]tí.
% Váš brácha severák vyfouká z vlasů zbytky sazí,
% děti kolejnic a pražců stěží se domů navrátí.
% \end{xverse}
%
% \begin{xverse}{R.~}
% Už [\large G]zpívá telegraf, seš [\large D]jedno velký [\large Ami]ucho
% a [\large G]{ťuká} zprávu [\large Fmaj7]zpráv: [\large E7]{}
% Jimmy - Jimmy - Jimmy - Jimmy, pozor, tunel! [\large Ami]{}
% \end{xverse}
%
% \begin{xverse}{2.~}
% Jak dravec letí vlak, možná, že mašinfíra šílí,
% ``Honem chyť mou ruku, holka, než padneš k vekslu na zobák.''
% Ten blázen žene stroj, že se snad zastaví až v Dillí,
% je špatný znamení, když v nebi krouží černej pták.
% \end{xverse}
%
% \begin{xverse}{R.~}
% Už zpívá telegraf ...
% \end{xverse}
%
% \end{song}
% \chords{ \chordFmajSeven \chordDmiSeven }

\begin{song}{Rychlé šípy}{Wabi Ryvola}

\begin{xverse}{1.~}
[\large Emi]Můj život je hned plný nesnází,
[\large A]na jaře když duben přichází,
já [\large C]vracím se do poválečnejch let,
[\large Emi]kdy vycházel náš starý dobrý Vpřed,
já [\large G]{žlu}tý kvítek za klopu si dám
a [\large Ami]píseň Vontů tiše zabroukám,
do [\large D]Stínadel se šerem vypravím,
snad [\large Emi]potkám cestou [\large H7]Losnu, co já vím.
\end{xverse}


\begin{xverse}{2.~}
Dunivá Kateřina burácí
a Široko má dávno po práci,
jen já se vracím Myší pastí sám,
nevím, co s ježkem v kleci dělat mám.
Bohouš,Dlouhé Bidlo, Štětináč,
pan Fišer pustil z okna květináč,
Jan Tleskač, Jiří Rymáň a tak dál,
pan Foglar tohle nikdy nenapsal.
\end{xverse}


\begin{xverse}{3.~}
To Rychlé šípy samy byly v nás
a žlutý kvítek symbolem byl krás,
co nemůže nám nikdy nikdo vzít,
kdo kopal studnu, aby druhej moh pít.
Snad jednou až se jaro navrátí,
můj život píseň Vontů obrátí,
já svobodný a čistý půjdu dál
a směšný bude ten. kdo se mi smál.
\end{xverse}


\begin{xverse}{4.~}
Tak Mirek Dušín s Červenáčkem jdou
a Jindra Hojer s Jarkou Metelkou,
za nima Rychlonožka s Bublinou,
naší krásnou chlapeckou krajinou.
[\large G]Duj, [\large D]duj, [\large C]duj, [\large Emi]fujaro vítězná.
\end{xverse}
\end{song}

\begin{song}{Starý příběh}{Spirituál kvintet}

\begin{xverse}{1.~}
Řek' [\large C]Mojžíš jednou [\large Fmaj7]lidu svému: [\large C]přišel [\large Fmaj7]{}čas,
dnes v [\large C]noci tiše [\large Emi]vytratí se [\large F]každý z [\large G]nás.
[\large C]Má[\large E]vá, [\large F]má[\large D7]vá nám [\large C]všem svo[\large Fmaj7]bodná [\large C|]{zem.} [\large Fmaj7|]{} [\large C|]{}
\end{xverse}


\begin{xverse}{2.~}
Já říkám rovnou: každý ať s tím počítá,
že naše cesta ke štěstí je trnitá.
Mává, mává nám všem svobodná zem.
\end{xverse}


\begin{xverse}{R.~}
[\large C]Kdo se bojí vodou jít,
ten podle tónů faraónů musí [\large G]{}žít.
[\large C]Má[\large E]vá, [\large F]má[\large D7]vá nám [\large C]všem svo[\large Fmaj7]bodná [\large C]{zem.} [\large Fmaj7|]{} [\large C|]{}
\end{xverse}


\begin{xverse}{3.~}
Až první krúček bude jednou za námi,
tak nikdo nesmí zaváhat, dát na fámy.
Mává, mává nám všem svobodná zem.
\end{xverse}


\begin{xverse}{4.~}
Pak tenhle vandr všem potomkům ukáže,
že šanci má jen ten, kdo má dost kuráže.
Mává, mává nám všem svobodná zem.
\end{xverse}


\begin{xverse}{R.~}
Kdo se bojí vodou jít ...
\end{xverse}


\begin{xverse}{5.~}
Ten starý příběh z Bible vám tu vykládám,
ať každý ví, že rozhodnout se musí sám.
Mává, mává nám všem svobodná zem.
\end{xverse}


\begin{xverse}{R.~}
Kdo se bojí vodou jít ...
\end{xverse}

\end{song}

\begin{song}{Strom}{Ozvěna}

\begin{xverse}{1.~}
[\large Ami]Polní cestou kráčeli šu[\large G]maři do vísky hrát,
[\large Ami]svatby, pohřby tahle cesta po[\large G]znala mnohokrát,
po [\large F]jedné svatbě se [\large G]chudým lidem [\large Ami]synek narodil
a [\large F]táta mu u [\large G]prašný cesty [\large E]{}života strom zasadil.
\end{xverse}

\begin{xverse}{R.~}
A on tam [\large A]stál, a koukal [\large F#mi]do polí,
byl jak [\large D]král, sám v celém [\large E]okolí,
korunu [\large A]měl, korunu [\large F#mi]měl, i když ne [\large D]ze zlata, [\large Dmi]{}
a jeho [\large A]pokladem byla [\large E]tráva střapa[\large A]tá.
\end{xverse}

\begin{xverse}{2.~}
Léta běží a na ten příběh si už nikdo nevzpomněl,
jen košatý strom se u cesty ve větru tiše chvěl,
a z vísky bylo město a to město začlo chtít
asfaltový koberec až na náměstí mít.
\end{xverse}

\begin{xverse}{R.~}
A on tam stál ...
\end{xverse}

\begin{xverse}{3.~}
Že strom stál v cestě plánované, to malý problém byl,
ostrou pilou se ten problém snadno vyřešil,
tak naposled se do nebe náš strom pak podíval
a tupou ránu do větvoví už snad ani nevnímal.
\end{xverse}

\begin{xverse}{R.~}
A on tam stál ...
\end{xverse}


\begin{xverse}{4.~}
Při stavbě se ukázalo, že silnice půjde dál,
a tak kousek od nové cesty smutný pařez stál,
dětem a výletníkům z výšky nikdo nemával
a jen přítel vítr si o něm píseň na strništích z nouze hrál.
\end{xverse}


% \begin{xverse}{R.~}
% Jak tam stál ...
% \end{xverse}

\end{song}

\begin{song}{Škrábej}{Hop trop}

\begin{xverse}{1.~}
[\large Emi]Trojstěžníku plachty k rozervání napnutý,
[\large G]třináctej den je to s náma nějaký nahnutý,
[\large D]smůlu táhnem za kormidlem s sebou po vl[\large Emi]nách,
my lodníci jsme na tom nejhůř, vím to na tuty,
[\large G]pískovcovou cihlu v ruce, záda vohnutý,
[\large D]bocman vříská, nejradši bych po krku mu [\large Emi]sáh'.
\end{xverse}


\begin{xverse}{R.~}
Hej, [\large Emi]hej! Škrábej ty prkna, ať jsou [\large C]bílý!
Hej, [\large Emi]hej! Škrábej, ty prkna musej' [\large C]bejt!
Hej, [\large Emi]hej! Říkej si klidně každou [\large C]chvíli:
[\large Emi]nebudem [\large D]spílat, [\large Emi]ruce [\large D]spínat, [\large Emi]{}žalmy [\large D]zpívat, [\large Emi]hou!
\end{xverse}


\begin{xverse}{2.~}
Bez vody jsme všichni skoro žízní leknutý,
nikomu z nás nevadí, že spíme vobutý,
stejně každej den jeden z nás končí na marách,
čert aby vzal bocmana a s ním i drhnutí,
ze kterýho máme teď ty záda vohnutý,
chcem bejt rovný, až do pekla překročíme práh.
\end{xverse}


\begin{xverse}{R.~}
Hej hej ...
\end{xverse}

\end{song}

\begin{song}{Šnečí blues}{Jarek Nohavica}

\begin{xverse}{1.~}
[\large G]Jednou [\large C7]jeden [\large G]{šnek}  [\large D/F#|]{} [\large G]{ší}le[\large C]ně se [\large G]lek', [\large D7|]{}
[\large G]nikdo už dnes [\large G7]neví, z [\large C]{če}ho se tak zjevil,
že se [\large G]dal hned [\large D7]na  ú[\large G]těk. [\large D7|]{}
\end{xverse}


\begin{xverse}{2.~}
Přes les a mýtinu rychlostí půl metru za hodinu,
z ulity pára, ohnivá čára,
měl cihlu na plynu.
\end{xverse}


\begin{xverse}{3.~}
Ale v jedné zatáčce, tam v mechu u svlačce,
udělal šnek chybu, nevyhnul se hřibu,
nevyhnul se bouračce.
\end{xverse}


\begin{xverse}{4.~}
Hned seběhl se celý les a dali šneka pod pařez,
tam v tom lesním stínu, jestli nezahynul,
tak leží ještě dnes.
\end{xverse}


\begin{xverse}{5.~}
A kdyby použil vůz anebo autobus,
/: nebylo by nutné zpívat tohle smutné,
   smutné šnečí blues. :/
\end{xverse}

\end{song}
\chords{ \chordDFis }

\begin{song}{Tak si tam stůj}{Hop trop}

\begin{xverse}{1.~}
Tak si tam [\large Dmi]stůj, já [\large C]dál tě klidně [\large Gmi]na krajnici [\large Bb]nechám,
tak si tam [\large Dmi]stůj, už [\large C]za chvíli se [\large Gmi]hodně seše[\large Bb]{}ří,
[\large Dmi]nezasta[\large Ami]vím, a [\large F]nebude to [\large Gmi]tím, že zrovna [\large Ami]spěchám,
nezasta[\large Gmi]vím, když vidím, že mi někdo nevě[\large Dmi]{}ří.
\end{xverse}

\begin{xverse}{R.~}
Polykám [\large C]dálku, letí mi [\large Dmi]{}čas,
když neza[\large Bb]máváš, [\large A7]tak vem tě [\large Dmi]{}ďas.
\end{xverse}


\begin{xverse}{2.~}
Tak si tam stůj, já napíšu do špíny na kontejner,
tak si tam stůj, já cestou s někým povídat si chtěl,
nezastavil můj udejchanej zablácenej trailer,
nezastavil, tvý pohrdavý oči uviděl.
\end{xverse}


\begin{xverse}{R.~}
Polykám dálku, letí mi čas...
\end{xverse}


\begin{xverse}{3.~}
Tak si tam stůj, snad naloží tě ňákej lepší auták,
tak si tam stůj, už za chvíli tě večer zastudí,
zastavím tam, kde za pár hodin vystřídá mě parťák,
zastavím tam, kde za zádama se mi probudí.
\end{xverse}

\end{song}

\begin{song}{Toulavej}{Vojta Kiďák Tomáško}

\begin{xverse}{1.~}
Někdo [\large Ami]z vás, kdo chutnal [\large G]dálku, jeden [\large Ami]z těch, co rozu[\large E]měj',
ať vám [\large Ami]poví, proč mi [\large G]{ří}kaj', proč mi [\large F]{ří}kaj' Toula[\large Ami]vej.
\end{xverse}


\begin{xverse}{2.~}
Kdo mě zná a v sále sedí, kdo si myslí: je mu hej,
tomu zpívá pro všední den, tomu zpívá Toulavej.
\end{xverse}


\begin{xverse}{R.~}
[\large F]Sobotní ráno [\large G]mě neuvidí u [\large G7]cesty s klukama [\large C]stát
[\large F]na půdě celta se [\large G]prachem stydí [\large F]a starý songy jsem [\large G]zapomněl hrát,
zapomněl [\large Ami]hrát.
\end{xverse}


\begin{xverse}{3.~}
Někdy v noci je mi smutno,často bejvám doma zlej,
malá daň za vaše ``umí'',kterou splácí Toulavej.
\end{xverse}


\begin{xverse}{4.~}
Každej měsíc jiná štace,čekáš, kam tě uložej',
je to fajn, vždyť přece zpívá,třeba smutně, Toulavej.
\end{xverse}

\begin{xverse}{R.~}
Sobotní ráno mě neuvidí ...
\end{xverse}

\begin{xverse}{5.~}
Vím, že jednou někdo přijde,tiše pískne: no tak jdem,
známí kluci ruku stisknou,řeknou: vítej, Toulavej.
\end{xverse}


\begin{xverse}{6.~}
Budou hvězdy jako tenkrát,až tě v očích zabolej',
celou noc jim bude zpívat jeden blázen - Toulavej.
\end{xverse}


\begin{xverse}{R.~}
Sobotní ráno nám poletí vstříc,budeme u cesty stát,
vypráším celtu a můžu vám říct,že na starý songy si vzpomenu rád,
vzpomenu rád.
\end{xverse}

\begin{xverse}{7.~}
Někdo [\large Ami]z vás, kdo chutnal [\large G]dálku, jeden [\large Ami]z těch, co rozu[\large E]měj',
ať vám [\large Ami]poví, proč mi [\large G]{ří}kaj', proč mi [\large F]{ří}kaj' [\large E]Toula[\large Ami]vej.
\end{xverse}
\end{song}

\begin{song}{Trh ve Scarborough}{Spirituál kvintet}

\begin{xverse}{1.~}
[\large Emi]Příteli, máš do [\large D]Scarborough [\large Emi]jít,
[\large G]dobře [\large Emi]vím, že [\large G]půjdeš [\large A]tam [\large Emi]rád,
tam dívku [\large G]najdi na [\large F#mi]Mar[\large Emi]ket [\large D]Street,
[\large Emi]co chtěla [\large A]dřív [\large C]mou [\large D]{že}[\large Emi]nou [\large D]se [\large Emi]stát.
\end{xverse}

\begin{xverse}{2.~}
Vzkaž ji, ať šátek začne mi šít,
za jehlu niť však smí jenom brát
a místo příze měsíční svit,
bude-li chtít mou ženou se stát.
\end{xverse}

\begin{xverse}{3.~}
Až přijde máj a zavoní zem,
šátek v písku přikaž ji prát
a ždímat v kvítí jabloňovém,
bude-li chtít mou ženou se stát.
\end{xverse}

\begin{xverse}{4.~}
Z vrkočů svých ať uplete člun,
v něm se může na cestu dát,
s tím šátkem ať vejde v můj dům,
bude-li chtít mou ženou se stát.
\end{xverse}

\begin{xverse}{5.~}
Kde útes ční za přívaly vln,
zorej dva sáhy pro růží sad,
za pluh ať slouží šípkový trn,
budeš-li chtít mým mužem se stát.
\end{xverse}

\begin{xverse}{6.~}
Osej ten sad a slzou ho skrop,
choď těm růžím na loutnu hrát,
až začnou kvést, tak srpu se chop,
budeš-li chtít mým mužem se stát.
\end{xverse}

\begin{xverse}{7.~}
Z trní si lůžko zhotovit dej,
druhé z růží pro mě nech stlát,
jen pýchy své a Boha se ptej,
proč nechci víc tvou ženou se stát.
\end{xverse}

\end{song}

\begin{song}{Tři bratři}{Spirituál kvintet}

\begin{xverse}{1.~}
[\large Ami]Tři bratři žili kdys v zemi skotské,
v domě zchudlém jim [\large D]souzeno [\large Ami]{žít}, [\large E]{}
ti [\large Ami]kostkama metali, kdo z nich má jíti,
[\large D]kdo z nich má [\large Ami|]{jít,} [\large Emi]{}
[\large F]kdo z nich má [\large C]na moři [\large G]pirátem [\large Ami]být.
\end{xverse}


\begin{xverse}{2.~}
Los padl a Henry už opouští dům,
ač je nejmladší z nich, vybrán byl,
by koráby přepadal, na moři žil,
na moři žil,
své bratry z nouze tak vysvobodil.
\end{xverse}


\begin{xverse}{3.~}
Po dobu tak dlouhou, jak v zimě je noc,
a tak krátkou, jak zimní je den,
už plaví se Henry, když před sebou objeví
loď, pyšnou loď:
``Napněte plachty a kanóny ven!''
\end{xverse}


\begin{xverse}{4.~}
Čím kratší byl boj, tím byl bohatší lup,
z vln už ční jenom zvrácený kýl,
teď Henry je boháč, když boháče oloupil,
loď potopil,
své bratry z nouze tak vysvobodil.
\end{xverse}


\begin{xverse}{5.~}
Do Anglie staré dnes smutná jde zvěst,
smutnou zprávu dnes dostane král,
ke dnu klesla pyšná loď poklady Henry si
vzal, on si vzal;
střezte se moře, on vládne tam dál!
\end{xverse}
\end{song}

\begin{song}{Tři kříže}{Hop trop}

\begin{xverse}{1.~}
[\large Dmi]Dávám sbohem všem [\large C]břehům prokla[\large Ami]tejm,
který v [\large Dmi]drápech má [\large Ami]{}ďábel [\large Dmi]sám,
bílou přídí [\large C]{}šalupa ``My [\large Ami]Grave''
míří k [\large Dmi]{}útesům, [\large Ami]který [\large Dmi]znám.
\end{xverse}


\begin{xverse}{R.~}
Jen tři [\large F]kříže z bí[\large C]lýho kame[\large Ami]ní
někdo [\large Dmi]do písku [\large Ami]posklá[\large Dmi]dal,
slzy v [\large F]očích měl a v [\large C]ruce, [\large Ami]znavený,
lodní [\large Dmi]deník, co [\large Ami]sám do něj [\large Dmi]psal.
\end{xverse}


\begin{xverse}{2.~}
První kříž má pod sebou jen hřích,
samý pití a rvačky jen,
chřestot nožů, při kterým přejde smích,
srdce-kámen a jméno Stan.
\end{xverse}


\begin{xverse}{R.~}
Jen tři kříže...
\end{xverse}


\begin{xverse}{3.~}
Já, Bob Green, mám tváře zjizvený,
štěkot psa zněl, když jsem se smál,
druhej kříž mám a spím pod zemí,
že jsem falešný karty hrál.
\end{xverse}


\begin{xverse}{R.~}
Jen tři kříže...
\end{xverse}


\begin{xverse}{4.~}
Třetí kříž snad vyvolá jen vztek,
Fatty Rogers těm dvěma život vzal,
svědomí měl, vedle nich si klek' ...
\end{xverse}

\begin{xverse}{Rec.~}
Snad se chtěl modlit:
"Vím, trestat je lidský,
ale odpouštět božský,
snad mi tedy Bůh odpustí ..."
\end{xverse}

\begin{xverse}{R.~}
Jen tři kříže z bílýho kamení
jsem jim do písku poskládal,
slzy v očích měl a v ruce, znavený,
lodní deník, a v něm, co jsem psal ...
\end{xverse}

\end{song}

% \begin{song}{Tu kytaru jsem koupil kvůli tobě}{Václav Neckář}
%
% \begin{xverse}{*.~}
% [\large E]Jak můžeš být tak [\large C#7]krutá,
% [\large F#mi]copak nemáš kouska citu v [\large H6]těle.
% \end{xverse}
%
%
% \begin{xverse}{1.~}
% [\large E]Tu [\large E6]kytaru [\large E]jsem [\large E6]kou[\large E]{pil}
% [\large E6]kvů[\large E]{li} [\large E6]to[\large E]{bě}
% a [\large E6]dal jsem za [\large E]ni
% [\large E6]ce[\large E]{lej} [\large E6]tá[\large E]{tův} [\large H7]plat,
% ta [\large E]dávno [\large H7]ještě [\large E9]byla ve vý[\large A]robě
% [\large Ami]a já už [\large E]věděl
% [\large H7]co ti budu [\large E]hrát. [\large H7]{}
% \end{xverse}
%
%
% \begin{xverse}{2.~}
% To ještě rostla v javorovém lese
% a jenom vítr na to dřevo hrál,
% a já už trnul, jestli někdy snese,
% [\large Ami]{žár}, který [\large E]ve mně [\large H7]denně narů[\large E]stal.
% \end{xverse}
%
%
% \begin{xverse}{}
% S tou [\large C#mi]kytarou teď stojím před tvým [\large G#mi]domem,
% měj [\large A]soucit aspoň k tomu javo[\large H]ru, [\large H7]{}
% jen [\large E]kvůli tobě [\large E7]přestal býti [\large A]stromem, [\large Ami]{}
% [\large Ami]tak už nás [\large E]oba [\large H7]pozvi naho[\large E]ru,
% [\large H7]pozvi naho[\large E]ru, [\large D]pozvi [\large D#]naho[\large E]ru.
% \end{xverse}
% \end{song}
% \chords{ \chordESix \chordENine }

\begin{song}{Tunel jménem Čas}{Hoboes}

\begin{xverse}{1.~}
Těch [\large E]strašnejch vlaků, [\large G#mi]co se ženou [\large E7]kolejí tvejch [\large A]snů,
těch [\large Ami]asi už se [\large E]nezbavíš [\large F#7]do posledních [\large H7]dnů,
a [\large E]hvězdy žhavejch [\large G#mi]uhlíků ti [\large E7]nikdy nedaj' [\large A]spát,
tvá [\large Ami]dráha míří [\large E]k tunelu a [\large Fmaj7/5-]tunel, ten má [\large E]hlad.
\end{xverse}


\begin{xverse}{2.~}
Už kolikrát ses mašinfíry zkusil na to ptát,
kdo nechal roky nejhezčí do vozů nakládat,
proč vlaky, co si každou noc pod voknem laděj' hlas,
spolyká díra kamenná, tunel jménem Čas.
\end{xverse}


\begin{xverse}{3.~}
Co všechno vlaky vodvezly, to jenom pán Bůh ví,
tvý starý lásky, mladej hlas a slova bláhový,
a po kolejích zmizela a padla za ní klec,
co bez tebe žít nechtěla a žila nakonec.
\end{xverse}


\begin{xverse}{4.~}
A zvonky nočních nádraží a vítr na tratích
a honky-tonky piána a uplakaný smích
a písničky a šťastný míle na tulácký pas
už spolkla díra kamenná, tunel jménem Čas.
\end{xverse}


\begin{xverse}{5.~}
Než poslední vlak odjede, a to už bude zlý,
snad ňákej minér šikovnej ten tunel zavalí
a veksl zpátky přehodí v té chvíli akorát,
i kapela se probudí a začne zase hrát.
\end{xverse}


\begin{xverse}{6.~}
Vlak v nula-nula-dvacetpět bude ten poslední,
minér svou práci nestačí dřív, než se rozední,
ten konec moh' bejt veselej, jen nemít tenhle kaz,
tu černou díru kamennou, tunel jménem Čas.
\end{xverse}

\end{song}
% \chords{ \chordFmajSevenFiveMinus }

\begin{song}{Už to nenapravím}{Máci}

\begin{xverse}{1.~}
V [\large Ami]devět hodin dvacet pět mě [\large D]opustilo štěstí,
ten [\large F]vlak, co jsem jím měl jet, na koleji [\large E]dávno [\large E7]nestál,
v [\large Ami]devět hodin dvacet pět ja[\large D]ko bych dostal pěstí,
já [\large F]za hodinu na náměstí měl jsem [\large E]stát, ale v [\large E7]jiným městě.
Tvá [\large A7]zpráva zněla prostě a byla tak krátká,
že [\large Dmi]stavíš se jen na skok, že nechalas' mi vrátka
[\large G]zadní otevřená, [\large E]zadní otevřená,
já [\large A7]naposled tě viděl, když ti bylo dvacet,
[\large Dmi]to jsi tenkrát řekla, že se nechceš vracet,
[\large G]{že} jsi unavená, [\large E]ze mě unavená.
\end{xverse}

\begin{xverse}{2.~}
Já čekala jsem, hlavu jako střep, a zdálo se, že dlouho,
může za to vinný sklep, že člověk často sleví,
já čekala jsem, hlavu jako střep, s podvědomou touhou,
já čekala jsem dobu dlouhou, víc než dost, kolik přesně, nevím.
Pak jedenáctá bila a už to bylo passé,
já dřív jsem měla vědět, že vidět chci tě zase,
láska nerezaví, láska nerezaví,
ten list, co jsem ti psala, byl dozajista hloupý,
byl odměřený moc, na vlídný slovo skoupý,
už to nenapravím, už to nenapravím.
\end{xverse}

\end{song}

\begin{song}{Válka růží}{Spirituál kvintet}

\begin{xverse}{1.~}
Už [\large Dmi]rozplynul se [\large G]hustý dým, [\large Dmi]derry down, hej, [\large A]down-a-down,
nad [\large Dmi]ztichlým polem [\large Gmi]válečným, derry [\large Dmi]down, [\large A]{}
jen [\large F]ticho stojí [\large C]kolko[\large A]lem a [\large Dmi]vítěz [\large Dmi/C|]{plení} [\large Bb]vlastní [\large A]zem,
je válka [\large Dmi]růží, down, [\large Gmi]derry, derry, [\large A]derry down-a-[\large Dmi]down.
\end{xverse}


\begin{xverse}{2.~}
Nečekej soucit od rváče, derry down, hej, down-a-down,
kdo zabíjí ten nepláče, derry down,
na těle mrtvé krajiny se mečem píšou dějiny,
je válka růží, down, derry, derry, derry down, a-down.
\end{xverse}


\begin{xverse}{3.~}
Dva erby, dvojí korouhev, derry down, hej, down-a-down,
dva rody živí jeden hněv, derry down,
kdo změří, kam se nahnul trůn, zda k Yorkům nebo k Lancastrům,
je válka růží, down, derry, derry, derry down, a-down.
\end{xverse}


\begin{xverse}{4.~}
Dva erby, dvojí korouhev, derry down, hej, down-a-down,
však hlína pije jednu krev, derry down,
ať ten či druhý přežije, vždy nejvíc ztratí Anglie,
je válka růží, down, derry, derry, derry down, a-down.
\end{xverse}

\end{song}
\chords{ \chordDmiC }

% \begin{song}{Zabili, zabili}{Balada pro banditu}
%
% \begin{xverse}{1.~}
% [\large C]Zabili [\large F]zabili [\large Dmi]chlapa z Kolo[\large F]{ča}vy
% [\large C]{ře}kněte [\large F]hrobaři [\large Dmi]kde je pocho[\large F]vaný
% \end{xverse}
%
% \begin{xverse}{R.~}
% Bylo tu [\large C]není tu havrani [\large F]na plotu
% bylo víno v [\large C]sudě teď tam voda [\large F]bude
% není [\large C]není tu
% \end{xverse}
%
% \begin{xverse}{2.~}
% Špatně ho zabili špatně pochovali
% vlci ho pojedli ptáci rozklovali
% \end{xverse}
%
% \begin{xverse}{R.~}
% Bylo tu, není tu ...
% \end{xverse}
%
% \begin{xverse}{3.~}
% Vítr ho roznesl po dalekém kraji
% havrani pro něho na poli krákají
% \end{xverse}
%
% \begin{xverse}{R.~}
% Bylo tu, není tu ...
% \end{xverse}
%
% \begin{xverse}{4.~}
% Kráká starý havran krákat nepřestane
% dokud v Koločavě živý chlap zůstane
% \end{xverse}
%
% \end{song}
%
% \begin{song}{Zafúkané}{Fleret}
%
% \begin{xverse}{1.~}
% [\large Ami]Větr sněh [\large A2]zanésl z [\large Ami]hor do [\large A2]polí,
% [\large Ami]já idu [\large C]přes kopce, [\large G]přes údo[\large Ami]lí,
% [\large C]idu k tvej [\large G]dědině zatúla[\large C]nej,
% [\large F]cestičky [\large C]sněhem sú [\large E]zafúkané. [\large Ami|]{} [\large Fmaj7|]{} [\large Ami|]{} [\large E4sus|]{}
% \end{xverse}
%
%
% \begin{xverse}{R.~}
% [\large Ami]Zafúka[\large C]né, [\large G]zafúka[\large C]né
% [\large F]kolem mňa [\large C]všecko je [\large Dmi]zafúka[\large E]né
% [\large Ami]Zafúkané [\large C|]{}, [\large G|]{} zafúka[\large C]né,
% [\large F]kolem mňa [\large C]fšecko je [\large E]zafúka[\large Ami]né
% [\large Emi|]{} [\large D|]{} [\large G|]{} [\large H7|]{} [\large Emi|]{} [\large D|]{} [\large G|]{} [\large H7|]{} [\large Emi|]{}
% \end{xverse}
%
%
% \begin{xverse}{2.~}
% Už vašu chalupu z dálky vidím,
% srdce sa ozvalo, bit ho slyším,
% snáď enom pár kroků mi zostává,
% a budu u tvého okénka stát.
% \end{xverse}
%
%
% \begin{xverse}{R.~}
% /: Ale zafúkané, zafúkané, okénko k tobě je zafúkané. :/
% \end{xverse}
%
%
% \begin{xverse}{3.~}
% Od tvého okna sa smutný vracám,
% v závějoch zpátky dom cestu hledám,
% spadl sněh na srdce zatúlané,
% aj na mé stopy - sú zafúkané.
% \end{xverse}
%
%
% \begin{xverse}{R.~}
% /: Zafúkané, zafúkané, mé stopy k tobě sú zafúkané. :/
% \end{xverse}
%
% \end{song}
% \chords{ \chordATwo \chordEFourSus \chordFmajSeven }

\begin{song}{Zachraňte koně}{Kamelot}

\begin{xverse}{1.~}
[\large Emi]Peklo byl ráj, když hořela stáj, [\large Ami7]příteli,
[\large C]věř mi, koně [\large D]pláčou, poví[\large G]dám, [\large C|]{} [\large H7|]{}
[\large Emi]to byla půlnoc, v tom křik o pomoc, už [\large Ami7]letěly
[\large C]hejna kohoutů, [\large H7]{} a bůhví [\large Emi]kam.
\end{xverse}

\begin{xverse}{R.~}
[\large G]Zachraňte koně, [\large Hmi]křičel jsem tisíc[\large C]krát,
[\large G]{žil} jsem jen pro ně, [\large Hmi]bránil je nejvíc[\large C]krát,
než přišla [\large Ami]chvíle, kdy hřívy [\large C]bílé
pročesal [\large Ami]plamen, spálil na [\large H7]troud.
\end{xverse}

\begin{xverse}{2.~}
Ohrady a stáj, a v plamenech kraj už nedýchal,
já viděl, jak to hříbě umírá,
klisna u něj a smuteční děj se odbývá,
jak tiše pláče, oči přivírá.
\end{xverse}

\begin{xverse}{R.~}
Zachraňte koně...
\end{xverse}

\end{song}
\chords{ \chordAmiSeven }

\begin{song}{Ze všech chlapů nejšťastnější chlap}{Hoboes}

\begin{xverse}{R.~}
To [\large Dmi]znám, to dobře [\large Ami]znám, znám, znám,
[\large E7]na kolejích nejsem nikdy [\large Ami]sám, sám, sám,
to [\large Dmi]znám, to dobře [\large Ami]znám, znám, znám,
[\large E7]na kolejích nejsem nikdy [\large Ami]sám.
\end{xverse}

\begin{xverse}{1.~}
[\large Ami]Shejbni hlavu, kamaráde, tunel před námi,
[\large Dmi]veksle tlučou, píšťaly řvou, zvonce vyzvání,
[\large E7]v boudě dobrej mašinfíra není žádnej srab,
[\large Ami]i v tom dešti [\large Dmi]sazí jsem ten [\large E7]nejšťastnější [\large Ami]chlap,
jó, [\large E]ze všech chlapů [\large E7]nejšťastnější [\large Ami]chlap.
\end{xverse}

\begin{xverse}{2.~}
Když z komína vod mašiny letí černej dým,
na tom světě jenom jednu věc na tuty vím,
na tom světě širokým věc jednu jistou mám,
na kolejích chudej hobo není nikdy sám,
jó, chudej hobo není nikdy sám.
\end{xverse}

\begin{xverse}{R.~}
To znám...
\end{xverse}


\begin{xverse}{3.~}
Za zádama Frisco, semafor je zelenej,
vlak to žene do tmy jako bejček splašenej,
radujte se, občánkové, hoboes jedou k vám,
na kolejích chudej hobo není nikdy sám,
jó, chudej hobo není nikdy sám.
\end{xverse}

\begin{xverse}{4.~}
Pod zádama uhlí mám a deku děravou,
místo lampy večerní jen hvězdy nad hlavou,
navečer jsem do vagónu zalez' jako krab,
i v tom dešti sazí jsem ten nejšťastnější chlap,
jó, ze všech chlapů nejšťastnější chlap.
\end{xverse}

\begin{xverse}{R.~}
To znám...
\end{xverse}

\begin{xverse}{5.~}
Viděl jsem ji u pangejtu vedle dráhy stát,
usmála se, zamávala, z vagónu jsem spad',
jářku: hallo! Sklopí voči, udělá to ``klap'',
i v tom dešti sazí jsem ten nejšťastnější chlap,
jó, ze všech chlapů nejšťastnější chlap.
\end{xverse}

\begin{xverse}{6.~}
=\ 2.
\end{xverse}

\begin{xverse}{R.~}
To znám...
\end{xverse}

\end{song}
\pagestyle{empty}

\fontsize{12.5pt}{16pt}
\selectfont

% \newpage
% \thispagestyle{empty}
%
% ZPĚVNÍK
% \bigskip
%
% 121. oddílu Stopaři
% \bigskip
%
% 11. střediska podplukovníka Vally
% \bigskip
%
% Junáka -- svazu skautů a skautek ČR
%
% \newpage
% \thispagestyle{empty}
% \mbox{}


\begin{song}{Amazonka}{Hop-trop}

\begin{xverse}{1. }
Byly krásný naše [A]plány,
byla jsi můj celej svět, [C#mi|]{} [Cmi|]{}
[Hmi]{}čas je vzal a nechal [A]rány,
[Hmi]starší jsme jen o pár [E]let.
\end{xverse}

\begin{xverse}{2. }
Tenkrát byly děti malý, ale život utíká,
už na ``táto'' slyší jinej, i když si tak neříká.
\end{xverse}

\begin{xverse}{R. }
Nebe modrý zrcad[A]lí se
v [F#7]{}řece, která všechno [Hmi]ví,
stejnou barvu jako [A]měly
[Hmi]tvoje oči džíno[E]vý.
\end{xverse}

\begin{xverse}{3. }
Kluci tenkrát, co tě znali,  všude, kde jsem s tebou byl,
Amazonka říkávali, a já hrdě přisvědčil.
\end{xverse}

\begin{xverse}{4. }
Tvoje strachy, že ti mládí pod rukama utíká
vedly k tomu, že ti nikdo Amazonka neříká.
\end{xverse}

\begin{xverse}{R. }
Nebe modrý zrcadlí se ...
\end{xverse}

\begin{xverse}{5. }
Zlatý kráse cingrlátek, jak sis časem myslela,
vadil možná trampskej šátek, nosit dál's ho nechtěla.
\end{xverse}

\begin{xverse}{R. }
Teď jsi víla z paneláku, samá dečka, samej krám,
já si přál jen, abys byla pořád stejná, přísahám,
pořád [Hmi]stejná, přísa[A]hám.
\end{xverse}

\end{song}

\setcounter{page}{1}

\begin{song}{Anděl}{Karel Kryl}

\begin{xverse}{1. }
[C]Z rozmláce[Ami]nýho kostela v [C]krabici s [G7]kusem mýdla
[C]přinesl [Ami]jsem si anděla, [C]poláma[G7]li mu [C]křídla,
díval se [Ami]na mě oddaně, [C]já měl jsem [G7]trochu trému,
[C]tak vtiskl [Ami]jsem mu do dlaně [C]lahvičku [G7]od par[C]fému.
\end{xverse}

\begin{xverse}{R. }
[C]A proto, [Ami]prosím, věř mi, [C]chtěl jsem ho [G7]{žá}dat,
[C]aby mi [Ami]mezi dveřmi [C]pomohl [G7]hádat,
[C]co mě čeká [Ami|]{}   [G7]a nemi[C]ne, co mě čeká [Ami|]{}   [G7]a    nemi[C]ne.
\end{xverse}

\begin{xverse}{2. }
Pak hlídali jsme oblohu, pozorujíce ptáky,
debatujíce o Bohu a hraní na vojáky,
do tváře jsem mu neviděl, pokoušel se ji schovat,
to asi ptákům záviděl, že mohou poletovat.
\end{xverse}

\begin{xverse}{R. }
A proto, prosím, věř mi ...
\end{xverse}

\begin{xverse}{3. }
Když novinky mi sděloval u okna do ložnice,
já křídla jsem mu ukoval z mosazný nábojnice,
a tak jsem pozbyl anděla, on oknem odletěl mi,
však přítel prý mi udělá novýho z mojí helmy.
\end{xverse}

\begin{xverse}{R. }
A proto, prosím, věř mi ...
\end{xverse}

\end{song}

\begin{song}{Až vzlétnou ptáci}{Spirituál Kvintet}

%TODO predelat komplet akordy

\begin{xverse}{1. }
[D]Na před[A]městí [D]stá[A]val [Hmi]dům, malý [A]chlapec [Hmi|]{si} [A]tam hrával
[D]drak, co [A]vzlétal k [D]ob[A]la[Hmi]kům, všechna [A]tajná [Hmi]přání [A]znával.
[G]Draci totiž [A]vždycky [G]ví to, co je klukům [A]nejvíc [G]líto
[Hmi]když mu[A]sí jít [Hmi]večer [A]spát.
\end{xverse}

\begin{xverse}{2. }
Jako víno dozrává, jako v mořích vlny hasnou
vzpomínka mi zůstává na tu smutnou zemi krásnou
na rybářské staré sítě, na draka a malé dítě,
které nemá si s kým hrát.
\end{xverse}

\begin{xverse}{3. }
Na provázku slunce měl, oblohou se za ním vznášel,
jako vánek šel kam chtěl smutné lampy lidem zhášel,
pohádkovou dýchal vůni, mluvil řečí horských tůní,
průzračnou jak dětský smích.
\end{xverse}

\begin{xverse}{4. }
Bílý koník běžel dál osamělou pustou plání,
na dlani sníh dětství tál i když padal bez ustání,
den začíná tichou flétnou, chvíli předtím, nežli vzlétnou
hejna ptáků ve větvích.
\end{xverse}

\begin{xverse}{5. }
= 1.
\end{xverse}

\end{song}

\begin{song}{Banka (Make love Cosa Nostra)}{Hoboes}

%TODO transponovat do od D

\begin{xverse}{1. }
Ze starejch [E]{}časáků a [F#]fotek zažlout[H7]lejch
šklebí se [E]chlap, co už se [F#]poldům dávno [H7]zdejch
gangster má na kahánku, když kouří [E]marijánku
miluje [H7]Paula Anku, je velkej [E]lump.
\end{xverse}

\begin{xverse}{2. }
Tuctovej ksicht a je to přece velkej šéf
na Pátý avenue se třese každej sejf
kolťáky vycíděný, heroin a krásný ženy
dolary upocený nejsou náš džob.
\end{xverse}

\begin{xverse}{R. }
[A]Jedeme přepadnout [D]veli[E]kánskou [A]banku
[A]ve vokýnku kulomet a [D]dvě stě [E]litrů v [A]tanku
[H7]ukradneme mraky dolarů
[E]sejdeme se večer u báru a tam je [E7]prochlastáme
[A]Zastřelíme poldu, [D]co tu [E]banku [A]chrání
[A]naši velký loupeži už [D]nikdo [E]neza[A]brání
[H7]soustředíme všechno úsilí
[E]poldové už shání posily a my je [E7]zastřelíme
[A]do břicha sekerou. [H7|]{} [E|]{}
\end{xverse}


\begin{xverse}{3. }
Nacpeme kolťáky do pouzder podpažních
namažem mechanismy zbraní trofejních
naskáčem do chrysleru kašleme na aféru
rozdělíme si sféru odkud až kam.
\end{xverse}


\begin{xverse}{4. }
Bezdýmej prach a prvotřídní auťáky
falešný dolary a whisky hekťáky
mafie naše máma decentně hejbe s náma
you are my sugar baby make love get back.
\end{xverse}


\end{song}

\begin{song}{Bedna vod whisky}{Hoboes}

\begin{xverse}{1. }
[Ami]Dneska už mně [C]fóry ňák [Ami]nejdou přes pys[E]ky,
[Ami]stojím s dlouhou [C]kravatou na [Ami]bedně [E]vod whis[Ami]ky,
stojím s dlouhým [C]vobojkem [Ami]jak stájovej [E]pinč,
tu [Ami]kravatu, co [C]nosím, mi [Ami]navlík' [E]soudce [A]Lynč.
\end{xverse}


\begin{xverse}{R. }
Tak [A]kopni do tý [D]bedny, ať [E]panstvo neče[A]ká,
jsou dlouhý schody [D]do nebe a [E]{}štreka dale[A]ká
do nebeskýho [D]baru, já [E]sucho v krku [A]mám,
tak kopni do tý [D]bedny, ať [E]na cestu se [A]dám. [Ami]{}
\end{xverse}


\begin{xverse}{2. }
Mít tak všechny bedny od whisky vypitý,
postavil bych malej dům na louce ukrytý,
postavil bych malej dům a z vokna koukal ven
a chlastal bych tam s Billem a chlastal by tam Ben.
\end{xverse}


\begin{xverse}{R. }
Tak kopni do tý bedny ...
\end{xverse}


\begin{xverse}{3. }
Kdyby jsi se, hochu, jen porád nechtěl rvát,
nemusel jsi dneska na týhle bedně stát,
moh' jsi někde v suchu tu svoji whisku pít,
nemusel jsi dneska na krku laso mít.
\end{xverse}


\begin{xverse}{R. }
Tak kopni do tý bedny ...
\end{xverse}


\begin{xverse}{4. }
Až kopneš do tý bedny, jak se to dělává,
do krku mi zvostane jen dírka mrňavá,
jenom dírka mrňavá a k smrti jenom krok,
má to smutnej konec, a whisky ani lok.
\end{xverse}


\begin{xverse}{R. }
Tak kopni do tý bedny, ať panstvo nečeká,
jsou dlouhý schody do nebe a štreka daleká
do nebeskýho baru, já sucho v krku mám,
tak kopni do tý bedny!
\end{xverse}


\end{song}

\begin{song}{Blátivá cesta}{Pacifik}

\begin{xverse}{R. }
[C]Blátivou, [Emi]blátivou cestou [F]dál nechceš [C]jít,
kde jen [F]máš touhu [C]blázni[Emi]vou,
kde jen [Dmi]máš, co chtěl jsi [G]mít, chtěl jsi [C]mít.
\end{xverse}

\begin{xverse}{1. }
[Ami]A tak se [G]koukáš, jak si [Ami]kolem hrajou děti,
[C]ve slunci [G]kotě [Ami]usíná,
[F]a jak si [G]před hospodou [C]vyprávějí [Ami]kmeti,
[D7]{}život prej stále začí[G]ná.
\end{xverse}

\begin{xverse}{2. }
Z města tě vyhánějí ocelový stíny,
jak dříve šel bys asi rád,
z bejvalejch cest ti zbyly potrhaný džíny,
čas běží, je to ale znát.
\end{xverse}

\begin{xverse}{R. }
Blátivou, blátivou ...
\end{xverse}

\begin{xverse}{3. }
Na poli pokoseným přepočítáš snopy,
do trávy hlavu položíš,
zdá se ti o holkách, co oči vždycky klopí,
po jiným ani netoužíš.
\end{xverse}

\begin{xverse}{4. }
Měkký jsou stíny, dole zrcadlí se řeka,
nad jezem kolébá se prám,
kolem je ticho, že i vlastní hlas tě leká,
a přesto necítíš se sám.
\end{xverse}

\begin{xverse}{R. }
Blátivou, blátivou ...
\end{xverse}

\end{song}

\begin{song}{Blues folsomské věznice}{Greenhorns}

\begin{xverse}{1. }
Můj [G]děda bejval blázen, texaskej ahasver,
a na půdě nám po něm zůstal [G7]ošoupanej kvér,
ten [C]kvér obdivovali všichni kámoši z oko[G]lí
a [D7]máma mi říkala: ``Nehraj si s tou pisto[G]lí!''
\end{xverse}

\begin{xverse}{2. }
Jenže i já byl blázen, tak zralej pro malér,
a ze  zdi jsem sundával tenhleten dědečkův kvér,
pak s kapsou vyboulenou chtěl jsem bejt chlap all right
a s holkou vykutálenou hrál jsem si na Bonnie and Clyde.
\end{xverse}

\begin{xverse}{3. }
Ale udělat banku, to není žádnej žert,
sotva jsem do ní vlítnul, hned zas vylít' jsem jak čert,
místo jako kočka já utíkám jak slon,
takže za chvíli mě veze policejní anton.
\end{xverse}

\begin{xverse}{4. }
Teď okno mřížovaný mi říká, že je šlus,
proto tu ve věznici zpívám tohle Folsom Blues.
pravdu měla máma, radila: ``Nechoď s tou holkou!'',
a taky mi říkala: ``Nehraj si s tou pistolkou!''
\end{xverse}

\end{song}

\begin{song}{Betty}{Hop trop}

\begin{xverse}{1. }
Já na [Gmi]plachtu svýho vozu jako [F]každej jsem si [D]psal:
``Hrab a [Gmi]dři anebo umři!'' a pak [F]na západ se [D]hnal,
z plachty [A]dávno jsou už cáry, ale [G]heslo platí [D]dál,
mě [G]vítá Kali[D]fornie, tak [A7]nač bych umí[D]ral. [D7]{}
\end{xverse}

\begin{xverse}{R. }
Betty, [G]vyndej z bedny soudek, rozžvejkám a spolknu [Ami]{}špunt,
upíchnem se právě tady, v ruce [D]{}žmoulám slibnej [G]grunt,
doufám, že to s nima zmáknu, vodsaď dál už nepu[Ami]dem,
navěky snad přece smůla nebu[D]de mým osu[G]dem. [Gmi|]{}
\end{xverse}

\begin{xverse}{2. }
Zarazíme první kolík, druhej támhle musí bejt,
za potok dej ty dva další, budeme v něm zlato mejt,
jedu sehnat ňakej ouřad, Betty, pojď mě vobejmout,
tebe přiklepli mi tenkrát, teď i dílec přiklepnou.
\end{xverse}

\begin{xverse}{R. }
Betty, vyndej z bedny soudek ...
\end{xverse}

\end{song}

\begin{song}{Divoké koně}{Jarek Nohavica}

\begin{xverse}{1. }
/: [Emi]Já viděl [Ami]divoké [Emi]koně, [G]běželi [Ami]soumra[Emi]kem, :/
[Ami]vzduch [Emi]těžký [Ami]byl a divně [Emi]voněl [Adim]{\,\,}tabá[C]kem,
[Ami]vzduch [Emi]těžký [Ami]byl a divně [Emi]voněl [H7]tabá[Emi]kem.
\end{xverse}

\begin{xverse}{2. }
Běželi, běželi bez uzdy a sedla krajinou řek a hor,
sper to čert, jaká touha je to vedla za obzor?
\end{xverse}

\begin{xverse}{3. }
Snad vesmír nad vesmírem, snad lístek na věčnost,
naše touho, ještě neumírej, sil máme dost.
\end{xverse}

\begin{xverse}{4. }
V nozdrách sládne zápach klisen na břehu jezera,
milování je divoká píseň večera.
\end{xverse}

\begin{xverse}{5. }
Stébla trávy sklání hlavu, staví se do šiku,
král s dvořany přijíždí na popravu zbojníků.
\end{xverse}

\begin{xverse}{6. }
Chtěl bych jak divoký kůň běžet, běžet, nemyslet na návrat,
s koňskými handlíři vyrazit dveře, to bych rád.
\end{xverse}

\begin{xverse}{7. }
Já viděl divoké koně ...
\end{xverse}

\end{song}
\chords{\chordAdim}

\begin{song}{Dokud se zpívá}{Jarek Nohavica}

\begin{xverse}{1. }
Z [C]Těšína [Emi]vyjíždí [Dmi7]vlaky co [F]{}čtvrthodi[C]nu, [Emi]{\qquad} [Dmi7]{\qquad} [G]{\qquad}
[C]včera jsem [Emi]nespal a [Dmi7]ani dnes [F]nespoči[C]nu,  [Emi]{\qquad} [Dmi7]{\qquad} [G]{\qquad}
[F]svatý Me[G]dard, můj pa[C]tron, ťuká [Ami]si na če[G]lo,
ale [F]dokud se [G]zpívá, [F]ještě se [G]neumře[C]lo.  [Emi]{\qquad} [Dmi7]{\qquad} [G]{\qquad}
\end{xverse}

\begin{xverse}{2. }
Ve stánku koupím si housku a slané tyčky,
srdce mám pro lásku a hlavu pro písničky,
ze školy dobře vím, co by se dělat mělo,
ale dokud se zpívá, ještě se neumřelo.
\end{xverse}

\begin{xverse}{3. }
Do alba jízdenek lepím si další jednu,
vyjel jsem před chvílí, konec je v nedohlednu,
za oknem míhá se život jak leporelo,
ale dokud se zpívá, ještě se neumřelo.
\end{xverse}

\begin{xverse}{4. }
Stokrát jsem prohloupil a stokrát platil draze,
houpe to, houpe to na housenkové dráze,
i kdyby supi se slítali na mé tělo,
tak dokud se zpívá, ještě se neumřelo.
\end{xverse}

\begin{xverse}{5. }
Z Těšína vyjíždí vlaky až na kraj světa,
zvedl jsem telefon a ptám se:"Lidi, jste tam?"
A z veliké dálky do uší mi zaznělo,
že dokud se zpívá, ještě se neumřelo,
že dokud se zpívá ještě se neumřelo
\end{xverse}

\end{song}
\chords{\chordDmiSeven}

\begin{song}{Dům U vycházejícího slunce}{}

\begin{xverse}{1. }
Snad [Ami]znáš ten [C]dům za [D]New Or[F]leans,
ve [Ami]{}štítu [C]znak slunce [E]má,
je to [Ami]dům, kde [C]lká sto [D]chlapců ubo[F]hejch
a [Ami]kde jsem [E]zkejs' i [Ami]já. \ \  [C|]{} [D|]{} [F|]{} [Ami|]{} [E|]{} [Ami|]{} [E|]{}
\end{xverse}

\begin{xverse}{2. }
Mý mámě Bůh dal věnem svatebním
jen prát a šít blue jeans,
táta můj se flákal jen
sám po New Orleans.
\end{xverse}

\begin{xverse}{3. }
Bankrotář se zhroutil před hernou,
jenom bídu svou měl a chlast,
k putykám pak táh' tu pouť mizernou
a znal jen pít a krást.
\end{xverse}

\begin{xverse}{4. }
Být matkou, dám svým synům
lepší dům, než má kdo z vás,
ten dům, kde spím, má emblém sluneční,
ale je v něm sníh a mráz.
\end{xverse}

\begin{xverse}{5. }
Kdybych směl se hnout z těch kleští,
pěstí vytrhnout tu mříž,
já jak v snách bych šel do New Orleans
a měl tam k slunci blíž.
\end{xverse}

\begin{xverse}{6. }
=\ 1.
\end{xverse}

\end{song}

\begin{song}{Fi-li-mi}{Spirituál Kvintet}

%TODO o pul stupne dolu

\begin{xverse}{1. }
[Fmi]{Čert} aby vzal už tuhle trať, kdo [G#]hledáš práci, tak se ztrať,
[Fmi]{že} nemáš prachy, no tak ať, jó, tak se [Cmi]na to [Fmi]dívám!
\end{xverse}

\begin{xverse}{R. }
[Fmi]Fi-li-mi-jo-ri-jú-ri-ej,  [G#]fi-li-mi-jo-ri-jú-ri-ej,
[Fmi]fi-li-mi-jo-ri-jú-ri-ej, vo tom [Cmi]si teď [Fmi]zpívám.
\end{xverse}

\begin{xverse}{2. }
Jen pražec chop a kolej suň, chyť lano, táhni jako kůň,
pod tíhou jako medvěd fuň, jó, tak se na to dívám!
\end{xverse}

\begin{xverse}{R. }
Fi-li-mi ...
\end{xverse}

\begin{xverse}{3. }
Z kůže se loupeš jako had, je vedro, že by jeden pad',
na vodu smíš jen vzpomínat, jó, tak se na to dívám!
\end{xverse}

\begin{xverse}{R. }
Fi-li-mi ...
\end{xverse}

\begin{xverse}{4. }
Když konečně máš vody dost, určitě přes ni stavíš most,
kláda ti ráda zlomí kost, jó, tak se na to dívám!
\end{xverse}

\begin{xverse}{R. }
Fi-li-mi ...
\end{xverse}

\begin{xverse}{5. }
Na rukách už jsem potěžkal většinu těch okolních skal,
ještě to cejtí každej sval, jó, vo tom si teď zpívám!
\end{xverse}

\begin{xverse}{R. }
Fi-li-mi ...
\end{xverse}

\begin{xverse}{6. }
Slunce už dělá z trávy troud,
jen kdybych se směl vodsaď hnout,
na tuhle trať zapomenout, jó, tak se na to dívám!
\end{xverse}

\begin{xverse}{R. }
Fi-li-mi ...
\end{xverse}

\begin{xverse}{6. }
Jen Bůh mi víru zachovej a nasednout mi sílu dej,
můj [Fmi]vagón bude [C#]pérovej,
jó, [Bbmi]vo tom [Cmi]si teď [Fmi]zpívám!
\end{xverse}

\begin{xverse}{R. }
Fi-li-mi ...
\end{xverse}

\end{song}

\begin{song}{Fram}{Wabi Daněk}

\begin{xverse}{1. }
[Ami]Zas mě to táhne o kus [Ami6|]{dál,} [Ami]{zas} nemám doma nikde [D]stání,
desítky důvodů si [Dmi]sháním, [E7]už abych na cestu se [Ami]dal. [Ami6]{}
\end{xverse}

\begin{xverse}{2. }
Pelikán křídly zamával, vítr je příhodný a stálý,
za námi slunce mosty pálí, tak proč bych ještě vyčkával.
\end{xverse}

\begin{xverse}{R. }
[Ami]Klenotník měsíc zavřel krám, [Ami/G#|]{} [Ami/G]{z vý}kladu [Ami/F#]svoje šperky [Emi]sklízí,
[Dmi]obrysy domů v dálce [Emi]mizí, [Dmi]tak naposled ti zamá[E7]vám
z paluby lodi jménem [Ami]Fram.
\end{xverse}

\begin{xverse}{3. }
Dávno už vyvětral se dým mých věčných cigaret a dýmek,
ty žiješ jenom ze vzpomínek, a já se stále nevracím.
\end{xverse}

\begin{xverse}{4. }
Námořní mapy pokryl prach, mé knihy nikdo neutírá,
nevíš, zda právě neumírám tam někde na ledových krách.
\end{xverse}

\begin{xverse}{R. }
Klenotník měsíc zavřel krám, z výkladu svoje šperky sbírá,
chlap jen tak lehce neumírá, na modré lodi jménem Fram
tě za pár roků vyhledám.
\end{xverse}

\end{song}
\chords{ \chordAmiSix \chordAmiGis \chordAmiG \chordAmiFis }

\begin{song}{Frankie Dlouhán}{}

\begin{xverse}{1. }
[G]Kolik je smutného, když [C]mraky černé [G]jdou
[G]lidem nad hla[D]vou, [C]smutnou dála[G]vou,
[G]já slyšel příběh, který [C]velkou pravdu [G]měl,
za čas odle[D]těl, [C]každý zapom[G]něl.
\end{xverse}

\begin{xverse}{R. }
Měl kapsu [D]prázdnou Frankie Dlouhán,
po státech [C]toulal se jen [G]sám,
a že byl [C]veselej, tak [G]každej měl ho [D]rád.
Tam ruce [C]k dílu mlčky přiloží a [G]zase jede [Emi]dál,
a [C]každej kdo s ním [D]chvilku byl,
tak [C]dlouho [D]se pak [G]smál.
\end{xverse}

\begin{xverse}{2. }
Tam kde byl pláč, tam Frankie hezkou píseň měl,
slzy neměl rád, chtěl se jenom smát.
A když pak ranče večer tiše usínaj,
Frankův zpěv jde dál, nocí s písní dál.
\end{xverse}

\begin{xverse}{3. }
Tak Frankieho vám jednou našli, přestal žít,
jeho srdce spí, tiše klidně spí.
Bůh ví jak,za co, tenhle smíšek konec měl,
farář píseň pěl, umíráček zněl.
\end{xverse}

\end{song}

% \begin{song}{František}{Buty}
%
% \begin{xverse}{1. }
% [G]Na hladinu rybníka svítí sluníč[C]ko
% [Emi]a kolem stojí v hustém kruhu [G]topoly,
% [Ami]které tam zasadil jeden hodný [Hmi]{čl}ověk,
% [Ami]jmenoval se František [D]Dobrota.
% \end{xverse}
%
%
% \begin{xverse}{2. }
% František Dobrota, rodák z blízké vesnice,
% měl hodně dětí a jednu starou babičku,
% která když umírala, tak mu řekla: Františku,
% teď dobře poslouchej, co máš všechno udělat.
% \end{xverse}
%
%
% \begin{xverse}{R. }
% [C]Balabambam, balabambam, [D|]{} [C|]{}
% balabambam, balabambam,  [D|]{} [C|]{}
% balabambam, balabambam,  [D|]{} [C|]{}
% [Ami]a kolem rybníka nahusto nasázet [D]topoly
% \end{xverse}
%
%
% \begin{xverse}{3. }
% František udělal všechno, co mu řekla,
% balabambam, balabambam,
% a po snídani poslal děti do školy,
% žebriňák s nářadím dotáhl od chalupy k rybníku,
% vykopal díry a zasadil topoly.
% \end{xverse}
%
%
% \begin{xverse}{4. }
% Od té doby vítr na hladinu nefouká,
% takže je klidná jako velké zrcadlo,
% sluníčko tam svítí vždycky rádo,
% protože v něm vidí Františkovu babičku.
% \end{xverse}
% \end{song}

\begin{song}{Hajnej Hruška}{Hop trop}

\begin{xverse}{1. }
[Emi]Na pařez já [C7]used[H7]nu si v [Emi]lesním polo[H7_]{še}[Emi]{ru}
a na hajnýho [C7]vzpome[H7]nu si, [Emi]jenž má hezkou [H7]dce[Emi]ru,
[D7]na hajnýho [G]Hrušku a [D7]jeho dceru - [G]samej skvost,
[D7]jenže von má [G]pušku a s [F#]puškou střeží [H7]dcery ctnost.
\end{xverse}

\begin{xverse}{2. }
Na pařezu přemejšlím, a dá to velkou fušku,
jak bych vyzrál na hajnýho, na hajnýho Hrušku,
na Hrušku a jeho zbraň a křepeláka Azora,
kterej hlídá jako saň vchod do hájovny ze dvora.
\end{xverse}

\begin{xverse}{R. }
A [Emi]{ště}ká přitom [C7]na srnce, [Emi]na datly i [C7]{žlů}vy,
[Emi]na ťuhýka [C7]na trnce, [Emi]na vejry i [C7]sůvy,
[Ami]na chudáka [Emi]vandráka von [Ami]{ště}ká ponej[Emi]více,
[Ami]vidí ve mně [Emi]pytláka, co [F#]líčí na za[H7]jíce.
\end{xverse}

\begin{xverse}{*. }
[Emi]Vrr haf, vrr haf, [C7]sypej [H7]si to, [Emi]vrr haf, vrr haf, [H7]fuj fuj [Emi]fuj,
[Emi]vrr haf, vrr haf, [C7]sypej [H7]si to, [Emi]padej pryč a [H7]upa[Emi]luj!
\end{xverse}

\begin{xverse}{3. }
Hruška zbystří sluch i zrak a vzkřikne:"Namouvěru,
zas je tu ten darebák, co zprznit mi chce dceru,
zas je tu ten chuligán, co slídí, kde je dcerka,
jenže já se do něj dám a proženu mu perka!"
\end{xverse}

\begin{xverse}{4. }
A už běží, v hubě pěnu, dělá dlouhý kroky,
pušku k palbě připravenu, má v ní srnčí broky,
letí, letí jeko blesk ze světnice na dvorek,
ve vočích má divnej lesk i jeho pejsek Azorek.
\end{xverse}

\begin{xverse}{R. }
Ten, kterej má rád štvanice na lišky i kance,
když vypukne pranice, vždycky v ní má šance,
on si troufne na zvíře tak, jako medvěd velký,
a milýho trempíře chce kousnout do pr...avý ruky.
\end{xverse}

\begin{xverse}{5. }
Vím, jak vyzrát na hajnýho, ba i na Azora:
vyštuduju na vrchního lesů revizora,
až přijedu na kontrolu se služební volhou,
postavím je do pozoru, což je mojí touhou.
\end{xverse}

\begin{xverse}{6. }
Počkej, hajnej, povím na tě, že jsi prodal jedli,
žes' ji střelil nastojatě, a on bude zbledlý,
``smilujou se, revizore, vždyť mám doma dceru,''
jenže, Hruško, na tvou dceru já už dávno ... hej, beru!
\end{xverse}

\begin{xverse}{Rec}
Tak teda, tatínku, do smrti dobrý, ne?
\end{xverse}

\begin{xverse}{*. }
A to bude asi všecko, na zdi visí puška,
pod ní kolíbá mi děcko můj tchán - hajnej Hruška ...
\end{xverse}

\end{song}

\begin{song}{Hlídač krav}{Jarek Nohavica}

\begin{xverse}{1. }
[C]Když jsem byl malý, říkali mi naši:
``Dobře se uč a jez chytrou kaši,
[F]až jednou vyrosteš, [G]budeš doktorem [C]práv,
takový doktor sedí pěkně v suchu,
bere velký peníze a škrábe se v uchu,''
[F]já jim ale na to řek': ``[G]Chci být hlídačem [C]krav.''
\end{xverse}

\begin{xverse}{R. }
Já chci [C]mít čapku s bambulí nahoře,
jíst kaštany, mýt se v lavoře,
[F]od rána po celý [G]den zpívat si [C]jen,
zpívat si: pam pam pa dam ...
\end{xverse}

\begin{xverse}{2. }
K vánocům mi kupovali hromady knih,
co jsem ale vědět chtěl, to nevyčet' jsem z nich:
nikde jsem se nedozvěděl, jak se hlídají krávy,
ptal jsem se starších a ptal jsem se všech,
každý na mě hleděl jako na pytel blech,
každý se mě opatrně tázal na moje zdraví.
\end{xverse}

\begin{xverse}{R. }
Já chci ...
\end{xverse}

\begin{xverse}{3. }
Dnes už jsem starší a vím, co vím,
mnohé věci nemůžu a mnohé smím,
a když je mi velmi smutno, lehnu si do mokré trávy,
s nohama křížem a s rukama za hlavou
koukám nahoru na oblohu modravou,
kde se mezi mraky honí moje strakaté krávy.
\end{xverse}

\begin{xverse}{R. }
Chtěl bych mít ...
\end{xverse}

\end{song}

\begin{song}{Hotel Hillary}{Poutníci}

\begin{xverse}{1. }
Tvař se [Ami]trochu nostalgicky, už tě nikdy nepotkám, [Emi|]{}
[F]máš to jistý [G]provždycky, nastav [Ami]uši vzpomínkám,
jak tě znám, i v tuhle chvíli měl bys řeči peprný, [Emi|]{}
jak [F]tenkrát, když nám [G]tvrdili, že je [Ami]vítr stříbrný.
\end{xverse}

\begin{xverse}{R. }
A [F]tváře měli kožený, my jim zdrhli z průvodu,
zaho[Dmi]dili lampióny a [D]našli hospodu,
ale [F]taky Jacquese Brela a s ním smutek z cizích vin
a [Dmi]{žádo}stivost těla a pak [D]radost z volovin,
a ta nám [Ami]zbejvá.
\end{xverse}

\begin{xverse}{2. }
Po večerech pro diváky dělali jsme kašpary,
pak na zemi dva spacáky - náš Hotel Hillary,
slavný sliby jsme už znali, i to, jak se neplní,
a cenzoři nám kázali o správným umění.
\end{xverse}


\begin{xverse}{R. }
A tváře měli kožený ...
\end{xverse}


\begin{xverse}{3. }
A tak válčím s nostalgií, bují ve mně jako mech,
a pořád všechno slibují starý hesla na domech,
ty jsi splatil všechny dluhy, i za Hotel Hillary,
a já vyhážu ty černý stuhy funebrákům navzdory.
\end{xverse}


\begin{xverse}{R. }
Vždyť mají tváře kožený, my jim zdrhnem z průvodu,
zahodíme lampióny a najdem hospodu,
a tam svýho Jacquese Brela a s ním smutek z cizích vin
a žádostivost těla a pak radost z volovin,
/: a ta nám zbejvá. :/
\end{xverse}
\end{song}

\begin{song}{Hudsonský šífy}{Wabi Daněk}

\begin{xverse}{1. }
Ten, kdo [Ami]nezná hukot vody lopat[C]kama vířený
jako [G]já, jó, jako [Ami]já,
kdo hudsonský slapy nezná sírou [G]pekla sířený,
ať se [Ami]na hudsonský [G]{}šífy najmout [Ami_]{dá}, [G]joho[Ami]ho.
\end{xverse}

\begin{xverse}{2. }
Ten, kdo nepřikládal uhlí, šíf když na mělčinu vjel,
málo zná, málo zná,
ten, kdo neměl tělo ztuhlý, až se nočním chladem chvěl,
ať se na hudsonský šífy najmout dá, johoho.
\end{xverse}

\begin{xverse}{R. }
A[F]hoj, páru tam [Ami]hoď,
ať [G]do pekla se dříve dohra[Ami]bem,
[G]joho[Ami]ho,  [G]joho[Ami]ho.
\end{xverse}

\begin{xverse}{3. }
Ten, kdo nezná noční zpěvy zarostenejch lodníků
jako já, jó, jako já,
ten, kdo cejtí se bejt chlapem, umí dělat rotyku,
ať se na hudsonský šífy najmout dá, johoho.
\end{xverse}

\begin{xverse}{4. }
Ten, kdo má na bradě mlíko, kdo se rumem neopil,
málo zná, málo zná,
kdo necejtil hrůzu z vody, kde se málem utopil,
ať se na hudsonský šífy najmout dá, johoho.
\end{xverse}

\begin{xverse}{R. }
Ahoj, páru tam hoď ...
\end{xverse}

\begin{xverse}{5. }
Kdo má roztrhaný boty, kdo má pořád jenom hlad
jako já, jó, jako já,
kdo chce celý noci čuchat pekelnýho vohně smrad,
ať se na hudsonský šífy najmout dá, johoho.
\end{xverse}

\begin{xverse}{6. }
Kdo chce zhebnout třeba zejtra, komu je to všechno fuk,
kdo je sám, jó, jako já,
kdo má srdce v správným místě, kdo je prostě príma kluk,
ať se na hudsonský šífy najmout dá, johoho.
\end{xverse}

\end{song}

\begin{song}{Jarní tání}{Brontosauři}

\begin{xverse}{1. }
Když první [Hmi]tání [Emi]cestu sněhu [D]zkříží
[G]a nad [Emi]ledem se [F#]voda obje[Hmi]ví,
voňavá zem se [Emi]sněhem tiše [D]plíží,
[G]tak nějak [Emi]líp si [F#]balím, proč, Bůh [Hmi]ví.
\end{xverse}

\begin{xverse}{R. }
Přišel čas [G]slunce, zrození a [D]tratí,
na kterejch [G]potkáš kluky ze všech [D]stran, [F#]{}
/: Hubenej [Hmi]Joe, Čára, Ušoun se ti [Emi]vrátí,
oživne [G]kemp, [F#]jaro, vítej k [Hmi]nám. :/
\end{xverse}

\begin{xverse}{2. }
Kdo ví, jak voní země, když se budí,
pocit má vždy, jak zrodil by se sám,
jaro je lék na řeči, co nás nudí,
na lidi, co chtěj' zkazit život nám.
\end{xverse}

\begin{xverse}{R. }
Přišel čas slunce ...
\end{xverse}

\begin{xverse}{3. }
Zmrznout by měla, kéž by se tak stalo,
srdce těch pánů, co je jim vše fuk,
pak bych měl naději, že i příští jaro
bude má země zdravá jako buk.
\end{xverse}

\begin{xverse}{R. }
Přišel čas slunce ...
+ oživne [G]kemp, [F#]jaro, vítej [Hmi]k nám ...
\end{xverse}

\end{song}

\begin{song}{Jdem zpátky do lesů}{Žalman}

\begin{xverse}{1. }
[Ami7]Sedím na kolejích, [D]které nikam neve[G]dou, [C|]{} [G|]{}
[Ami7]koukám na kopretinu, jak [D]miluje se s lebe[G]dou, [C|]{} [G|]{}
[Ami7]mraky vzaly slunce [D]zase pod svou ochra[G]nu, [Emi|]{}
[Ami7]jen ty nejdeš, holka zlatá, [D]kdypak já tě dosta[G]nu? [D|]{}
\end{xverse}

\begin{xverse}{R. }
Z [G]ráje, my vyhnaní z [Emi]ráje,
kde není už [Ami7]místa, [C7]prej něco se [G]chystá, [D|]{}
z [G]ráje nablýskaných [Emi]plesů
jdem zpátky do [Ami7|]{lesů} [C7]{}za nějaký [G]{}čas.
\end{xverse}

\begin{xverse}{2. }
Vlak nám včera ujel ze stanice do nebe,
málo jsi se snažil, málo šel jsi do sebe,
šel jsi vlastní cestou, a to se zrovna nenosí,
i pes, kterej chce přízeň, napřed svýho pána poprosí.
\end{xverse}

\begin{xverse}{R. }
Z ráje...
\end{xverse}

\begin{xverse}{3. }
Už tě vidím z dálky, jak máváš na mě korunou,
a jestli nám to bude stačit, zatleskáme na druhou,
zabalíme všechny, co si dávaj' rande za branou,
v ráji není místa, možná v pekle se nás zastanou.
\end{xverse}

\begin{xverse}{R. }
Z ráje...
\end{xverse}

\end{song}
% \chords{ \chordAmiSeven }

% \begin{song}{Jednou mi fotr povídá}{Ivan Hlas}
%
% \begin{xverse}{1. }
% [A7]Jednou mi fotr povídá, [D7]zůstali jsme už sami dva,
% že [E7]si chce začít taky trochu [A7]{žít},
% nech si to projít palicí a nevracej se s vopicí,
% snaž se mě hochu trochu pochopit.
% \end{xverse}
%
%
% \begin{xverse}{R. }
% Já [E7]{šel}, šel dál, baby, [A7]kam mě Pánbůh zval,
% já [E7]{šel}, šel dál, baby, a [D7]furt jen tancoval,
% [A7]na každý divný hranici, [D7]na policejní stanici
% [E7]hrál jsem jenom rock'n'roll for [A7]you.
% \end{xverse}
%
%
% \begin{xverse}{2. }
% Přiletěl se mnou černej čáp, zobákem dělal klapy klap
% a nad kolíbkou Elvis Presley stál,
% obrovskej bourák v ulici, po boku krásnou slepici
% a lidi šeptaj: přijel ňákej král.
% \end{xverse}
%
%
% \begin{xverse}{R. }
% Já šel, šel dál, baby, kam mě Pánbůh zval, ...
% \end{xverse}
%
%
% \begin{xverse}{3. }
% Pořád tak ňák nemohu, chytit štěstí za nohu
% a nemůžu si najít klidnej kout,
% bláznivý ptáci začnou řvát a nový ráno šacovat
% a do mě pustí vždycky silnej proud.
% \end{xverse}
%
%
% \begin{xverse}{R. }
% Já šel, šel dál, baby, kam mě Pánbůh zval, ...
% \end{xverse}
%
% \end{song}

\begin{song}{Kdysi a kdesi}{Šlitr/Suchý}

\begin{xverse}{1. }
[C]Kdysi a kdesi [F]bylo nebylo,
[G]minomety metaly a [C]dělo pálilo,
pan velitel roty na to [F]nebral ohledy,
[G]{řek'}, abych si obul boty [C]{a šel} na zvědy.
\end{xverse}


\begin{xverse}{R. }
[C]Vyfasujem kvér a flašku džinu,
skrze tmu si [G]tunel vydla[C]bem,
přes Waterloo za Hercegovinu,
podél Mississippi až do [G]{Ústí} nad La[C]bem.
\end{xverse}


\begin{xverse}{2. }
I vyšel jsem za malou chvíli směrem k severu,
aby Turci netušili, že je nežeru,
že mám bodák na bodání, pažbu k bušení,
Taliáni nemaj' zdání ani tušení.
\end{xverse}


\begin{xverse}{R. }
Vyfasujem kvér a flašku džinu ...
\end{xverse}


\begin{xverse}{3. }
V zákopech si Němci tiše seděli,
aniž tu neděli o mě něco věděli,
času bylo málo a mě to hnalo tam,
kde se zdálo, že Tatarům hlavu zamotám.
\end{xverse}


\begin{xverse}{R. }
Vyfasujem kvér a flašku džinu ...
\end{xverse}

\begin{xverse}{4. }
Švédové si právě pekli vepřový,
když tu jsem na ně náhle vyběh' ze křoví,
jejich jediná mě střela minula,
a tak jsem tu bitvu v Kentu vyhrál tři-nula.
\end{xverse}


\begin{xverse}{R. }
Vyfasujem kvér a flašku džinu ...
\end{xverse}

\begin{xverse}{R. }
Měl jsem jednou kvér a flašku džinu,
tmou si rádi, kamarádi, tunel vydlabem,
přes Waterloo za Hercegovinu,
podél Mississippi až do Ústí nad Labem.
\end{xverse}

\end{song}

\begin{song}{Kláda}{Hop trop}

\begin{xverse}{1. }
[Hmi]Celý roky prachy jsem si skládal,
[D]nikdy svýho [A]floka nepro[Hmi]pil,
vod lopaty měl vohnutý záda,
[D]paty od baráku [A]jsem neodle[Hmi]pil,
[A]nikdo neví, do čeho se [Hmi]někdy zamotá,
tohle [D]já už [A]dávno pocho[Hmi]pil.
\end{xverse}


\begin{xverse}{2. }
Taky kdysi vydělat jsem toužil,
brácha řek' mi, že by se mnou šel,
tak jsem háky, lana, klíny koupil,
a sekyru jsme svoji každej doma měl,
a plány veliký, jak fajn budem se mít,
nikdo z nás pro holku nebrečel.
\end{xverse}


\begin{xverse}{R. }
[G]Duní [D]kláda kory[Emi]tem, bacha [Hmi]dej, [A]hej, bacha [Hmi]dej!
S [G]tou si, [D]bráško, nety[Emi]kej, nety[F#]kej !
\end{xverse}


\begin{xverse}{3. }
Dřevo dostat k pile, kde ho koupí,
není těžký, vždyť jsme fikaný,
ten rok bylo jaro ale skoupý,
a teď jsme na dně my i vory svázaný,
a k tomu můžem říct jen, že nemáme nic,
jen kus práce nedodělaný.
\end{xverse}


\begin{xverse}{R. }
Duní kláda ...
\end{xverse}
\end{song}

\begin{song}{Kluziště}{Karel Plíhal}

\begin{xverse}{1. }
[C]Strejček [Emi7/H|]{kovář} [Ami7]chytil k[C/G]leště,[Fmaj7] uštíp' z [C]noční [Fmaj7_]{oblo}[G]{hy}
[C]jednu [Emi7/H|]{malou} [Ami7]kapku [C/G]deště, [Fmaj7]ta mu sp[C]adla [Fmaj7]pod no[G]{hy,}
[C]nejdřív [Emi7/H|]{ale} [Ami7|]{chytil} [C/G]slinu, [Fmaj7]pak šáh' [C]kamsi [Fmaj7]pro pi[G]{vo,}
[C]pak při[Emi7/H|]{táhl} [Ami7]kovad[C/G]linu [Fmaj7|]{}a ob[C]rovský [Fmaj7_]{kladi}[G]{vo.}
\end{xverse}

\begin{xverse}{R. }
Zatím [C]tři bílé [Emi7/H]vrány pě[Ami7]kně za se[C/G]bou
kolem [Fmaj7]jdou, někam [C]jdou, do rytm[D7]u se kýva[G]jí,
tyhle [C]tři bílé [Emi7/H]{vrány} pěk[Ami7]ně za seb[C/G]ou
kolem [Fmaj7]jdou, někam [C]jdou, nedojd[Fmaj7]ou, nedo[C]jdou.
\end{xverse}

\begin{xverse}{2. }
Vydal z hrdla mocný pokřik ztichlým letním večerem,
pak tu kapku všude rozstřík' jedním mocným úderem,
celej svět byl náhle v kapce a vysoko nad námi
na obrovské mucholapce visí nebe s hvězdami.
\end{xverse}

\begin{xverse}{R. }
Zatím tři bílé vrány ...
\end{xverse}

\begin{xverse}{3. }
Zpod víček mi vytrysk' pramen na zmačkané polštáře,
kdosi mě vzal kolem ramen a políbil na tváře,
kdesi v dálce rozmazaně strejda kovář odchází,
do kalhot si čistí ruce umazané od sazí.
\end{xverse}
\end{song}
\chords{ \chordAmiSeven \chordFmajSeven }

\begin{song}{Kometa}{Jarek Nohavica}

\begin{xverse}{1. }
[Ami]Spatřil jsem kometu, oblohou letěla,
chtěl jsem jí zazpívat, ona mi zmizela,
[Dmi]zmizela jako laň [G7]u lesa v remízku,
[C]v očích mi zbylo jen [E7]pár žlutých penízků.
\end{xverse}

\begin{xverse}{2. }
Penízky ukryl jsem do hlíny pod dubem,
až příště přiletí, my už tu nebudem,
my už tu nebudem, ach, pýcho marnivá,
spatřil jsem kometu, chtěl jsem jí zazpívat.
\end{xverse}

\begin{xverse}{R. }
[Ami]O vodě, o trávě, [Dmi]o lese,
[G7]o smrti, se kterou smířit [C]nejde se,
[Ami]o lásce, o zradě, [Dmi]o světě
[E]a o všech lidech, co [E7]kdy žili na téhle [Ami]planetě.
\end{xverse}

\begin{xverse}{3. }
Na hvězdném nádraží cinkají vagóny,
pan Kepler rozepsal nebeské zákony,
hledal, až nalezl v hvězdářských triedrech
tajemství, která teď neseme na bedrech.
\end{xverse}

\begin{xverse}{4. }
Velká a odvěká tajemství přírody,
že jenom z člověka člověk se narodí,
že kořen s větvemi ve strom se spojuje
a krev našich nadějí vesmírem putuje.
\end{xverse}

\begin{xverse}{R. }
Na na na ...
\end{xverse}


\begin{xverse}{5. }
Spatřil jsem kometu, byla jak reliéf
zpod rukou umělce, který už nežije,
šplhal jsem do nebe, chtěl jsem ji osahat,
marnost mne vysvlékla celého donaha.
\end{xverse}


\begin{xverse}{6. }
Jak socha Davida z bílého mramoru
stál jsem a hleděl jsem, hleděl jsem nahoru,
až příště přiletí, ach, pýcho marnivá,
my už tu nebudem, ale jiný jí zazpívá.
\end{xverse}

\begin{xverse}{R. }
O vodě, o trávě, o lese,
o smrti, se kterou smířit nejde se,
o lásce, o zradě, o světě,
bude to písnička o nás a kometě ...
\end{xverse}

\end{song}

% \begin{song}{Král a klaun}{Karel Kryl}
%
% \begin{xverse}{1. }
% [D]Král [C]do boje [G]táh',[C][G] do [C]veliké [G]dálky,[C|]{} [G|]{}
% a s [C]ním do té [G]války [D7]jel na mezku [G]klaun,
% [D]než [C]hledí si [G]stáh' [C] [G] , tak z [C]výrazu [G]tváře [C|]{} [G|]{}
% [C]bys nepoznal [G]lháře, [D7]co zakrývá [G]strach.
% [D7]Tiše šeptal při té hrůze: "[G]Inter arma silent musae,"
% [A]místo zvonku cinkal brně[D7]ním, [C#7|]{} [D7|]{}
% [C]král do boje [G]táh' [C] [G] , do [C]veliké [G]dálky, [C|]{} [G|]{}
% a s [C]ním do té [G]války [D7]jel na mezku [G]klaun. [H|]{} [C|]{} [G|]{} [A7|]{}
% \end{xverse}
%
% \begin{xverse}{2. }
% Král do boje táh', a sotva se vzdálil,
% tak vesnice pálil a dobýval měst,
% klaun v očích měl hněv, když sledoval žháře,
% jak smývali v páře prach z rukou a krev.
% Tiše šeptal při té hrůze:"Inter arma silent musae,"
% místo loutny držel v ruce meč,
% král do boje táh', a sotva se vzdálil,
% tak vesnice pálil a dobýval měst.
% \end{xverse}
%
% \begin{xverse}{3. }
% Král do boje táh', s tou vraždící lůzou
% klaun třásl se hrůzou a odvetu kul,
% když v noci byl klid, tak oklamal stráže
% a, nemaje páže, sám burcoval lid.
% Všude křičel do té hrůzy, ve válce že mlčí Múzy,
% muži by však mlčet neměli,
% král do boje táh', s tou vraždící lůzou
% klaun třásl se hrůzou a odvetu kul.
% \end{xverse}
%
% \begin{xverse}{4. }
% Král do boje táh', a v červáncích vlídných
% zřel, na čele bídných jak vstříc jde mu klaun,
% když západ pak vzplál, tok potoků temněl,
% klaun tušení neměl jak zahynul král:
% kdekdo křičel při té hrůze:"Inter arma silent musae,"
% krále z toho strachu trefil šlak,
% klaun tiše se smál a zem žila dále
% a neměla krále, klaun na loutnu hrál,
% [D7]klaun na loutnu [G]hrál, [D7]klaun na loutnu [G]hrál ...
% \end{xverse}
%
% \end{song}
% \chords{\chordCisSeven}

\begin{song}{Krutá válka}{Spirituál kvintet}

\begin{xverse}{1. }
Tmou [E]zní zvony [C#mi]z dálky, o [F#mi]{}čem to, milý, [G#mi]sníš,
[G#]hoří [A]dál plamen [F#mi]války a [E]rá[A6]no je [F#mi]blíž,
[H7]chci [E]být stále s [C#mi]tebou, až [F#mi]trubka začne [G#mi]znít,
[G#]lásko [A]má, vem mě s [F#mi]sebou! [E]Ne, to [A]nesmí [E]být!
\end{xverse}

\begin{xverse}{2. }
Můj šál skryje proud vlasů, na pás pak připnu nůž,
poznáš jen podle hlasu, že já nejsem muž,
tvůj kapitán tě čeká, pojď, musíme už jít,
noc už svůj kabát svléká ... Ne, to nesmí být!
\end{xverse}

\begin{xverse}{3. }
Až dým vítr stočí, tvář změní pot a prach,
do mých dívej se očí, tam není strach,
když výstřel tě raní, kdo dával by ti pít,
hlavu vzal do svých dlaní ... Ne, to nesmí být!
\end{xverse}

\begin{xverse}{4. }
Ach, má lásko sladká, jak mám ti to jen říct,
každá chvíle je krátká a já nemám víc,
já mám jenom tebe, můj dech jenom tvůj zná,
nech mě jít vedle sebe ... Pojď, lásko má!
\end{xverse}

\end{song}
\chords{\chordASix}

\begin{song}{Krysař}{Pacifik}

\begin{xverse}{1. }
[Emi]Bylo to v dobách [C]osvícených, před [D]branou krysař [Emi]stál,
[Emi]městem šlo jako [Ami]pohlazení, když [D]na píšťalu [Emi]hrál.
[G]Viděl jak brány [C]otvírají, [Ami]jak každý šel mu [D]vstříc,
[Emi]{že} lidé jeho [Ami]píseň znají, [D]každý chtěl slyšet [Emi]víc.
\end{xverse}


\begin{xverse}{2. }
Bylo to v dobách osvícených, před branou krysař stál,
až jednou se svým doprovodem přišel i sám pan král,
prodej mi flétnu, chlapče milý, já všechno zlato ti za ni dám,
ten nápěv tolik roztomilý, dávno v srdci mám.
\end{xverse}


\begin{xverse}{3. }
Povídá krysař: pane králi, ať píšťalka je tvou,
ať v kámen nikdy nepromění písničku nevinnou.
Vyjdi s tou písní mezi lidi, na každou ze všech cest,
ať slepý rázem krásu vidí, otvírej brány měst.
\end{xverse}


\begin{xverse}{4. }
Otvírej srdce zatvrzelá, tulákům lámej hůl,
ať s tebou zpívá země celá, dej království všem půl.
Krejčíkům plátno, rybářům síť a včelám květů pyl,
já zase musím svou cestou jít, ty zpívej ze všech sil.
\end{xverse}


\begin{xverse}{5. }
Bylo to v dobách osvícených, před branou krysař stál,
městem šlo jako pohlazení, když na píšťalu hrál.
Kde jsou ty doby osvícené, zatímco svět šel dál,
kde jsou ty písně zanícené, kam zmizel ten, co hrál.
\end{xverse}
\end{song}

\begin{song}{Kulatý vobdélníky}{Hop trop}

\begin{xverse}{R. }
/: Kulatý [D]obdélníky, kulatý obdélníky,
fialovej [A7]les a žlutá [D]voda. :/
\end{xverse}

\begin{xverse}{1. }
[D]Pojď se mnou, ty moje poupě,
já [A7]ukážu ti opiový [D]doupě,
tam v těžkým dýmu omamnejch jedů
uvidíš [A7]fialovej les a žlutou [D]vodu.
\end{xverse}

\begin{xverse}{R. }
Kulatý obdélníky ...
\end{xverse}

\begin{xverse}{2. }
Ležím si na břiše, na zádech bednu kytu,
v kapse hrst hašiše, žiju si v blahobytu,
dva kufry algeny dostal jsem za chatu
a potom za auťák LSD lopatu.
\end{xverse}

\begin{xverse}{R. }
Kulatý obdélníky ...
\end{xverse}

\begin{xverse}{3. }
Fenmetrák posvačím, čuchnu si čikuli,
mám z toho čistidla frňák jak bambuli,
konečně v kómatu rysy mi přituhly,
sako a kravatu dají mi do truhly.
\end{xverse}

\begin{xverse}{R. }
Kulatý obdélníky ...
\end{xverse}

\end{song}

\begin{song}{Lodníkův lament}{Hop trop}

\begin{xverse}{1. }
[Emi]Já snad [G]hned, když jsem se [D]narodil,
na [G]bludnej [D]kámen [G]{šláp'},
a do školy moc [D]nechodil, i [Emi]tak je [D]ze mě [Emi]chlap,
velký [G]dusno, který [D]nad hlavou mi [G]doma [D]vise[G]lo,
drsnýmu chlapu [D]nesvědčí,
já [Emi]{ťuk'} si [D]na če[Emi]lo.
\end{xverse}


\begin{xverse}{R. }
[D|]{} [G|]{} [D|]{} [G|]{} [C|]{} [G|]{}
Má[G]ma mě doma držela a [D]táta na mě dřel,
já moh' jsem jít hned študovat, kdy[G]bych jen trochu chtěl,
voženit se, vzít si ňákou [D]trajdu copatou
a za její lásku platit [G]celou vejplatou, hó [Emi]hou.
\end{xverse}


\begin{xverse}{2. }
Potom do knajpy jsem zašel a tam uslyšel ten žvást,
že na lodích je veselo a fasujou tam chlast,
a tak honem jsem se nalodil na starej vratkej křáp,
kde kapitán byl kořala a řval na nás jak dráb.
\end{xverse}


\begin{xverse}{3. }
Vlny s kocábkou si házely a každej dostal strach
a my lodníci se vsázeli, kdo přežije ten krach,
všechny krysy z lodi zmizely a v dálce maják zhas'
a první byl hned kapitán, kdo měl korkovej pás.
\end{xverse}


\begin{xverse}{4. }
Kolem zubama už cvakali žraloci hladoví,
moc nikomu se nechtělo do vody ledový,
k ránu bouře trochu ustala, já mořskou nemoc měl,
všem, co můžou chodit po zemi, jsem tolik záviděl.
\end{xverse}

\begin{xverse}{5. }
Jako zázrakem jsme dojeli, byl každý živ a zdráv
a všichni byli veselí, jen já jsem rukou máv',
na loď nikdy víc už nevlezu, to nesmí nikdo chtít,
teď lituju a vzpomínám, jak jen jsem se moh' mít.
\end{xverse}

\end{song}

\begin{song}{Louisiana}{Hop trop}

\begin{xverse}{1. }
Ten, [Emi]kdo by jednou chtěl bejt vopravdickej chlap
a na [G]{šífu} křížit [D]svět ho nele[Emi]ká,
teď příležitost má a stačí, aby se jí drap',
ať [G]na tu chvíli [D]dlouho neče[Emi]ká.
\end{xverse}


\begin{xverse}{R. }
Louisi[G_]{a}[D]{na}, Louisi[G_]a[D]na [G]zná už [D]dálky modra[A]vý, [Emi|]{}
[G]bí[D]lá Louisi[G_]a[D]na, jako [G]víra pevná [D]loď,
podepiš a s náma [Emi]pojď, taky hned si z bečky [Hmi]nahni na zdra[Emi]ví.
\end{xverse}


\begin{xverse}{2. }
Jó, tady každej z nás má ruku k ruce blíž,
když to musí bejt, i do vohně ji dá,
proti nám je pracháč i kostelní myš,
nám stačí dejchat volně akorát.
\end{xverse}


\begin{xverse}{R. }
Louisiana, Louisiana ...
\end{xverse}


\begin{xverse}{3. }
Až budem někde dál, kde není vidět zem,
dvě hnáty křížem vzhůru vyletí,
zas bude Černej Jack smát se nad mořem,
co je hrobem jeho obětí.
\end{xverse}

\begin{xverse}{R. }
Louisiana, Louisiana ...
\end{xverse}

\end{song}

\begin{song}{Malý velký muž}{Pacifik}


\begin{xverse}{1. }
Dokud [F#mi]tráva bude růst
Řeky potečou a [D]stoupat bude dým
Léta utečou a [E]kam padne tvůj stín
Země tvá bude [F#mi]tvou

Dokud [Hmi]noci střídá den
vítr bude vát a [A]mraky poplujou
Slunce bude plát a [C#]tak jak léta jdou
Země tvá bude [F#mi]tvou
\end{xverse}


\begin{xverse}{R. }
Jen [A]malý velký muž
tolik dobře věděl [F#mi]co je vostrej nůž
smutek prázdnych sedel
[D]malý velký muž čekal svý zname[E]ní
Jen [A]malý velký muž
žehnal ohni sílu, [F#mi]z rudých kamenů
vítal dýmku míru, [D]přesto pohřbil sen
velký [E]sen u Wounded [F#mi]Knee
\end{xverse}


\begin{xverse}{2. }
Dokud tráva bude růst
ruce špinavý až v plání vztyčí kříž
řeky zastaví se, plakat uslyšíš
Slunce zář krvavou

Dokud noci střídá den
slova neplatí, a co je vlastně jen
ono prokletí, co padlo na tvou zem
na tvou zem ztracenou
\end{xverse}


\begin{xverse}{R. }
Jen malý ...
\end{xverse}

\begin{xverse}{3. }
Dokud tráva bude růst
rány nezhojí a neopláchne déšť
řeky nespojí se v jeden silný proud
silný proud nadějí ..

Dokud noci střídá den
srdce zlomená, a jejich dávný sen
skalní ozvěna už nevrátí tvou zem
tu tvou zem ztracenou
\end{xverse}

\begin{xverse}{R. }
Jen malý ...
\end{xverse}
\end{song}

\begin{song}{Mississippi blues}{Pacifik}


\begin{xverse}{1. }
[Ami]{Ří}kali mu Charlie a [Dmi]jako každej kluk
[Ami]kalhoty si [G]o plot potr[Ami]hal,
říkali mu Charlie a [Dmi]byl to Toma vnuk,
[Ami]na plácku rád [G]košíkovou [Ami]hrál,
[C]křídou kreslil po ohradách [F]plány dětskejch snů,
[Dmi]až mu jednou ze tmy řekli: [E]konec je tvejch dnů,
[Ami]někdo střelil zezadu a [Dmi]vrub do pažby vryl,
nikdo [Ami]neplakal a [G]nikdo nepro[Ami]sil.
\end{xverse}


\begin{xverse}{R. }
Missis[C]sippi, Missis[Ami]sippi, [F]{čer}ný tělo [G]nese říční [C]proud,
Mississippi, Missis[Ami]sippi, [F]po ní bude [G]jeho duše [C]plout. [Ami|]{}
\end{xverse}


\begin{xverse}{2. }
Říkali mu Charlie a jako každej kluk
na trubku chtěl ve smokingu hrát,
v kapse nosil kudlu a knoflíkovej pluk,
uměl se i policajtům smát,
odmalička dobře věděl, kam se nesmí jít,
který věci jinejm patří a co sám může mít,
že si do něj někdo střelí jak do hejna hus,
netušil, a teď mu řeka zpívá blues.

\end{xverse}

\begin{xverse}{R. }
Mississippi, Mississippi ...
\end{xverse}


\begin{xverse}{3. }
Chlapec jménem Charlie, a jemu patří blues,
ve kterým mu táta sbohem dal,
chlapec jménem Charlie snad ušel cesty kus,
jako slepý na kolejích stál,
nepochopí jeho oči, jak se může stát,
jeden že má ležet v blátě, druhej klidně spát,
jeho blues se naposledy řekou rozletí,
kdo vyléčí rány, smaže prokletí.
\end{xverse}

\begin{xverse}{R. }
Mississippi, Mississippi ...
\end{xverse}

\end{song}

% \begin{song}{Mladičká básnířka}{Jarek Nohavica}
%
% \begin{xverse}{1. }
% [G]Mladičká básnířka s [Hmi]korálky nad kotníky [Emi|]{} [D|]{}
% [G]bouchala na dvířka [Hmi]paláce poetiky,  [Emi|]{} [D|]{}
% s někým se [G]vyspala, někomu [Hmi]nedala,láska jako [Emi]hobby,
% [Cmi]pak o tom napsala [D]blues na čtyři [G]doby. [Hmi|]{} [Emi|]{} [D|]{}
% \end{xverse}
%
% \begin{xverse}{2. }
% Své srdce skloňovala podle vzoru Ferlinghetti,
% ve vzduchu nechávala viset vždy jen půlku věty,
% plná tragiky, plná mystiky, plná spleenu,
% pak jí to otiskli v jednom magazínu, ho ho hó.
% \end{xverse}
%
% \begin{xverse}{3. }
% Bývala viděna v malém baru u rozhlasu,
% od sebe kolena a cizí ruka kolem pasu,
% trochu se napila, trochu se opila na účet redaktora
% a týden nato byla hvězdou Mikrofóra, ho ho hó.
% \end{xverse}
%
% \begin{xverse}{4. }
% Pod paží nosila rozepsané rukopisy,
% ráno se budila vedle záchodové mísy,
% životem potřísněná, můzou políbená, plná zázraků
% a pak ji vyhodili z gymplu a hned nato i z baráku, ho ho hó.
% \end{xverse}
%
% \begin{xverse}{5. }
% Ve třetím měsíci dostala chuť na jahody,
% ale básníci-tatíci nepomýšlej' na rozvody,
% cítila u srdce, jak po ní přešla železná bota,
% tak o tom napsala sonet, a ten byl ze života.
% \end{xverse}
%
% \end{song}

\begin{song}{Mlýny}{Spirituál kvintet}

\begin{xverse}{R. }
[G]Slyším mlýnský kámen, jak se otáčí,
[C]slyším mlýnský kámen, jak se otá[G]{čí},
já slyším mlýnský kámen, [H7]jak se otá[Emi]{čí},
[C_]o[D]tá[G]{čí}, otá[D]{čí}, otá[C]{čí}.
\end{xverse}

\begin{xverse}{1. }
Ty mlýny [G]melou celou [C]noc a melou [G]celý den,
melou [C7]bez výhod a melou [G]stejně všem,
melou doleva [C]jen a melou [G]doprava,
melou [A]pravdu i lež, když zrovna [D]vyhrává,
melou [G]otro[C]káře, melou [G]otroky,
melou [C]na minuty, na hodiny, [G]na roky,
melou [H7]pomalu a jistě, ale [Emi]melou [C]včas,
já už [G]slyším [D7]jejich [G]hlas.
\end{xverse}

\begin{xverse}{R. }
Slyším mlýnský kámen ...
\end{xverse}

\begin{xverse}{2. }
Ó, já, chtěl bych aspoň na chvíli být mlynářem,
pane, já bych mlel, až by se chvěla zem,
to mi věřte, uměl bych dobře mlít,
já bych věděl komu ubrat, komu přitlačit,
ty mlýny čekají někde za námi, až zdola zazní naše volání,
až zazní jeden lidský hlas: no tak už melte, je čas!
\end{xverse}

\begin{xverse}{R. }
Slyším mlýnský kámen ...
\end{xverse}

\end{song}

\begin{song}{Mrtvej vlak}{Hoboes}

\begin{xverse}{1. }
[Ami]Znáš tu trať, co jezdit po ní [Dmi]je tak zrovna k zbláznění[Ami],
v semaforu místo lampy [Dmi]svítěj' kosti zkříže[E7]ný,
[Dmi]pták tam zpívat zapomněl a [F]vítr jenom v drátech [E7]zní,
[Ami]jednou za rok touhle tratí [Dmi]zaduní vlak pohřební[Ami].
\end{xverse}

\begin{xverse}{2. }
Po koleji rezavý, tam, kde jsou mosty zřícený,
bez páry a bez píšťaly, kotle dávno studený,
nikdo lístky neprohlíží, s brzdou je to zrovna tak,
s pavučinou místo kouře jede nocí mrtvej vlak.
\end{xverse}

\begin{xverse}{R. }
Mrtvej [Dmi]vlak, mrtvej [F+]vlak nedr[Ami]{}ží jízdní [Ddim]{}řád,
dálku [Ami]máš přece [Ddim|]{rád,} [E7]nase[Ami]dat,
neměj [Dmi]strach, ve ska[F+]lách zadu[Ami]ní mrtvej [Ddim]vlak,
chceš mít [Ami]klid, máš ho [Ddim]mít, už jede [Ami]vlak.
\end{xverse}

\begin{xverse}{3. }
V životě jsi neměl prachy, zato jsi měl řádnej pech,
kamarádi pochcípali v sakra nízkejch tunelech,
že jsi zůstal sám a že jsi jenom hobo ubohej,
zasloužíš si za to všechno aspoň funus fajnovej.
\end{xverse}

\begin{xverse}{4. }
Jednou vlezeš pod vagón a budeš to mít hotový,
kam jsi tímhle vlakem odjel, nikdo už se nedoví,
slunce tady nevychází, cesty zpátky nevedou,
ďábel veksl přehodí a stáhne šraňky za tebou.
\end{xverse}

\begin{xverse}{R. }
Mrtvej vlak ...
už jede [Ami]vlak, už jede vlak ...
\end{xverse}

\end{song}
% \chords{ \chordDdim \chordFplus }

\begin{song}{Na cestě - On the Road}{Wabi Daněk}


\begin{xverse}{1. }
[G]Kdysi u silnice [D]stával, deku do půl [G]zad,
ať si mával, jak si [D]mával, nechtěli ho [G]brát,
[C]nikdy [G/H]Kerouaca [Ami]nečet' a [E]neznal třetí [Ami]proud,
[C]přesto [G]býval [D]spolu s [G]Deanem [D]každej [G]víkend [D]on the [G]road.
\end{xverse}

\begin{xverse}{2. }
Nikdy neměl ani zdání, jak se hrával bop,
měl jen slinu na toulání a překážel mu strop,
životem na plný pecky a neubírat plyn,
tuhle víru na svý pouti vždycky vzýval Sal i Dean.
\end{xverse}

\begin{xverse}{R. }
[C]Tak mi [G]{}řekni, [C]na co vlastně [G]mám
[C]moudrosti [G]vyčtený z [Ami]knížek,
[C]co je [G]dobrý, [C]na to přijdu [G]sám,
[C]co je [G]{}špatný, za tím [F]křížek udě[D]lám.
\end{xverse}

\begin{xverse}{3. }
Tohle na cestě mi říkal, já ho jednou vzal,
potom zavolal jen ``díky'' a já frčel dál,
od těch dob jsem vždycky hlídal, ať kamkoliv jsem jel,
nestojí-li u patníku se svou vírou Dean a Sal.
\end{xverse}

\begin{xverse}{4. }
Vždycky u silnice stával, vlasy do půl zad,
ať si mával, jak si mával, nechtěli ho brát,
nikdy tuhle knížku nečet' a neznal třetí proud,
přesto býval spolu s Deanem každej víkend on the road,
[D]on the [G]road ...
\end{xverse}

\end{song}
% \chords{ \chordGH }

% \begin{song}{Na kolena}{Ivan Hlas}
%
% \begin{xverse}{1. }
% Táhněte [C]{do} háje, všichni [Ami]pryč,
% chtěl jsem jít [C]do ráje a nemám [Ami]klíč,
% jak si tu [C]můžete takhle [Ami]{žrát,}
% ztratil jsem [F]holku, co ji mám [G]rád.
% \end{xverse}
%
% \begin{xverse}{2. }
% Napravo, nalevo, nebudu mít klid,
% dala mi najevo, že mě nechce mít,
% zbitej a špinavej, tancuju sám,
% váš pohled [Dmi]káravej už dávno [G]znám.
% \end{xverse}
%
%
% \begin{xverse}{R. }
% Pořád jen /: [F]na kolena, na kolena, :/ [C]jé jé jé,
% pořád jen /: [F]na kolena, na kolena, :/ [C]jé jé jé,
% pořád jen /: [F]na kolena, na kolena, :/ [C]je to [Ami]tak,
% a vaše [F]saka vám posere [G]pták.
% \end{xverse}
%
%
% \begin{xverse}{3. }
% Cigáro do koutku si klidně dám,
% tuhletu pochoutku vychutnám sám,
% kašlu vám na bonton, vejmysly chytrejch hlav,
% sere mě Tichej Don a ten váš tupej dav.
% \end{xverse}
%
%
% \begin{xverse}{R. }
% Pořád jen na kolena, na kolena, ...
% a tenhle barák vám posere pták.
% \end{xverse}
%
% \end{song}

\begin{song}{Nebeští jezdci}{Waldemar Matuška}

\begin{xverse}{1. }
Po [Ami]zasmušilé pustině jel [C]starý honec krav,
den [Ami]temný byl a vítr ševe[C]lil ve stéblech trav,
tu [Ami]honák k nebi pohleděl a v hrůze zůstal stát,
když z [F]rozedraných [Ami]oblaků uviděl stádo krav se hnát.
\end{xverse}

\begin{xverse}{R. }
Jipija [C]hej, jipija [Ami]hou,to [F]přízraky [Dmi]táhnou [Ami]tmou.
\end{xverse}

\begin{xverse}{2. }
Ten skot měl nohy z ocele a oči krvavý
a na bocích mu plápolaly cejchy řeřavý.
A oblohou se neslo jeho kopyt dunění
a za ním jeli honáci až k smrti znavení.
\end{xverse}

\begin{xverse}{R. }
Jipija hej ...
\end{xverse}


\begin{xverse}{3. }
Ti muži byli sinavý a kalný měli zrak
a marně stádo stíhali,jak mračno stíhá mrak.
A proudy potu máčely jim cáry košilí
a starý honák uslyšel ten jekot kvílivý.
\end{xverse}

\begin{xverse}{R. }
Jipija hej ...
\end{xverse}

\begin{xverse}{4. }
Tu jeden z jezdců zastavil a pravil: ``Pozor dej'',
svou duši hříchu vyvaruj a ďáblu odpírej,
bys nemusel se po smrti na věky věků štvát
a nekonečnou oblohou to stádo s náma hnát.
\end{xverse}

\begin{xverse}{R. }
Jipija hej ...
\end{xverse}

\end{song}

\begin{song}{Nehrálo se o ceny}{Hop trop}

\begin{xverse}{1. }
[F#mi]Měli jsme bundy zele[C#mi]ný,
[D]někomu občas lezly [A]krkem,
[Hmi]kdekdo si o nás myslel [E7]svý,
[Hmi]jako by nikdy nebyl [E7]klukem.
\end{xverse}

\begin{xverse}{2. }
Vod lidí pohled kyselej
a kam jet, to nám bylo volný,
každej už hrozně dospělej,
i když to věkem bylo sporný.
\end{xverse}

\begin{xverse}{R. }
Když [D]na nádraží při pátku
nám čekání se kdysi zdálo [A]dlouhý,
víc [D]než milión v prasátku
bylo nabídnutí cigarety [A]pouhý,
\end{xverse}

\begin{xverse}{}
tam [Hmi]vo zábradlí vopřený, dvě kytary
a syrovej sbor [E7]hlasů,
tam [Hmi]nehrálo se o ceny,
ale pro radost a ukrácení [E7]{ča}su.
\end{xverse}

\begin{xverse}{3. }
Jméno si každej vysloužil
a bral ho stejně jako pravý,
vždyť na tom, jakej kdo z nás byl,
stálo, jak bude přiléhavý.
\end{xverse}

\begin{xverse}{R. }
Když na nádraží při pátku ...
\end{xverse}

\begin{xverse}{4. }
Přesto, že každej jinam šel
životem úspěchů i pádů,
/: těžko by asi zapomněl
na partu dobrejch kama[(A)]rádů. :/
\end{xverse}

\end{song}

\begin{song}{Nejdelší vlak}{Spirituál Kvintet}

\begin{xverse}{1. }
[C]Proud řeky ví, kdy kámen pohla[C7]dí,
[F]ví, kdy ho zastaví [C]hráz, [C7|]{}
[F]ví, voda ví, kdy ji [C]noc [C/H]ochla[Ami]dí,
a [C]zná, jak [G]pálí [C]mráz.
\end{xverse}


\begin{xverse}{2. }
Plout s vlnou výš a znát, kde je břeh tvůj,
slůvka tři prostá ti říct,
tam někde v dálce je návěstí ``stůj!'',
ráda tě mám, nic víc.
\end{xverse}


\begin{xverse}{R. }
[C]Ví[G]tr [Emi]stín [F]tvůj [G]svál,
[F]nejdelší vlak jel [Cmaj7]dál,
[F]{šest}náct vagónů [C]měl, [C/H]tenkrát [Ami]měl,
v [Dmi]posledním z [G]nich jsi [C]stál.
\end{xverse}


\begin{xverse}{3. }
Kouř zprávu hlásí: nejdelší vlak tvůj
vrátí se, půjdu mu vstříc,
blízko mých očí je návěstí ``stůj!'',
ráda tě mám, nic víc.
\end{xverse}


\begin{xverse}{R. }
Vítr stín tvůj svál ...
\end{xverse}

\end{song}
\chords{\chordCmajSeven \chordCH}

\begin{song}{Ohradník}{Hop trop}

\begin{xverse}{1. }
Už [Dmi]sníh se ztrácí [G]ze strání a [Bb]zem začíná [F]{žít,}
jenom [Dmi]blejsklo slunce [G]do louží, už [Bb]parťák na nás [F]vlít',
toho flákání prej [Bb]po farmě má [F]právě ako[Bb]rát,
proto: ``[Dmi]Chlapi, skočte [F]do vozů a [G]natáhneme [Dmi]drát.''
\end{xverse}


\begin{xverse}{2. }
Ty auťáky maj shnilej rám a rozrachtanej plech,
dva džípy z války poslední jen stěží chytaj' dech,
kolikrát mi ten můj nechtěl jet, kolikrát bych do něj kop',
ale ohradníky stavím rád, je to náš jarní džob.
\end{xverse}


\begin{xverse}{R. }
[Dmi]Stovky ran [Bb]palicí a [F]kůly budou [C]stát
[Bb]pro míle [F]dlouhý vede[C]ní,
[Bb]dvě stopy [F]nad zemí pak [C]izolátor [G]dát,
[Dmi]stáda nám [F]ohlídají [Ami]dráty mědě[Dmi]ný.
\end{xverse}


\begin{xverse}{3. }
Až pak za struny drátěný ten ohradníku drát
bolavý ruce vymění a večer začnou hrát,
budeme si všichni zpívat a spánek okradem,
ale ráno, až se rozední, tak zase dál se hnem.
\end{xverse}

\begin{xverse}{R. }
Stovky ran palicí a kůly budou stát ...
\end{xverse}

\end{song}

\begin{song}{Oregon / Touha žít}{Pacifik}

\begin{xverse}{1. }
Kdo [Emi]vyhnal tě tam na cestu dalekou --
touha [G]{}žít, touha [Emi]{}žít,
kdo zboural ti dům a pravdu staletou --
touha [G]{}žít, touha [Emi]{}žít,
[Ami]těžko se ti dejchá v [Emi]těsným ovzduší,
[Ami]{}že máš hlad a žízeň, to [H7]nikdo netuší,
kdo [Emi]přes pláně hnal tvůj osamělej vůz --
touha [G]{}žít, touha [Emi]{}žít.
\end{xverse}

\begin{xverse}{R. }
Ore[C]gon, Ore[D]gon, slyšíš, jak v [G]dálce bije [Emi]zvon,
Ore[C]gon, Ore[D]gon, slyšíš ho [Emi]znít.
\end{xverse}

\begin{xverse}{2. }
Kdo pár cajků tvých pod plachty naložil --
 touha žít, touha žít,
kdo studenou zbraň ti k líci přiložil --
 touha žít, touha žít,
na týhletý cestě jen dvě možnosti máš:
buďto někde chcípnout, anebo držet stráž,
kdo vysnil ti zem a odvahu ti dal - touha žít, touha žít.
\end{xverse}

\begin{xverse}{R. }
Oregon, Oregon, slyšíš, jak v dálce bije zvon ...
\end{xverse}

\begin{xverse}{3. }
Kdo vést bude pluh, až půdu zakrojí --
 touha žít, touha žít,
kdo zavolá den a úly vyrojí --
 touha žít, touha žít,
člověk se drápe až někam k nebi blíž,
dostává rány, a přesto leze výš,
hledej svůj sen, ať sílu neztratí touha žít, touha žít.
\end{xverse}

\begin{xverse}{R. }
Oregon, Oregon, slyšíš, jak v dálce bije zvon ...
\end{xverse}

\begin{xverse}{R. }
Oregon, Oregon, nesmíš tu stát jak uschlej strom,
Oregon, Oregon, dál musíš jít.
\end{xverse}

\end{song}

\begin{song}{Outsider waltz}{Wabi Daněk}

\begin{xverse}{1. }
Dnes [G]ráno, když bylo půl, při [Hmi]pravidelný hygieně
[Ami]poklesls' hodně v ceně, když jsi [C]zahlíd' svůj [D]zjev,
už [Ami]nejsi, co jsi [D]býval, tu [G]tvář nespraví ti [Emi]masáž,
[Ami]marně se, hochu, [D]kasáš, už nejsi [G]lev a velkej [D]{}šéf.
\end{xverse}

\begin{xverse}{R. }
[G]Máš svůj svět a [Emi]ten se ti hroutí,
to [G]dávno znám, já [E]prožil to sám,
[Ami]kráčíš [D]dál a [Ami]cesta se [D]kroutí,
až [Ami]potkáš nás [Hmi]na ní, tak   [D]přidej se k [G]nám.
\end{xverse}

\begin{xverse}{2. }
Jsi z vojny doma čtrnáct dnů, a na radnici velká sláva,
to se ti holka vdává, cos' jí dva roky psal,
ulicí tiše krouží ten blbej motiv z Lohengrina,
není ta - bude jiná, dopisy spal a jde se dál.
\end{xverse}

\begin{xverse}{R. }
Máš svůj svět a ten se ti hroutí ...
\end{xverse}

\begin{xverse}{3. }
Za sebou máš třicet let a zejtra ráno třetí stání
a nemáš ani zdání, jak to potáhneš dál,
ten, komus' kdysi hrával, se znenadání někam ztratil,
už nemáš, čím bys platil, no tak se sbal a šlapej dál.
\end{xverse}

\begin{xverse}{R. }
Máš svůj svět a ten se ti hroutí ...
\end{xverse}

\end{song}

\begin{song}{Píseň, co mě učil listopad}{Wabi Daněk}

\begin{xverse}{1. }
[G]Málo jím a [C]málo spím a [G]málokdy tě [C]vídám,
[G]málokdy si [Hmi]nechám něco [Ami]zdát, [D7]{}
[C]doma nemám [G]stání [G/F#|]{už} [Emi]od jarního [C]tání,
[F]cítím, že se blíží listo[G]pad.
\end{xverse}

\begin{xverse}{R. }
Listopado[F]vý písně [C]od léta už [G]slýchám,
vítr ledo[Ami|]{vý} [C]přinesl je k [G]nám,
tak mě neče[F]kej, dneska [C]nikam nepos[G]píchám,
listopado[Ami|]{vý} [C]písni naslou[G]chám.
\end{xverse}

\begin{xverse}{2. }
Chvíli stát a poslouchat, jak vítr větve čistí,
k zemi padá zlatý vodopád,
pod nohama cinká to poztrácené listí,
vím, že právě zpívá listopad.
\end{xverse}

\begin{xverse}{R. }
Listopadový písně ...
\end{xverse}

\begin{xverse}{3. }
Dál a dál tou záplavou, co pod nohou se blýská,
co mě nutí do zpěvu se dát,
tak si chvíli zpívám a potom radši pískám
píseň, co mě učil listopad.
\end{xverse}

\begin{xverse}{R. }
Listopadový písně ...
\end{xverse}

\end{song}

% \chords{ \chordGFis }
%
\begin{song}{Pískající cikán}{Spirituál Kvintet}

\begin{xverse}{1. }
[G]Dívka [Ami]loudá se [G]vini[Ami|]{cí,} [G]tam, kde [Ami]zídka je [Hmi_]{níz}[Ami]{ká},
[G]tam, kde [Ami]stráň končí [Hmi]voní[C]cí, si [G]písnič[C]ku někdo [GC_]{pí}[D]{ská}.
\end{xverse}

\begin{xverse}{2. }
Ohlédne se a ``propána!'', v stínu, kde stojí líska,
švarného vidí cikána, jak leží, písničku píská.
\end{xverse}


\begin{xverse}{3. }
Chvíli tam stojí potichu, písnička si jí získá,
domů jdou spolu ve smíchu, je slyšet cikán, jak píská.
\end{xverse}


\begin{xverse}{4. }
Jenže tatík, jak vidí cikána, pěstí do stolu tříská,
``ať táhne pryč, vesta odraná, groš nemá, něco ti spíská.''
\end{xverse}


\begin{xverse}{5. }
Teď smutnou dceru má u vrátek, jen Bůh ví, jak se jí stýská,
``kéž vrátí se mi zas nazpátek ten, který v dálce si píská.''
\end{xverse}


\begin{xverse}{6. }
Pár šídel honí se po louce, v trávě rosa se blýská,
cikán, rozmarýn v klobouce, jde dál a písničku píská.
\end{xverse}


\begin{xverse}{7. }
Na závěr zbývá už jenom říct, v čem je ten kousek štístka:
peníze často nejsou nic, má víc, kdo po svém si píská ...
\end{xverse}

\end{song}

\begin{song}{Pochod marodů}{Jarek Nohavica}

\begin{xverse}{1. }
[Ami]Krabička cigaret a [C]do kafe [G]rum, [F]rum, [Ami]rum,
dvě vodky a fernet a teď, [C]doktore, [G]{čum}, [F]{čum}, [Ami]{čum},
chra[Dmi]pot v hrud[F]ním ko[Ami]{ši}, no [Dmi]to je [F]záži[E]tek,
[Ami]my jsme kámoši řidi[C]{čů} sani[G]tek, -[F]tek, -[Ami]tek.
\end{xverse}


\begin{xverse}{2. }
Měli jsme ledviny, ale už jsou nadranc, -dranc, -dranc,
i tělní dutiny už ztratily glanc, glanc, glanc,
u srdce divný zvuk, co je to, nemám šajn,
je to vlastně fuk, žijem fajn, žijem fajn, fajn, fajn.
\end{xverse}


\begin{xverse}{R. }
[Ami]Cirhóza, [C]trombóza, [G]dávivý [C]kašel,
[Dmi]tuberku[Ami]lóza - [E]jó, to je [Ami]naše!
neuróza, [C]skleróza, [G]ohnutá [C]záda,
[Dmi]paraden[Ami]tóza, no [E]to je pa[Ami]ráda!
Jsme [Dmi]slabí na tě[Ami]le, ale [G]silní na du[C]chu,
[Dmi]{ži}jem vese[Ami]le, [E]juchuchuchu[Ami]chu!
\end{xverse}


\begin{xverse}{3. }
Už kolem nás chodí pepka mrtvice, -ce, -ce,
tak pozor, marodi, je zlá velice, -ce, -ce,
zná naše adresy a je to čiperka,
koho chce, najde si, ten natáhne perka, -rka, -rka.
\end{xverse}


\begin{xverse}{4. }
Zítra nás odvezou, bude veselo, -lo, -lo,
doktoři vylezou na naše tělo, -lo, -lo,
budou nám řezati ty naše vnitřnosti
a přitom zpívati ze samé radosti, -sti, -sti.
\end{xverse}


\begin{xverse}{R. }
Zpívati: cirhóza, trombóza, dávivý kašel,
tuberkulóza, hele, já jsem to našel!
Neuróza, skleróza, křivičná záda,
paradentóza, no to je paráda!
Byli slabí na těle, ale silní na duchu,
žili vesele, než měli poruchu.
\end{xverse}

\end{song}

\begin{song}{Pole s bavlnou}{Rangers/Plavci}

\begin{xverse}{1. }
Pane [C]můj, co v nebi je [C7]tvůj dům,
má máma můj [F]{ži}vot dala [C]katům,
katům mým v polích s bavl[G7]nou.
Pane [C]můj, co v nebi je [C7]tvůj dům,
má máma můj [F]{ži}vot dala [C]katům,
katům mým v [G7]polích s bavl[C]nou. [F|]{} [C|]{}
\end{xverse}

\begin{xverse}{R. }
[C7]Den za dnem [F]kůže zná bič katů,
vidíš [C]jen černý záda bratrů,
jak tam dřou v polích s bavl[G7]nou,
to, co [C]znáš ty v Lousi[C7]aně,
černý záda [F]znaj' i v Texa[C]caně,
i tam jsou v [G7]polích s bavl[C]nou. [F|]{} [C|]{}
\end{xverse}

\begin{xverse}{2. }
Já vím, brzy musí přijít soud,
černý záda práva na něm vyhrajou,
Boží soud v polích s bavlnou.
\end{xverse}

\begin{xverse}{R. }
Den za dnem ...
\end{xverse}

\begin{xverse}{3. }
Chtěl bych jít na potem zvlhlý lány,
tak řekni, Pane můj, černý zvoňte hrany
katům mým v polích s bavlnou.
\end{xverse}

\begin{xverse}{R. }
Den za dnem ...
\end{xverse}

\begin{xverse}{4. }
Dnes měj, Pane, co v nebi je tvůj dům,
mou duši, když život patří katům,
katům mým v polích s bavlnou.
\end{xverse}

\begin{xverse}{R. }
Den za dnem ...
\end{xverse}

\end{song}

\begin{song}{Pošťák}{Hop trop}

\begin{xverse}{1. }
[Ami]Psal jsem ti, [G]brácho, a [Ami]na ouřad [G]psaní šel [F]dát,
psaní vo [C]tom, že jsem [Dmi]{čer}nej, že z [Ami]fleku bych [E7]krad',
z [Ami]váčku jsem [G]lovil pár [Ami]centů a [G]{šéf} mi vtom [F]{řek':}
pošťák se [C]nevrátil, [Dmi]jestli bych [Ami]vzal po něm [E7]flek,
[Ami]sáně mi [G]dal, tresky v [Ami]balíku [G]pro psy a [F]kvér,
brašnu, v ní [C]lejstra, a [Dmi]po zádech [Ami]plác' mě: ``Buď [E7]fér!''
\end{xverse}

\begin{xverse}{R. }
[A]Pošťák se má, za [D]známky nepla[A]tí,
[D]hlavně když se s [C]pytlem prachů [E7]někam neztra[A]tí,
pošťák se má, a [D]když se neztra[A]tí,
[D]za pět roků [C]utopený [E7]sáně zapla[A]tí.
\end{xverse}

\begin{xverse}{2. }
Ten rok byl divnej, i slunce si přišlo ňák dřív,
led se měl hnout, a když ne, tak to stal by se div,
místo jsem našel, kde předjíždět Kobuk měl jít,
dál předák nechtěl a já nerad musel psy bít,
zázrakem živej pak dostal se na druhej břeh
bez psů a sání, a proklínal zbytečnej spěch.
\end{xverse}

\begin{xverse}{R. }
Pošťák se má ...
\end{xverse}

\end{song}

% \begin{song}{Pozor, tunel!}{Kamelot}
%
% \begin{xverse}{1. }
% [Ami]Slyšíš rachot kol a k rozednění půlhodina [Fmaj7]schází,
% Sára [Dmi7]protře rosou oči a [Emi]prázdný kapsy obrá[Ami]tí.
% Váš brácha severák vyfouká z vlasů zbytky sazí,
% děti kolejnic a pražců stěží se domů navrátí.
% \end{xverse}
%
% \begin{xverse}{R. }
% Už [G]zpívá telegraf, seš [D]jedno velký [Ami]ucho
% a [G]{ťuká} zprávu [Fmaj7]zpráv: [E7]{}
% Jimmy - Jimmy - Jimmy - Jimmy, pozor, tunel! [Ami]{}
% \end{xverse}
%
% \begin{xverse}{2. }
% Jak dravec letí vlak, možná, že mašinfíra šílí,
% ``Honem chyť mou ruku, holka, než padneš k vekslu na zobák.''
% Ten blázen žene stroj, že se snad zastaví až v Dillí,
% je špatný znamení, když v nebi krouží černej pták.
% \end{xverse}
%
% \begin{xverse}{R. }
% Už zpívá telegraf ...
% \end{xverse}
%
% \end{song}
% \chords{ \chordFmajSeven \chordDmiSeven }

\begin{song}{Rychlé šípy}{Wabi Ryvola}

\begin{xverse}{1. }
[Emi]Můj život je hned plný nesnází,
[A]na jaře když duben přichází,
já [C]vracím se do poválečnejch let,
[Emi]kdy vycházel náš starý dobrý Vpřed,
já [G]{žlu}tý kvítek za klopu si dám
a [Ami]píseň Vontů tiše zabroukám,
do [D]Stínadel se šerem vypravím,
snad [Emi]potkám cestou [H7]Losnu, co já vím.
\end{xverse}


\begin{xverse}{2. }
Dunivá Kateřina burácí
a Široko má dávno po práci,
jen já se vracím Myší pastí sám,
nevím, co s ježkem v kleci dělat mám.
Bohouš,Dlouhé Bidlo, Štětináč,
pan Fišer pustil z okna květináč,
Jan Tleskač, Jiří Rymáň a tak dál,
pan Foglar tohle nikdy nenapsal.
\end{xverse}


\begin{xverse}{3. }
To Rychlé šípy samy byly v nás
a žlutý kvítek symbolem byl krás,
co nemůže nám nikdy nikdo vzít,
kdo kopal studnu, aby druhej moh pít.
Snad jednou až se jaro navrátí,
můj život píseň Vontů obrátí,
já svobodný a čistý půjdu dál
a směšný bude ten. kdo se mi smál.
\end{xverse}


\begin{xverse}{4. }
Tak Mirek Dušín s Červenáčkem jdou
a Jindra Hojer s Jarkou Metelkou,
za nima Rychlonožka s Bublinou,
naší krásnou chlapeckou krajinou.
[G]Duj, [D]duj, [C]duj, [Emi]fujaro vítězná.
\end{xverse}
\end{song}

\begin{song}{Starý příběh}{Spirituál kvintet}

\begin{xverse}{1. }
Řek' [C]Mojžíš jednou [Fmaj7]lidu svému: [C]přišel [Fmaj7]{}čas,
dnes v [C]noci tiše [Emi]vytratí se [F]každý z [G]nás.
[C]Má[E]vá, [F]má[D7]vá nám [C]všem svo[Fmaj7]bodná [C|]{zem.} [Fmaj7|]{} [C|]{}
\end{xverse}


\begin{xverse}{2. }
Já říkám rovnou: každý ať s tím počítá,
že naše cesta ke štěstí je trnitá.
Mává, mává nám všem svobodná zem.
\end{xverse}


\begin{xverse}{R. }
[C]Kdo se bojí vodou jít,
ten podle tónů faraónů musí [G]{}žít.
[C]Má[E]vá, [F]má[D7]vá nám [C]všem svo[Fmaj7]bodná [C]{zem.} [Fmaj7|]{} [C|]{}
\end{xverse}


\begin{xverse}{3. }
Až první krúček bude jednou za námi,
tak nikdo nesmí zaváhat, dát na fámy.
Mává, mává nám všem svobodná zem.
\end{xverse}


\begin{xverse}{4. }
Pak tenhle vandr všem potomkům ukáže,
že šanci má jen ten, kdo má dost kuráže.
Mává, mává nám všem svobodná zem.
\end{xverse}


\begin{xverse}{R. }
Kdo se bojí vodou jít ...
\end{xverse}


\begin{xverse}{5. }
Ten starý příběh z Bible vám tu vykládám,
ať každý ví, že rozhodnout se musí sám.
Mává, mává nám všem svobodná zem.
\end{xverse}


\begin{xverse}{R. }
Kdo se bojí vodou jít ...
\end{xverse}

\end{song}

\begin{song}{Strom}{Ozvěna}

\begin{xverse}{1. }
[Ami]Polní cestou kráčeli šu[G]maři do vísky hrát,
[Ami]svatby, pohřby tahle cesta po[G]znala mnohokrát,
po [F]jedné svatbě se [G]chudým lidem [Ami]synek narodil
a [F]táta mu u [G]prašný cesty [E]{}života strom zasadil.
\end{xverse}

\begin{xverse}{R. }
A on tam [A]stál, a koukal [F#mi]do polí,
byl jak [D]král, sám v celém [E]okolí,
korunu [A]měl, korunu [F#mi]měl, i když ne [D]ze zlata, [Dmi]{}
a jeho [A]pokladem byla [E]tráva střapa[A]tá.
\end{xverse}

\begin{xverse}{2. }
Léta běží a na ten příběh si už nikdo nevzpomněl,
jen košatý strom se u cesty ve větru tiše chvěl,
a z vísky bylo město a to město začlo chtít
asfaltový koberec až na náměstí mít.
\end{xverse}

\begin{xverse}{R. }
A on tam stál ...
\end{xverse}

\begin{xverse}{3. }
Že strom stál v cestě plánované, to malý problém byl,
ostrou pilou se ten problém snadno vyřešil,
tak naposled se do nebe náš strom pak podíval
a tupou ránu do větvoví už snad ani nevnímal.
\end{xverse}

\begin{xverse}{R. }
A on tam stál ...
\end{xverse}


\begin{xverse}{4. }
Při stavbě se ukázalo, že silnice půjde dál,
a tak kousek od nové cesty smutný pařez stál,
dětem a výletníkům z výšky nikdo nemával
a jen přítel vítr si o něm píseň na strništích z nouze hrál.
\end{xverse}


\begin{xverse}{R. }
Jak tam stál ...
\end{xverse}

\end{song}

\begin{song}{Škrábej}{Hop trop}

\begin{xverse}{1. }
[Emi]Trojstěžníku plachty k rozervání napnutý,
[G]třináctej den je to s náma nějaký nahnutý,
[D]smůlu táhnem za kormidlem s sebou po vl[Emi]nách,
my lodníci jsme na tom nejhůř, vím to na tuty,
[G]pískovcovou cihlu v ruce, záda vohnutý,
[D]bocman vříská, nejradši bych po krku mu [Emi]sáh'.
\end{xverse}


\begin{xverse}{R. }
Hej, [Emi]hej! Škrábej ty prkna, ať jsou [C]bílý!
Hej, [Emi]hej! Škrábej, ty prkna musej' [C]bejt!
Hej, [Emi]hej! Říkej si klidně každou [C]chvíli:
[Emi]nebudem [D]spílat, [Emi]ruce [D]spínat, [Emi]{}žalmy [D]zpívat, [Emi]hou!
\end{xverse}


\begin{xverse}{2. }
Bez vody jsme všichni skoro žízní leknutý,
nikomu z nás nevadí, že spíme vobutý,
stejně každej den jeden z nás končí na marách,
čert aby vzal bocmana a s ním i drhnutí,
ze kterýho máme teď ty záda vohnutý,
chcem bejt rovný, až do pekla překročíme práh.
\end{xverse}


\begin{xverse}{R. }
Hej hej ...
\end{xverse}

\end{song}

\begin{song}{Šnečí blues}{Jarek Nohavica}

\begin{xverse}{1. }
[G]Jednou [C7]jeden [G]{šnek}  [D/F#|]{} [G]{ší}le[C]ně se [G]lek', [D7|]{}
[G]nikdo už dnes [G7]neví, z [C]{če}ho se tak [Cmi]zjevil,
že se [G]dal [C7]hned [D7]na  ú[G]těk. [D7|]{}
\end{xverse}


\begin{xverse}{2. }
Přes les a mýtinu rychlostí půl metru za hodinu,
z ulity pára, ohnivá čára,
měl cihlu na plynu.
\end{xverse}


\begin{xverse}{3. }
Ale v jedné zatáčce, tam v mechu u svlačce,
udělal šnek chybu, nevyhnul se hřibu,
nevyhnul se bouračce.
\end{xverse}


\begin{xverse}{4. }
Hned seběhl se celý les a dali šneka pod pařez,
tam v tom lesním stínu, jestli nezahynul,
tak leží ještě dnes.
\end{xverse}


\begin{xverse}{5. }
A kdyby použil vůz anebo autobus,
/: nebylo by nutné zpívat tohle smutné,
   smutné šnečí blues. :/
\end{xverse}

\end{song}
\chords{ \chordDFis }

\begin{song}{Tak si tam stůj}{Hop trop}

\begin{xverse}{1. }
Tak si tam [Dmi]stůj, já [C]dál tě klidně [Gmi]na krajnici [Bb]nechám,
tak si tam [Dmi]stůj, už [C]za chvíli se [Gmi]hodně seše[Bb]{}ří,
[Dmi]nezasta[Ami]vím, a [F]nebude to [Gmi]tím, že zrovna [Ami]spěchám,
nezasta[Gmi]vím, když vidím, že mi někdo nevě[Dmi]{}ří.
\end{xverse}

\begin{xverse}{R. }
Polykám [C]dálku, letí mi [Dmi]{}čas,
když neza[Bb]máváš, [A7]tak vem tě [Dmi]{}ďas.
\end{xverse}


\begin{xverse}{2. }
Tak si tam stůj, já napíšu do špíny na kontejner,
tak si tam stůj, já cestou s někým povídat si chtěl,
nezastavil můj udejchanej zablácenej trailer,
nezastavil, tvý pohrdavý oči uviděl.
\end{xverse}


\begin{xverse}{R. }
Polykám dálku, letí mi čas...
\end{xverse}


\begin{xverse}{3. }
Tak si tam stůj, snad naloží tě ňákej lepší auták,
tak si tam stůj, už za chvíli tě večer zastudí,
zastavím tam, kde za pár hodin vystřídá mě parťák,
zastavím tam, kde za zádama se mi probudí.
\end{xverse}

\end{song}

\begin{song}{Toulavej}{Vojta Kiďák Tomáško}

\begin{xverse}{1. }
Někdo [Ami]z vás, kdo chutnal [G]dálku, jeden [Ami]z těch, co rozu[E]měj',
ať vám [Ami]poví, proč mi [G]{ří}kaj', proč mi [F]{ří}kaj' Toula[Ami]vej.
\end{xverse}


\begin{xverse}{2. }
Kdo mě zná a v sále sedí, kdo si myslí: je mu hej,
tomu zpívá pro všední den, tomu zpívá Toulavej.
\end{xverse}


\begin{xverse}{R. }
[F]Sobotní ráno [G]mě neuvidí u [G7]cesty s klukama [C]stát
[F]na půdě celta se [G]prachem stydí [F]a starý songy jsem [G]zapomněl hrát,
zapomněl [Ami]hrát.
\end{xverse}


\begin{xverse}{3. }
Někdy v noci je mi smutno,často bejvám doma zlej,
malá daň za vaše ``umí'',kterou splácí Toulavej.
\end{xverse}


\begin{xverse}{4. }
Každej měsíc jiná štace,čekáš, kam tě uložej',
je to fajn, vždyť přece zpívá,třeba smutně, Toulavej.
\end{xverse}

\begin{xverse}{R. }
Sobotní ráno mě neuvidí ...
\end{xverse}

\begin{xverse}{5. }
Vím, že jednou někdo přijde,tiše pískne: no tak jdem,
známí kluci ruku stisknou,řeknou: vítej, Toulavej.
\end{xverse}


\begin{xverse}{6. }
Budou hvězdy jako tenkrát,až tě v očích zabolej',
celou noc jim bude zpívat jeden blázen - Toulavej.
\end{xverse}


\begin{xverse}{R. }
Sobotní ráno nám poletí vstříc,budeme u cesty stát,
vypráším celtu a můžu vám říct,že na starý songy si vzpomenu rád,
vzpomenu rád.
\end{xverse}

\begin{xverse}{7. }
Někdo [Ami]z vás, kdo chutnal [G]dálku, jeden [Ami]z těch, co rozu[E]měj',
ať vám [Ami]poví, proč mi [G]{ří}kaj', proč mi [F]{ří}kaj' [E]Toula[Ami]vej.
\end{xverse}
\end{song}

\begin{song}{Trh ve Scarborough}{Spirituál kvintet}

\begin{xverse}{1. }
[Emi]Příteli, máš do [D]Scarborough [Emi]jít,
[G]dobře [Emi]vím, že [G]půjdeš [A]tam [Emi]rád,
tam dívku [G]najdi na [F#mi]Mar[Emi]ket [D]Street,
[Emi]co chtěla [A]dřív [C]mou [D]{že}[Emi]nou [D]se [Emi]stát.
\end{xverse}

\begin{xverse}{2. }
Vzkaž ji, ať šátek začne mi šít,
za jehlu niť však smí jenom brát
a místo příze měsíční svit,
bude-li chtít mou ženou se stát.
\end{xverse}

\begin{xverse}{3. }
Až přijde máj a zavoní zem,
šátek v písku přikaž ji prát
a ždímat v kvítí jabloňovém,
bude-li chtít mou ženou se stát.
\end{xverse}

\begin{xverse}{4. }
Z vrkočů svých ať uplete člun,
v něm se může na cestu dát,
s tím šátkem ať vejde v můj dům,
bude-li chtít mou ženou se stát.
\end{xverse}

\begin{xverse}{5. }
Kde útes ční za přívaly vln,
zorej dva sáhy pro růží sad,
za pluh ať slouží šípkový trn,
budeš-li chtít mým mužem se stát.
\end{xverse}

\begin{xverse}{6. }
Osej ten sad a slzou ho skrop,
choď těm růžím na loutnu hrát,
až začnou kvést, tak srpu se chop,
budeš-li chtít mým mužem se stát.
\end{xverse}

\begin{xverse}{7. }
Z trní si lůžko zhotovit dej,
druhé z růží pro mě nech stlát,
jen pýchy své a Boha se ptej,
proč nechci víc tvou ženou se stát.
\end{xverse}

\end{song}

\begin{song}{Tři bratři}{Spirituál kvintet}

\begin{xverse}{1. }
[Ami]Tři bratři žili kdys v zemi skotské,
v domě zchudlém jim [D]souzeno [Ami]{žít}, [E]{}
ti [Ami]kostkama metali, kdo z nich má jíti,
[D]kdo z nich má [Ami|]{jít,} [Emi]{}
[F]kdo z nich má [C]na moři [G]pirátem [Ami]být.
\end{xverse}


\begin{xverse}{2. }
Los padl a Henry už opouští dům,
ač je nejmladší z nich, vybrán byl,
by koráby přepadal, na moři žil,
na moři žil,
své bratry z nouze tak vysvobodil.
\end{xverse}


\begin{xverse}{3. }
Po dobu tak dlouhou, jak v zimě je noc,
a tak krátkou, jak zimní je den,
už plaví se Henry, když před sebou objeví
loď, pyšnou loď:
``Napněte plachty a kanóny ven!''
\end{xverse}


\begin{xverse}{4. }
Čím kratší byl boj, tím byl bohatší lup,
z vln už ční jenom zvrácený kýl,
teď Henry je boháč, když boháče oloupil,
loď potopil,
své bratry z nouze tak vysvobodil.
\end{xverse}


\begin{xverse}{5. }
Do Anglie staré dnes smutná jde zvěst,
smutnou zprávu dnes dostane král,
ke dnu klesla pyšná loď poklady Henry si
vzal, on si vzal;
střezte se moře, on vládne tam dál!
\end{xverse}
\end{song}

\begin{song}{Tři kříže}{Hop trop}

\begin{xverse}{1. }
[Dmi]Dávám sbohem všem [C]břehům prokla[Ami]tejm,
který v [Dmi]drápech má [Ami]{}ďábel [Dmi]sám,
bílou přídí [C]{}šalupa ``My [Ami]Grave''
míří k [Dmi]{}útesům, [Ami]který [Dmi]znám.
\end{xverse}


\begin{xverse}{R. }
Jen tři [F]kříže z bí[C]lýho kame[Ami]ní
někdo [Dmi]do písku [Ami]posklá[Dmi]dal,
slzy v [F]očích měl a v [C]ruce, [Ami]znavený,
lodní [Dmi]deník, co [Ami]sám do něj [Dmi]psal.
\end{xverse}


\begin{xverse}{2. }
První kříž má pod sebou jen hřích,
samý pití a rvačky jen,
chřestot nožů, při kterým přejde smích,
srdce-kámen a jméno Stan.
\end{xverse}


\begin{xverse}{R. }
Jen tři kříže...
\end{xverse}


\begin{xverse}{3. }
Já, Bob Green, mám tváře zjizvený,
štěkot psa zněl, když jsem se smál,
druhej kříž mám a spím pod zemí,
že jsem falešný karty hrál.
\end{xverse}


\begin{xverse}{R. }
Jen tři kříže...
\end{xverse}


\begin{xverse}{4. }
Třetí kříž snad vyvolá jen vztek,
Fatty Rogers těm dvěma život vzal,
svědomí měl, vedle nich si klek' ...
\end{xverse}

\begin{xverse}{Rec}
Snad se chtěl modlit:
"Vím, trestat je lidský,
ale odpouštět božský,
snad mi tedy Bůh odpustí ..."
\end{xverse}

\begin{xverse}{R. }
Jen tři kříže z bílýho kamení
jsem jim do písku poskládal,
slzy v očích měl a v ruce, znavený,
lodní deník, a v něm, co jsem psal ...
\end{xverse}

\end{song}

% \begin{song}{Tu kytaru jsem koupil kvůli tobě}{Václav Neckář}
%
% \begin{xverse}{*. }
% [E]Jak můžeš být tak [C#7]krutá,
% [F#mi]copak nemáš kouska citu v [H6]těle.
% \end{xverse}
%
%
% \begin{xverse}{1. }
% [E]Tu [E6]kytaru [E]jsem [E6]kou[E]{pil}
% [E6]kvů[E]{li} [E6]to[E]{bě}
% a [E6]dal jsem za [E]ni
% [E6]ce[E]{lej} [E6]tá[E]{tův} [H7]plat,
% ta [E]dávno [H7]ještě [E9]byla ve vý[A]robě
% [Ami]a já už [E]věděl
% [H7]co ti budu [E]hrát. [H7]{}
% \end{xverse}
%
%
% \begin{xverse}{2. }
% To ještě rostla v javorovém lese
% a jenom vítr na to dřevo hrál,
% a já už trnul, jestli někdy snese,
% [Ami]{žár}, který [E]ve mně [H7]denně narů[E]stal.
% \end{xverse}
%
%
% \begin{xverse}{}
% S tou [C#mi]kytarou teď stojím před tvým [G#mi]domem,
% měj [A]soucit aspoň k tomu javo[H]ru, [H7]{}
% jen [E]kvůli tobě [E7]přestal býti [A]stromem, [Ami]{}
% [Ami]tak už nás [E]oba [H7]pozvi naho[E]ru,
% [H7]pozvi naho[E]ru, [D]pozvi [D#]naho[E]ru.
% \end{xverse}
% \end{song}
% \chords{ \chordESix \chordENine }

\begin{song}{Tunel jménem Čas}{Hoboes}

\begin{xverse}{1. }
Těch [E]strašnejch vlaků, [G#mi]co se ženou [E7]kolejí tvejch [A]snů,
těch [Ami]asi už se [E]nezbavíš [F#7]do posledních [H7]dnů,
a [E]hvězdy žhavejch [G#mi]uhlíků ti [E7]nikdy nedaj' [A]spát,
tvá [Ami]dráha míří [E]k tunelu a [Fmaj7/5-]tunel, ten má [E]hlad.
\end{xverse}


\begin{xverse}{2. }
Už kolikrát ses mašinfíry zkusil na to ptát,
kdo nechal roky nejhezčí do vozů nakládat,
proč vlaky, co si každou noc pod voknem laděj' hlas,
spolyká díra kamenná, tunel jménem Čas.
\end{xverse}


\begin{xverse}{3. }
Co všechno vlaky vodvezly, to jenom pán Bůh ví,
tvý starý lásky, mladej hlas a slova bláhový,
a po kolejích zmizela a padla za ní klec,
co bez tebe žít nechtěla a žila nakonec.
\end{xverse}


\begin{xverse}{4. }
A zvonky nočních nádraží a vítr na tratích
a honky-tonky piána a uplakaný smích
a písničky a šťastný míle na tulácký pas
už spolkla díra kamenná, tunel jménem Čas.
\end{xverse}


\begin{xverse}{5. }
Než poslední vlak odjede, a to už bude zlý,
snad ňákej minér šikovnej ten tunel zavalí
a veksl zpátky přehodí v té chvíli akorát,
i kapela se probudí a začne zase hrát.
\end{xverse}


\begin{xverse}{6. }
Vlak v nula-nula-dvacetpět bude ten poslední,
minér svou práci nestačí dřív, než se rozední,
ten konec moh' bejt veselej, jen nemít tenhle kaz,
tu černou díru kamennou, tunel jménem Čas.
\end{xverse}

\end{song}
% \chords{ \chordFmajSevenFiveMinus }

\begin{song}{Už to nenapravím}{Máci}

\begin{xverse}{1. }
V [Ami]devět hodin dvacet pět mě [D]opustilo štěstí,
ten [F]vlak, co jsem jím měl jet, na koleji [E]dávno [E7]nestál,
v [Ami]devět hodin dvacet pět ja[D]ko bych dostal pěstí,
já [F]za hodinu na náměstí měl jsem [E]stát, ale v [E7]jiným městě.
Tvá [A7]zpráva zněla prostě a byla tak krátká,
že [Dmi]stavíš se jen na skok, že nechalas' mi vrátka
[G]zadní otevřená, [E]zadní otevřená,
já [A7]naposled tě viděl, když ti bylo dvacet,
[Dmi]to jsi tenkrát řekla, že se nechceš vracet,
[G]{že} jsi unavená, [E]ze mě unavená.
\end{xverse}

\begin{xverse}{2. }
Já čekala jsem, hlavu jako střep, a zdálo se, že dlouho,
může za to vinný sklep, že člověk často sleví,
já čekala jsem, hlavu jako střep, s podvědomou touhou,
já čekala jsem dobu dlouhou, víc než dost, kolik přesně, nevím.
Pak jedenáctá bila a už to bylo passé,
já dřív jsem měla vědět, že vidět chci tě zase,
láska nerezaví, láska nerezaví,
ten list, co jsem ti psala, byl dozajista hloupý,
byl odměřený moc, na vlídný slovo skoupý,
už to nenapravím, už to nenapravím.
\end{xverse}

\end{song}

\begin{song}{Válka růží}{Spirituál kvintet}

\begin{xverse}{1. }
Už [Dmi]rozplynul se [G]hustý dým, [Dmi]derry down, hej, [A]down-a-down,
nad [Dmi]ztichlým polem [Gmi]válečným, derry [Dmi]down, [A]{}
jen [F]ticho stojí [C]kolko[A]lem a [Dmi]vítěz [Dmi/C|]{plení} [Bb]vlastní [A]zem,
je válka [Dmi]růží, down, [Gmi]derry, derry, [A]derry down-a-[Dmi]down.
\end{xverse}


\begin{xverse}{2. }
Nečekej soucit od rváče, derry down, hej, down-a-down,
kdo zabíjí ten nepláče, derry down,
na těle mrtvé krajiny se mečem píšou dějiny,
je válka růží, down, derry, derry, derry down, a-down.
\end{xverse}


\begin{xverse}{3. }
Dva erby, dvojí korouhev, derry down, hej, down-a-down,
dva rody živí jeden hněv, derry down,
kdo změří, kam se nahnul trůn, zda k Yorkům nebo k Lancastrům,
je válka růží, down, derry, derry, derry down, a-down.
\end{xverse}


\begin{xverse}{4. }
Dva erby, dvojí korouhev, derry down, hej, down-a-down,
však hlína pije jednu krev, derry down,
ať ten či druhý přežije, vždy nejvíc ztratí Anglie,
je válka růží, down, derry, derry, derry down, a-down.
\end{xverse}

\end{song}
\chords{ \chordDmiC }

\begin{song}{Zabili, zabili}{Balada pro banditu}
\begin{xverse}{1. }
[C]Zabili [F]zabili [Dmi]chlapa z Kolo[F]{ča}vy
[C]{ře}kněte [F]hrobaři [Dmi]kde je pocho[F]vaný
\end{xverse}

\begin{xverse}{R. }
Bylo tu [C]není tu havrani [F]na plotu
bylo víno v [C]sudě teď tam voda [F]bude
není [C]není tu
\end{xverse}

\begin{xverse}{2. }
Špatně ho zabili špatně pochovali
vlci ho pojedli ptáci rozklovali
\end{xverse}

\begin{xverse}{R. }
Bylo tu, není tu ...
\end{xverse}

\begin{xverse}{3. }
Vítr ho roznesl po dalekém kraji
havrani pro něho na poli krákají
\end{xverse}

\begin{xverse}{R. }
Bylo tu, není tu ...
\end{xverse}

\begin{xverse}{4. }
Kráká starý havran krákat nepřestane
dokud v Koločavě živý chlap zůstane
\end{xverse}

\end{song}

\begin{song}{Zafúkané}{Fleret}

\begin{xverse}{1. }
[Ami]Větr sněh [A2]zanésl z [Ami]hor do [A2]polí,
[Ami]já idu [C]přes kopce, [G]přes údo[Ami]lí,
[C]idu k tvej [G]dědině zatúla[C]nej,
[F]cestičky [C]sněhem sú [E]zafúkané. [Ami|]{} [Fmaj7|]{} [Ami|]{} [E4sus|]{}
\end{xverse}


\begin{xverse}{R. }
[Ami]Zafúka[C]né, [G]zafúka[C]né
[F]kolem mňa [C]všecko je [Dmi]zafúka[E]né
[Ami]Zafúkané [C|]{}, [G|]{} zafúka[C]né,
[F]kolem mňa [C]fšecko je [E]zafúka[Ami]né
[Emi|]{} [D|]{} [G|]{} [H7|]{} [Emi|]{} [D|]{} [G|]{} [H7|]{} [Emi|]{}
\end{xverse}


\begin{xverse}{2. }
Už vašu chalupu z dálky vidím,
srdce sa ozvalo, bit ho slyším,
snáď enom pár kroků mi zostává,
a budu u tvého okénka stát.
\end{xverse}


\begin{xverse}{R. }
/: Ale zafúkané, zafúkané, okénko k tobě je zafúkané. :/
\end{xverse}


\begin{xverse}{3. }
Od tvého okna sa smutný vracám,
v závějoch zpátky dom cestu hledám,
spadl sněh na srdce zatúlané,
aj na mé stopy - sú zafúkané.
\end{xverse}


\begin{xverse}{R. }
/: Zafúkané, zafúkané, mé stopy k tobě sú zafúkané. :/
\end{xverse}

\end{song}
% \chords{ \chordATwo \chordEFourSus \chordFmajSeven }

\begin{song}{Zachraňte koně}{Kamelot}
\begin{xverse}{1. }
[Emi]Peklo byl ráj, když hořela stáj, [Ami7]příteli,
[C]věř mi, koně [D]pláčou, poví[G]dám, [C|]{} [H7|]{}
[Emi]to byla půlnoc, v tom křik o pomoc, už [Ami7]letěly
[C]hejna kohoutů, [H7]{} a bůhví [Emi]kam.
\end{xverse}

\begin{xverse}{R. }
[G]Zachraňte koně, [Hmi]křičel jsem tisíc[C]krát,
[G]{žil} jsem jen pro ně, [Hmi]bránil je nejvíc[C]krát,
než přišla [Ami]chvíle, kdy hřívy [C]bílé
pročesal [Ami]plamen, spálil na [H7]troud.
\end{xverse}

\begin{xverse}{2. }
Ohrady a stáj, a v plamenech kraj už nedýchal,
já viděl, jak to hříbě umírá,
klisna u něj a smuteční děj se odbývá,
jak tiše pláče, oči přivírá.
\end{xverse}

\begin{xverse}{R. }
Zachraňte koně...
\end{xverse}

\end{song}
\chords{ \chordAmiSeven }

\begin{song}{Ze všech chlapů nejšťastnější chlap\\}{Hoboes}

\begin{xverse}{*. }
To [Dmi]znám, to dobře [Ami]znám, znám, znám,
[Emi]na kolejích nejsem nikdy [Ami]sám.
\end{xverse}

\begin{xverse}{1. }
[Ami]Shejbni hlavu, kamaráde, tunel před námi,
[Dmi]veksle tlučou, píšťaly řvou, zvonce vyzvání,
[E7]v boudě dobrej mašinfíra není žádnej srab,
[Ami]i v tom dešti [Dmi]sazí jsem ten [E7]nejšťastnější [Ami]chlap,
jó, [E]ze všech chlapů [E7]nejšťastnější [Ami]chlap.
\end{xverse}

\begin{xverse}{2. }
Když z komína vod mašiny letí černej dým,
na tom světě jenom jednu věc na tuty vím,
na tom světě širokým věc jednu jistou mám,
na kolejích chudej hobo není nikdy sám,
jó, chudej hobo není nikdy sám.
\end{xverse}

\begin{xverse}{R. }
To [Dmi]znám, to dobře [Ami]znám, znám, znám,
[E7]na kolejích nejsem nikdy [Ami]sám, sám, sám,
to [Dmi]znám, to dobře [Ami]znám, znám, znám,
[E7]na kolejích nejsem nikdy [Ami]sám.
\end{xverse}

\begin{xverse}{3. }
Za zádama Frisco, semafor je zelenej,
vlak to žene do tmy jako bejček splašenej,
radujte se, občánkové, hoboes jedou k vám,
na kolejích chudej hobo není nikdy sám,
jó, chudej hobo není nikdy sám.
\end{xverse}

\begin{xverse}{4. }
Pod zádama uhlí mám a deku děravou,
místo lampy večerní jen hvězdy nad hlavou,
navečer jsem do vagónu zalez' jako krab,
i v tom dešti sazí jsem ten nejšťastnější chlap,
jó, ze všech chlapů nejšťastnější chlap.
\end{xverse}

\begin{xverse}{R. }
To znám...
\end{xverse}

\begin{xverse}{5. }
Viděl jsem ji u pangejtu vedle dráhy stát,
usmála se, zamávala, z vagónu jsem spad',
jářku: hallo! Sklopí voči, udělá to ``klap'',
i v tom dešti sazí jsem ten nejšťastnější chlap,
jó, ze všech chlapů nejšťastnější chlap.
\end{xverse}

\begin{xverse}{6. }
=\ 2.
\end{xverse}

\begin{xverse}{R. }
To znám...
\end{xverse}

\end{song}
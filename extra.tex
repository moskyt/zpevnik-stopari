% Kde jsou...

\begin{song}{Ahoj, slunko}{Jaromír Nohavica}

\begin{xverse}{1.~}
[F]Ahoj, slun[C]ko, [Hdim]tobě to teda dneska [Ami7]sekne,
[Dmi7]včera mi bylo [G7]smutno, ale [C]dneska už je [C/H]zase mírně [Ami]pěkně,
[Dmi7]jó, někdy se [G7]zdá, že něco [C]nejde, ale [C/H]ono to pak [Ami]jde,
[Dmi7]doba, ta není [G7]zlá, to jenom [F]některé vte[G]{ři}ny jsou [C]zlé.
\end{xverse}

\begin{xverse}{2.~}
Ahoj, ahoj, slunko, jsem rád, že jsi ještě,
osuš mi moje vlasy, jsou promáčené předvčerejším deštěm,
no, prostě jsem zmok', naštěstí voda teče vždycky shora dolů,
třicátý druhý rok to takhle přes překážky táhnem spolu.
\end{xverse}

\begin{xverse}{3.~}
Ahoj, ahoj, slunko, jsem rád, že tě vidím,
doma mi bylo smutno, a tak jsem vyšel zase mezi lidi,
srdce potřebuje svý, a kdo bere, měl by taky něco dát,
člověk, ten není zlý, jen prostě některé lidi nemám rád.
\end{xverse}

\begin{xverse}{4.~}
Ahoj, ahoj, slunko, ty máš dneska hezké tváře
a kdyby mělo snad být smutno, tak na to já mám svoje opraváře,
a tak dál mi sviť, a až zalezeš za mraky jako tečka,
vylezeš zase, viď, a tu chvilku, než to bude, tu já přečkám.
\end{xverse}

\begin{xverse}{}
[C]Ahoj, [F]a[G]hoj, slun[C]ko, ahoj, [F]a[G]hoj, slun[C]ko,
ahoj, [F]a[G]hoj, slun[C]ko, [G]a[C]hoj ...
\end{xverse}

\end{song}

\begin{song}{Akordy}{Karel Plíhal}

\begin{xverse}{1.~}
[E]Nejkrásnější akord bude [Amaj7]A-maj,
[E]prstíky se při něm nepo[C]lámaj'.
[A]Pomohl mi k [H7]pěkné holce s [G#mi]absolutním [C#mi]sluchem,
[A]každý večer [C]naplníme [E]balón horkým [Amaj7]vzduchem.
\end{xverse}

\begin{xverse}{2.~}
V stratosféře hrajeme si A-maj,
i když se nám naši známí chlámaj'.
Potom, když jsme samým štěstím opilí až namol,
stačí místo A-maj jenom zahrát třeba A-moll.
\end{xverse}

\begin{xverse}{3.~}
A-moll všechny city rázem schladí,
dopadneme na zem na pozadí.
Sedneme si do trávy a budem koukat vzhůru,
dokud nás čas nenaladí aspoň do A-duru.
\end{xverse}

\begin{xverse}{R:~}
[A]Od A-dur je jenom kousek k [Amaj7]A-maj,
[F#mi]proto všem těm, [F#mi/F]co se v lásce [F#mi/E]zklamaj': [F#mi/D#|]{}
[A]vyždímejte [H7]kapesníky [G#mi]a nebuďte [C#mi]smutní,
[A]každá holka [C]pro někoho [E]má sluch abso[Amaj7]lutní.
\end{xverse}

\begin{xverse}{4.~}
V každém akord zní, aniž to tuší,
zkusme tedy nebýt k sobě hluší.
Celej svět je jeden velkej koncert lidských duší,
jenže jako A-maj nic tak srdce nerozbuší.
\end{xverse}

\begin{xverse}{*.~}
[E]Pro ty, co to A-maj v lásce [Amaj7]nemaj',
[E]moh' bych zkusit zahrát třeba [Cmaj7]C-maj ...
\end{xverse}

\end{song}


\begin{song}{Bahama rum}{Jaromír Nohavica}

\begin{xverse}{1.~}
[D]Celou noc pili jsme [D/C#]Bahama rum a [G/H]pojídali sušené [D]keksy, [C|]{} [A7|]{}
[D]měla jsi na sobě [D/C#]{žluté} Harley tričko a [G/H]bylo ti knap a [D]sexy, [C|]{} [A7|]{}
pak jsme se [Hmi]milovali a z [A]kazeťáku zpívala [D]Barbra,
tak to šlo [G]noc co noc, hm [D]hm, od [C]Mikuláše [A7]do Ván[D]oc. [D/C#|]{} [G/H|]{} [D|]{}
\end{xverse}

\begin{xverse}{2.~}
Pak došel dopis na národní výbor, tvůj soused je patrně fízl,
řeklas' mi:``Chlapče, je mi toho vážně líto,
ale bacha, abys něco neslízl.''
A tak jsem odešel pryč a tvoje vlasy voněly hlínou,
paks' tady nebyla, hm hm, od Vánoc do apríla, la la ...
\end{xverse}

\begin{xverse}{3.~}
V červnu mi přišla domů pohlednice s razítkem Baden Baden,
jó, bylas' vždycky velká krasavice, a teď je s tebou amen,
v Potravinách mi vyprodali Bahama rum,
za nos's mě vodila, hm hm, od apríla do apríla.
\end{xverse}

\end{song}



\begin{song}{Bleděmodře}{Pavel Dobeš}

\begin{xverse}{R:~}
[G]Bleděmodře je [C]{ži}vot nadepsaný [G]sešit,
který ovšem není trha[D]cí,
[G]zpátky dá se [C]těžko něco [G]{ře}šit,
v něm jenom dopředu každý [D]list se obra[G]cí.
\end{xverse}

\begin{xverse}{1.~}
[G]Když člověk končí [C]hru prohra[G]nou,
[A]odstranit chce stránku z toho [D]dne,
[G]do koše ji [H7]hodí [E7]zmačkanou
[C]a na konci mu [D]druhá vypad[G]ne. [D7|]{}
\end{xverse}

\begin{xverse}{R:~}
Bleděmodře...
\end{xverse}

\begin{xverse}{2.~}
Když stránku jednou někdo polije
anebo ji v slzách umáčí,
písmo se na ní rozpije
a život není jinačí.
\end{xverse}

\begin{xverse}{R:~}
Bleděmodře...
\end{xverse}

\end{song}


%
%
% Jaromír Nohavica – Dál se háže kamením
%
% \begin{xverse}{.~}
%     Dmi                                               Fdim
% 1. Ve městě jménem Jeruzalém, v hlavním městě římské kolonie,
%                A7                        Dmi
%    na sklonku velikonoc Pilát si ruce myje,
%                                            Fdim
%    otrhaný Ježíš stojí opodál, dav se mu směje zblízka,
%               A7                         Dmi Gmi Dmi A7 Dmi
%    jak je to dál? Dál se háže kamením a píská ...
%  \end{xverse}
%
%    \begin{xverse}{.~}
%       Dmi
% R1: Kříže se nemění, jen příjmení a jména,
%                            Fdim
%     když letí kamení, jde kolem huby pěna,
%      Gmi                Dmi
%     potom se omluvíme, pomníky postavíme
%     A7               Dmi
%     a housky posvačíme.
%
%     \begin{xverse}{.~}
% 2. Břevnovským klášterem jde dým, kříže se pálí,
%    ve jménu kalicha oltáře rozsekali
%    a dobrý muž schovaný za portál křičí: To ďábel spískal!
%    Jak je to dál? Dál se háže kamením a píská ...
%  \end{xverse}
%
%    \begin{xverse}{.~}
% R2: Kříže se nemění, jen příjmení a jména,
%     když letí kamení, jde kolem huby pěna,
%     potom se omluvíme, pomníky postavíme
%     a hlavy ozdobíme.
%            \end{xverse}
%
%     \begin{xverse}{.~}
% 3. Kostel v Arles se kácí dle žabí perspektivy,
%    bláznivá kráso, Vincent je jurodivý,
%    profesor akademie předvádí lineál: jak radno viděti, je to ryska,
%    jak je to dál? Dál se háže kamením a píská, na na na ...
%  \end{xverse}
%
%    \begin{xverse}{.~}
% R2:
% \end{xverse}
%
% \begin{xverse}{.~}
% 4. Malý muž s kytarou na pódiu otvírá vrátka,
%    Pravda je v proscéniu, krutá a bez pozlátka,
%    a lidé opouštějí sál, zaujmou stanoviska,
%    jak je to dál? Dál se háže kamením a píská, na na na ...
%  \end{xverse}
%
%    \begin{xverse}{.~}
% R2:
% \end{xverse}
%
% \begin{xverse}{5.~}
%     Dmi
% 5. Ve městě jménem ..., to je vlastně jedno,
%                              Fdim
%    na sklonku přítomnosti a historie
%              A7                    Dmi
%    miliarda Pilátů zase si ruce myje ...
%     \end{xverse}
%
%
%
% \end{song}


\begin{song}{Dvakrát}{Vlasta Redl}

\begin{xverse}{1.~}
[A]Dvakrát dvě deci [D]přede mě [E]postavil [A]pán,
musel si přeci [D]všimnout, [E]{že} jsem tu [A]sám,
dvakrát se uklonil, [D]dvakrát [E]poděko[A]val,
skupina spustila ``[D]Andulko [E]{Ša}fářo[A]vá''.
\end{xverse}

\begin{xverse}{2.~}
Ach, to je muzika, až se mi srdce svírá,
ještě že na dvojitý Nelson dvakrát se neumírá,
od stolu vstávají ti, co tu nejsou sami,
tak proč stojíš u mě, dívenko s květinami.
\end{xverse}

\begin{xverse}{3.~}
Holky rty špulí na stébla v citrokolách,
všechny jsou v stejném - copánky, mašle, volán,
všechny si naráz dávají hádat z dlaně,
ta, co je nejblíž, vypadá odhodlaně.
\end{xverse}

\begin{xverse}{R:~}
A tak [D]provlékám [D/C]skrz to hlu[H7]ché,
[E7]velblouda i jehly u[A]chem, [A7|]{}
to jsem [D]netušil, [D/C]{že} je to tak
[H7]náramně jednodu[E7]ché.
\end{xverse}

\begin{xverse}{4.~}
To je ta pravá chvíle pro autogramy,
plaše se sklání a voní fialkami,
raději jděte mi z očí, tajemná Eurydiko,
nějak dnes nesnáším pohled na krev a mlíko.
\end{xverse}

\begin{xverse}{5.~}
Samoto, samoto, laskavá domovino,
syn se ti vrátil, tak ještě dvakrát víno,
ať ještě chvíli je všechno tak hezky sudé,
ať se smím loudat pár kroků za osudem.
\end{xverse}

\begin{xverse}{R:~}
A ona ať oči klopí,
ať se v těch hlubinách neutopím,
kéž bych tak neviděl, neslyšel,
co nikdy, nikdy nepochopím.
\end{xverse}

\begin{xverse}{6.~}
Ještě se třpytí jak slza na kameni,
a už tu není, zmizela bez loučení,
ani ten v černém dvakrát si neví rady,
pán si přál platit, zvlášť nebo dohromady?
\end{xverse}

\end{song}


%
%    Jaromír Nohavica – Heřmánkové štěstí
%
%    \begin{xverse}{1.~}
%        G
%       Madonnu Donatella jsem zahlídl, jak zhášela svíci,
%       heřmánkem zavoněla, já nevěděl, co jí mám říci,
%        C                     G
%       nad horizontem už mi hvězda bliká,
%        C          G           Ami       D7
%       ještě není moje, ještě se mě nedotýká,
%       G                 Ami
%       ucítil jsem vůni heřmánku,
%                 D7                Emi    C        G
%       už mi to začíná, už křičím ze spánku: Martina.
%     \end{xverse}
%
%       \begin{xverse}{2.~}
%     Někdo mě popadl a postavil ke zdi, měsíc se dral oknem do pokoje,
%       počkejte chvíli, teď padají hvězdy a jedna z nich je ta moje,
%       odněkud ze tmy brnká slepý kytarista,
%       polibek letmý, řekou plyne voda čistá,
%       ucítil jsem vůni heřmánku,
%       už mi to začíná, už křičím ze spánku: Martina.
%     \end{xverse}
%
%       \begin{xverse}{.~}
%        D        Emi          C G C G C G
%    R: Láska je hora, vysoká hora,
%         D            Emi                 C G C G C G
%       kdo z nás ji zdolá, kdo z nás ji zdolá,
%        Ami               D7       G
%       láska je hora vysoká, hora vysoká.
%     \end{xverse}
%
%
%       \begin{xverse}{.~}
%        Emi                   G
%    *: Venku je bláto, prší štěstí, cinká zlato,
%             C                       Emi
%       po náměstí s parazolem chodí lidi kolem,
%                             G                       C
%       na nebi ptáci, brečí dítě, tramvajáci táhnou sítě,
%                       Emi
%       ulicemi padají ryby k zemi,
%        Ami                     D7
%       nad domy, nad střechami letí smutně čáp,
%        Ami                   D7
%       mávám mu gerberami, sedá si na okap.
%     \end{xverse}
%
%       \begin{xverse}{.~}
%    3. Herodes v bílém plášti vraždí neviňátko,
%       v kamnech kosti praští a po koláčích sladko,
%       pod okny na mě troubí limuzína:
%       nebreč a pojď, tak zle to nezačíná,
%       ach, smutky, smutky z heřmánku,
%       už mi to začíná, už křičím ze spánku: Martina.
%     \end{xverse}
%
%       \begin{xverse}{.~}
%    4. Ve stráních karlovických lesů červnové slunce zlátne,
%       co našel jsem, to u srdce si nesu, co ztratil, mizí v nenávratně,
%       na bříze na kmeni vyrostla holubinka,
%       navždycky ztracený slyším, jak srdce cinká,
%       ach, lásko, lásko z heřmánku,
%       už mi to začíná, už křičím ze spánku: Martina.
%     \end{xverse}
%
%       \begin{xverse}{.~}
%    R:
%       Madonna Donatella, Madonna Donatella ...
%     \end{xverse}
%
%
%
%



\begin{song}{Já neumím}{Jaromír Nohavica}

\begin{xverse}{1.~}
[G]Já neumím psát velkou [C]poezi[G]i,
[Emi]daktylský [Hmi]hexametr [C]není můj [G]svět,
[Emi]mám jeden starý svetr, v [D]kterém žiji
už moc [G]let. [Ami|]{} [D7|]{}
\end{xverse}

\begin{xverse}{2.~}
Pan Havlíček, než umřel na souchotě,
učil mne velkou českou abecedu,
a tak se poflakuju po životě,
jak dovedu.
\end{xverse}

\begin{xverse}{R:~}
Světem se [G]poflakuju, tužkou ho [Ami]obkreslu[D7]ju,
boty si [G]nepucuju a hřeben [Ami]nemilu[D7]ju,
sám sebe [G]povoluju, sám sebe [Ami]zakazu[D7]ju,
ničeho [G]nelituju, -[Ami]ju, -ju, -[D7]ju ...
\end{xverse}

\begin{xverse}{3.~}
Jsem voják, který nemá ani frčku,
vojín, co běží vždycky v první řadě,
chrání mě jenom úzký kmínek smrčku
v kanonádě.
\end{xverse}

\begin{xverse}{4.~}
S četaři absolventy hraju pokra,
ale když přijde na lámání chleba,
je jejich obuv suchá, moje mokrá,
jak je třeba.
\end{xverse}

\begin{xverse}{R:~}
Světem se poflakuju, tužkou ho obkresluju,
boty si nepucuju a hřeben nemiluju,
sám sebe zakazuju, povoluju, nelituju, -ju, -ju, -ju ...
\end{xverse}

\end{song}





\begin{song}{Ještě mi scházíš}{Jaromír Nohavica}

\begin{xverse}{1.~}
[Ami]Ještě se mi o tobě [Ami/G]zdá, ještě mi nejsi [F]lhostejná,
[Ami]ještě mě budí v noci [Ami/G]takový zvláštní [F]pocit,
[Dmi]ještě si zouvám boty, [F]abych snad neušpinil
[D#dim]náš nový běhoun v [E]síni.
\end{xverse}

\begin{xverse}{R:~}
Ještě mi [Ami]scházíš, ještě jsem [G]nepřivykl,
že nepři[C]cházíš, [C/H]{že} nepři[Ami]jdeš,
že zvonek [F]nezazvoní, dveře se [E]neotevřou,
že prostě [Ami]jinde s [G]jiným [C]jseš,
[E]ještě mi [F]scházíš, [E]ještě stále mi [Ami]scházíš.  [Ami/G|]{} [F|]{} [Ami|]{} [Ami/G|]{} [F|]{}
\end{xverse}

\begin{xverse}{2.~}
Ještě je na zrcadle dech, mlhavá stopa po tvých rtech
a v každém koutě jako čert kulhavý skřítek Adalbert,
ještě tě piju v kávě a snídám v bílé vece,
ještě jsi ve všech věcech.
\end{xverse}

\begin{xverse}{R:~}
Ještě mi scházíš...
\end{xverse}

\end{song}




\begin{song}{Jumbo jet}{Pavel Dobeš/Jarek Nohavica}

\begin{xverse}{1.~}
V [G]parkhotelu rozjížděl se fet,
my museli jsme na letiště jet,
[C]dneska už nám [D]nikde nenale[G]jou.
Byl to přesně ten bar,
kde kluci řežou do kytar
a f[C]urt dokola zpívaj' ``[G]go go [G]go''.
\end{xverse}

\begin{xverse}{R:~}
[Emi]{Ško}da, že holky tu nejsou s [A]námi,
[Ami]no ale lodě plujou - [D]a přístavy zůstávají.
[G]Začínáme noční let nad planetou jménem svět,
teď [C]ke startu se krčí [D]naše Jumbo [G]jet.
\end{xverse}

\begin{xverse}{2.~}
Na sto procent otáček jdou motory,
pak ohně začnou šlehat z každé komory
a zvuk se začne třepat nad plochou.
Noční Praha, neónové reklamy, pak Frankfurt, Londýn pod námi,
a nám zní v uších stále ještě ``go go go''.
\end{xverse}

\begin{xverse}{R:~}
V kabině budík na budíku,
naše lasery měří, jak se vzdouvaj' vlny Atlantiku,
rychlost: patnáct set, výška: deset pět,
   nocí letí naše parta starým dobrým Jumbo jet.
\end{xverse}

\begin{xverse}{3.~}
Pozdravujem velitele vzdušných sil
a taky George na majáku v Orresville,
a vůbec všechny, co jsou stále na nohou.
Zvětšují se žluté pásy ranveje,
dobře víš, jak lidské slovo zahřeje,
když ve sluchátkách slyšíš ``go go go''.
\end{xverse}

\begin{xverse}{R:~}
Škoda, že holky tu nejsou s námi,
no ale lodě plujou - přístavy zůstávají.
Ať barman dělá led, ať zablokuje levý střed,
ta čtyři pravá místa pro naši partu v Jumbo jet,
ta čtyři pravá místa pro naši partu v Jumbo jet,
ta čtyři pravá místa pro naši partu v Jumbo jet.
\end{xverse}

\end{song}


% Když jsi smutná
% Interpret/autor: Karel Plíhal
%               C            Fmaj7       C  Fmaj7
% 1. Když jsi smutná, tak i kapky deště bolí,
%            C          Fmaj7     A7
%    rány se otevřou a naplní se solí,
%              Dmi       G            Emi Ami
%    držím tě za ruku a nemám žádnou záruku,
%        Fmaj7    C           Fmaj7
%    že nezůstanu o žebrácké holi.
% 2. Když jsi smutná, tak mi něco ruce sváže,
%    do mé hlavy mlčky vpochodují stráže,
%    všechny mé nápady hned zahánějí do řady
%    a střílí puškou té nejtěžší ráže.
% 3. Když jsi smutná, tak i sochy v parku brečí,
%    stromy procitnou a mluví lidskou řečí,
%    tiše tě konejší, jak umí stromy vezdejší,
%    [: a náhle jsi víc svoje, nežli něčí ... :/


\begin{song}{Koclířov 1, Svitavy 5}{Pavel Dobeš}

\begin{xverse}{1.~}
[G]U silnice čekám na svůj [D]stop,
[Emi]uprostřed mezi lány [H7]polí,
a [C]od obzoru po obzor je ticho jako [G]hrob
[Ami]tam, kde stará státní kříží cestu třicet[D]pět,
Koclířov jedna, [G]Svitavy pět.
\end{xverse}

\begin{xverse}{2.~}
Pak slunce začlo bzučet, když jsem se k němu otočil,
a rozostřovat se mi před očima,
to žlutý práškovací čmelák vybral si mě nejspíš jako cíl,
čekám už jenom, kdy se z jeho střílny ozve těžkej kulomet,
Koclířov jedna, Svitavy pět.
\end{xverse}

\begin{xverse}{3.~}
Ve snu vidím husté řady aut,
které se po dálnicích ženou,
pak jedem spolu na velbloudech, letíme, a já jsem kosmonaut,
dneska už se přece nebudeme vracet zpět,
Koclířov jedna, Svitavy pět.
\end{xverse}

\begin{xverse}{4.~}
Táhneš mě jak řeka vodu z hor
a jako řeku táhne moře,
možná se jen mezi námi někde cestou zbláznil semafor,
a někdo už zařídí, že auta začnou jezdit, že odbrzdí se svět,
Koclířov jedna, Svitavy pět ...
\end{xverse}
\end{song}



\begin{song}{Kostelíček}{Pavel Dobeš}

\begin{xverse}{1.~}
[G]Na kopci je kostelíček, [C]cesta k němu není složi[G]tá,
[D]na zahradě tolik svíček, kdo je spočí[G]tá.
\end{xverse}

\begin{xverse}{2.~}
Jaro, léto, podzim, zima, nevedou tam cesty tramvají,
když tam dojdu se svým klukem, už v kostelíčku zpívají.
\end{xverse}

\begin{xverse}{R:~}
Co [C]naučil se chodit, tak v devět hodin ráno každou [G]neděli
[D]peřinu mi krade a skáče po mé [G]posteli.
\end{xverse}

\begin{xverse}{R:~}
Týden člověk jezdí a těší se zas na doma,
nad ránem se vrací, a cesta bývá pitomá.
\end{xverse}

\begin{xverse}{3.~}
Po schodech mi vždycky zdrhne, žvatlá přitom ty své nesmysly,
prý v kostelíčku zvoní, tak abysme o nic nepřišli.
\end{xverse}

\begin{xverse}{4.~}
Zkontroluje vlčí máky, které se teď v týdnu rozvily,
hlava se mu třepe v obilí.
\end{xverse}

\begin{xverse}{R:~}
Ztratí se mi v poli, a spěch se ho už absolutně netýká,
když potká Ferdu Mravence anebo Brouka Pytlíka.
\end{xverse}

\begin{xverse}{R:~}
Leze po tý zemi, a vůbec ani neví, že je kulatá,
jó, vona není špatná, nebejt tolik od bláta.
\end{xverse}

\begin{xverse}{5.~}
Do kopce se kradem zadem až ku staré brance dřevěné,
někteří z nás ji přeskočili, někteří jí projdeme.
\end{xverse}

\begin{xverse}{6.~}
Než postaví všechny vázy, to je vždycky dávno poledne,
vítr kytky shodí, a už je nezvedne.
\end{xverse}

\begin{xverse}{R:~}
Nevěšíme hlavy, vždyť vůbec o nic neběží,
oba dva jsme zdrávi a nikdo nám tam neleží.
\end{xverse}

\begin{xverse}{7.~}
Cizí město, cizí jména, lidi vycházejí z kostela,
zbyli jsme tu sami, auta odjela.
\end{xverse}

\begin{xverse}{8.~}
Smeká se to dolů z kopce, křeníme se přitom na sebe,
zase jdeme špatně - opačně než do nebe.
\end{xverse}

\begin{xverse}{R:~}
Vracíme se domů cestou kolem košatého javoru,
jak na něm rostou vrtulky, co létají bez motoru.
\end{xverse}

\begin{xverse}{R:~}
My radujem se z toho, i když třeba tráva žloutne po létě,
my, co jsme jen jednou na světě.
My musíme být šťastní, i když jenom tráva žloutne po létě,
my, co jsme jen jednou na světě.
\end{xverse}

\end{song}



\begin{song}{Lisa z N.Y.C.}{Pavel Dobeš}

\begin{xverse}{1.~}
[E]Novinové stánky, billboardy a kolečkoví [F#mi]bruslaři, [H7|]{}
z [F#mi]Atlantiku přišla noc [H7]a stále je [E]hic,
tisíce liber špaget a uvaření [F#mi]kuchaři [H7|]{}
[F#mi]ze žhavejch zdí mrakodrapů [H7]města a italskejch [E]pizz,
[Hmi]a můj žlutej taxík Broadwayí to stříhá přes [A]ulice a třídy
[Ami]se spoustou aut a neonovejch reklam kolem [E]nás, [F#mi|]{}
[E]dneska mám rande s Lisou v New York [F#mi]City [H7|]{}
[F#mi|]{} [H7]a běží mi [E]{čas}. [F#mi|]{} [H7|]{} [F#mi|]{} [H7|]{} [E|]{}
\end{xverse}

\begin{xverse}{2.~}
Pak se mnou Lisa dělá chodecké závody
od sochy Svobody po Brooklyn Bridge,
naše dvě zničená těla - spojené národy
padají do vody a odplouvaj' pryč,
cítíme štiplavý dým, je situace špatná,
nad námi firemanů team hasí hořící dům,
prosím tě, nechej tam hrát Erica Claptona,
tu du du du du.
\end{xverse}

\begin{xverse}{3.~}
``Zítra mě,'' říkám tamtý, ``rozhodně nevolej'',
vždyť ze všech životů tady mi poslední zbyl,
taky jsem úplně country neboli švorcovej,
rád projdu se vašimi sady sám napodýl,
to možná proto, že chci umřít strachy,
možná abych nebyl tolik pyšný,
poskládám si sako na noc do křoví,
nešumí tu air conditiony
a nepřinesou žrádlo pánovi.
\end{xverse}

\begin{xverse}{4.~}
Novinové stánky, billboardy a kolečkoví bruslaři,
Číňan, co umývá chodník, než otevře krám,
na Bosně jezděj' tanky, střílej' se ogaři,
nejlíp se člověku vrací, když zhruba ví, kam,
pak bágly jedou rentgenem a ty si říkáš:
možná jednou někdy příště,
a pak jde všechno hladce jako poprava,
Boeing stoupá nad letiště
a naposledy křídly zamává.
\end{xverse}

\end{song}


%
%
%
% \begin{song}{Možná, že se mýlím}{Jaromír Nohavica}
%
% \begin{xverse}{.~}
%         C                      Fadd9
% 1. Mám rozestláno na posteli pro hosty,
%             C                   Fadd9
%    zuby si čistím cizím zubním kartáčkem,
%          C                Emi
%    já, snůška zděděných vlastností
%      F          Dmi    G
%    a obyvatel planety Zem,
%              F                     C
%    bláznivé Markétě zpívám druhý hlas,
%               Dmi                F                  C Fadd9 C Fadd9
%    jsem tady na světě na krátký víkend na cestovní pas.
%  \end{xverse}
%
%    \begin{xverse}{.~}
% 2. Pan farář nabízel mi věčný život,
%    říkal: musíš, chlapče, přece v něco věřit,
%    a já si dal na lačno jedno pivo
%    a spatřil anděly, jimž pelichá peří,
%    bláznivé Markétě zpívám druhý hlas,
%    jsem tady na světě, dokud nevyprší můj čas.
%  \end{xverse}
%
%    \begin{xverse}{.~}
%     C            F C             F C
% R: Možná, že se mýlím, možná se mýlím,
%                Emi          F         C
%    snad mi to dojde léty, snad mi to dojde léty,
%                       F C                     F C
%    mám na to jenom chvíli, dojít od startu k cíli,
%               Emi             F                     C                Fadd9 C Fadd9
%    a tak si zpívám, a tak se dívám, a tak si dávám do trumpety.
%  \end{xverse}
%
%    \begin{xverse}{.~}
% 3. Jsou prý věci mezi nebem a zemí,
%    já o nich nevím a možná měl bych vědět,
%    já nikdy nikomu jak Ježíš nohy nemyl,
%    já nechtěl nikdy na trůnu sedět,
%    bláznivé Markétě zpívám druhý hlas,
%    jsem tady na světě, dokud nevyprší můj čas.
%  \end{xverse}
%
%    \begin{xverse}{.~}
% 4. Vy, náčelníci dobrých mravů,
%    líbezní darmopilové a marnojedky,
%    proutkaři pohlaví a aranžéři davů,
%    vy jste mi nebyli na svatbě za svědky,
%    bláznivá Markéta, ta mi svědčila,
%    že kdo vchází do světa, jako bys vypustil motýla.
%  \end{xverse}
%
%    \begin{xverse}{.~}
% R:
% \end{xverse}
%
% 5. V pokoji, kde jsem včera spal,
%    vypnuli topení a topila krása,
%    měli jsme na sobě jen flaušový šál
%    a já jsem křičel, že láska je zásah,
%    bláznivá Markéta ať nám zapěje,
%    že až sejdem ze světa, čáry máry fuk, nic se neděje,
%    nic se neděje, nic se neděje ...
%  \end{xverse}
%
% \end{song}

\begin{song}{Na pranýři}{Pavel Dobeš}

\begin{xverse}{1.~}
V [C]hospodě Na pranýři [G7]dneska vůbec nezapadá [C]slunce
a [F]zavírá se [G7]o hodinu [C]dýl
[F]na počest [G7]traktoristy [C]{Šul}ce, [Ami|]{}
[D7]co se včera znovu naro[G]dil. [G7|]{}
\end{xverse}

\begin{xverse}{2.~}
Sedíme, a je už po půlnoci
a číšník nese další rundu s kofolou,
posloucháme, jaký je to pocit,
když se člověk nechtě stane mrtvolou.
\end{xverse}

\begin{xverse}{3.~}
Je to, jako když vám pole pod traktorem někam uhne
a pluh nemůže brázdu zachytit,
blbý na tom je, že člověk ztuhne
a nemůže to nijak ovlivnit.
\end{xverse}

\begin{xverse}{4.~}
Pak vás dají do společné šatny,
zhasnou světla a zapnou chlazení,
norkový kožich by nebyl špatný,
a něco od Rettigové ke čtení.
\end{xverse}

\begin{xverse}{5.~}
Po městě pendluje černý stejšn,
jednoho dne pro vás přihasí,
neuslyšíte ``congratulation'',
odveze vás v každém počasí.
\end{xverse}

\begin{xverse}{6.~}
V očích máte smrt a na tacháči deset nebo dvacet
a stuha s tím, jak měli vás rádi, příjemně šimrá pod nosem,
když už si ten úřad s kosou na vás zased',
ukryje vás nejlíp černozem.
\end{xverse}

\begin{xverse}{7.~}
Jen černé turistické značky vaši cestu nyní měří
a nepomůže už žádné snažení,
pomůže jen to, v co člověk věřil,
dokud ještě chodil po zemi.
\end{xverse}

\begin{xverse}{8.~}
Všechny nás to pomyšlení zebe,
jedni se bojí a druzí strachují,
my nevěříme v peklo ani nebe
a natož v lidi, co nás sledují.
\end{xverse}

\begin{xverse}{9.~}
Vytvořili jsme si svoje vlastní báje,
my nechcem mezi žádný čerty rohatý,
pojedem do březového háje
a rozptýlit se dáme děvčaty.
\end{xverse}

\begin{xverse}{10.~}
Až to na nás přijde, tak ať muzika hraje,
dechovka ať stojí před vraty,
pojedem do březového háje
a rozptýlit se [G7]dáme děvča[C]ty.
\end{xverse}

\end{song}



% Na půdě
% Interpret/autor: Karel Plíhal
%     G6
% 1. Za svitu rezavý baterky
%    cumláme stoletý hašlerky
%     C       Emi7/H   Ami7     G
%    na půdě baráku, kde bydlí teta,
%       C          Emi7/H   Ami7     D
%    a listujem v zažloutým atlase světa.
%   \end{xverse}
%
% 2. Šimráš mě copama na bradě,
%    když hledáš Ohio v Kanadě,
%    tropické slunce se zlověstně tlemí,
%    když táhneš mě za ruku Ohňovou zemí.
%   \end{xverse}
%
%     G                        C          G
% R: Zatímco tvé oči bloudí v cizokrajné dálce,
%     C          G          C           D
%    nemám vůbec odvahu ti rozechvěle sdělit,
%        C              G         C            D
%    že nejhezčí místo na mapě je otisk tvého palce
%    C            Emi7/H        Ami7      D
%    od tetiných fantastických jitrnic a jelit.
%   \end{xverse}
%
% 3. Nejspíš bys praštila atlasem
%    a nazvala mě přízemním mamlasem,
%    jenže ta voňavá palcová země,
%    [: [: [: tu nikde nenajdeš, ta je jen ve mně ... :/ :/ :/


\begin{song}{Něco o lásce}{Pavel Dobeš}

\begin{xverse}{1A.~}
[C]Za ledovou [F]horou a černými [C]lesy
je stříbrná [F]{ře}ka a za ní [C]kdesi
stojí [F]domek bez ad[C]resy a bez de[Dmi]chu, [G|]{}
[F]bydlí v něm - nechci říkat ``[C]víla'',
ale co [F]na tom, i kdyby [C]byla,
[F]před lidmi se trošku [C]skryla
a [Dmi]víme o ní [G]hlavně z dosle[C]chu.
\end{xverse}

\begin{xverse}{2B.~}
Že lidi [Dmi]rozumné blbnout [G]nutí
a není [C]na ni nej[C/H]menší spoleh[Ami]nutí,
[Dmi]co ji zrovna napad[G]ne, to udě[C]lá:  [C/H|]{} [Ami|]{}
z puberťáků [Dmi]chlapy a z chlapů puber[G]{ťá}ky,
o ženských [C]nemluvím, [C/H]tam to platí [Ami]taky,
a [Dmi]urážlivá je a [G]hořkosladkokyse[C]lá.
\end{xverse}

\begin{xverse}{3C.~}
Genetičtí [Dmi]inženýři [G]lámou její [C]kód, [C/H|]{} [Ami|]{}
po Praze se [Dmi]o nich šíří, [G]{že} jezdí tramva[C]jí, [C/H|]{} [Ami|]{}
strkají [Dmi]hlavy [G]pod vodo[C]vod  [C/H|]{} [Ami|]{}
a pak i [Dmi]oni nakonec [G]podléha[C]jí.
\end{xverse}

\begin{xverse}{4B.~}
A holubicím dál rostou křídla dravců,
družstevním rolníkům touha mořeplavců
a lásce, té potvoře, sebevědomí,
že jednou bude vládnout světem,
tedy i nám a po nás našim dětem,
které na tom budou stejně špatně jako my.
\end{xverse}

\begin{xverse}{5B.~}
Když chlap zmagoří láskou, utíká za ní,
platí i s úroky a napočítá s daní,
u ženských je to přímo námět na horor,
papuče letí pod pohovku,
nákupní tašky padaj' na vozovku,
ať si tramvaj zvoní, ať se zblázní semafor.
\end{xverse}

\begin{xverse}{6A.~}
Až vám ta potvora zastoupí cestu,
sedněte na zadek a seďte jak z trestu,
jen ať si táhne, jak to dělají vandráci,
láska se totiž, i když je prevít,
nikomu dvakrát nemůže zjevit,
láska se totiž, i když je prevít, nevrací.
\end{xverse}

\begin{xverse}{7A.~}
A nesmí vám to nikdy přijít líto,
kupte si auto a cucejte chito,
odreagujte se psychicky,
protože jestli byste na ni měli myslet,
to radši vstaňte a jděte za ní ihned,
utíkejte, než vám zmizí navždycky.
\end{xverse}

\begin{xverse}{8B.~}
Převrhněte stůl, opusťte dům,
fíkusy rozdejte sousedům,
nechte vanu vanou, ať si přeteče,
na světě není větší víra,
pro žádnou z nich se tolik neumírá
ani v žádné jiné zemi na světě.
\end{xverse}

\end{song}


%
% \begin{song}{Nový rok}{Jaromír Nohavica}
%
% \begin{xverse}{1.~}
%    D        F#mi    G       D   A7     D
%    O půlnoci, když láme se rok jako chléb,
%              Edim             Hmi
%    sáhnu si na tepnu, ucítím tep,
%     Emi      A7          D   A Hmi G A
%    to tepe krev slepého hodináře
%    Emi        A7           D          Hmi
%    a je to zvláštní pocit, a je to zvláštní pocit
% Emi7 A7      D                F#mi Emi F#mi A7 D
%    sám sebe vyrvat z kalendá[F#mi]ře. [Emi [F#mi [A7 [D
% \end{xverse}
%
%    \begin{xverse}{2.~}
%  Venku na obloze svá pírka rozprostřel páv,
%    půlnoc je chvíle prvních agenturních zpráv
%    a pevných slibů do nového roku,
%    /: svět jako po narkóze :/
%    vstal a volá do útoku.
%
%    \begin{xverse}{R:~}
%    V [A]okýnku orloje se [Hmi]loutky uklání,
%    [G]hokynář s [D]měšcem, vo[G]jáček se zbra[D]ní,
%    smrt [Emi]kosu nabrou[F#]sí, šašek [A7]cinkne čine[D]ly
%    a [G]lidé přejou [D]si [G]šťastný, [A7]šťastný, [D]šťastný a vese[F#mi]lý. [G|]{} [F#mi|]{} [A|]{} [D|]{}
% \end{xverse}
%
%    \begin{xverse}{3.~}
%  Rozcuchaná holčičko z Bangladéše,
%    už tě nikdo nikdy neučeše,
%    jsi moje nejsmutnější loňská fotka,
%    /: láska je vyprodána :/
%    a já jsem bolest potkal.
%     \end{xverse}
%
%    \begin{xverse}{4.~}
%  Jsem o rok starší a smutnější, než jsem byl,
%    tak málo jsem dosud pochopil,
%    i když jsem první těžkou nemoc přestál,
%    /: za mým vozem se práší :/
%    a vpředu neznámá je cesta.
%    \end{xverse}
%
%    \begin{xverse}{R:~}
%  V okýnku orloje...
%    \end{xverse}
%
%   \end{song}

%
%
% Jaromír Nohavica/Boris Vian – Pánové nahoře
%
%       C        Emi    Dmi7           G7
% 1. Pánové nahoře, já píšu vám dnes psaní
%       Dmi7         C       Dmi7       G7
%    a nevím vlastně ani, budete-li ho číst,
%         C            Emi   Dmi7        G7
%    přišlo mi ve středu do války předvolání,
%        Dmi7       C          Dmi7  G7      C
%    je to bez odvolání, tím prý si mám být jist.
%       F        Cdim                  Emi
%    Pánové nahoře, já už to lejstro spálil,
%         Ami           Dmi7                     G7
%    už jsem si kufry sbalil, správcové vrátil klíč,
%       C        Emi   Dmi7         G7
%    pánové nahoře, uctivě se vám klaním
%       Dmi7         C        Dmi7   G7     C Emi Dmi7 G7
%    a zítra vlakem ranním odjíždím někam pryč.
%
% 2. Co už jsem na světě, viděl jsem zoufat matky
%    nad syny, kteří zpátky se nikdy nevrátí,
%    slyšel jsem dětský pláč a viděl jejich slzy,
%    které snadno a brzy se z očí neztratí.
%    Znám vaše věznice, znám vaše kriminály,
%    i ty, kterým jste vzali život či minulost,
%    vím také, že máte solidní arzenály,
%    i když jste povídali o míru víc než dost.
%
% 3. Pánové nahoře, říká se, že jste velcí
%    a na věci to přece vůbec nic nemění,
%    pánové nahoře, na to jste vážně malí,
%    abyste vydávali rozkazy k vraždění.
%    Musí-li války být, běžte si válčit sami
%    a vaši věrní s vámi, mě, mě nechte být,
%    jestli mě najdete, můžete klidní být,
%                                           C Emi Dmi7
%    střílejte, neváhejte, já zbraň nebudu mít,
%     G7     C Emi Dmi7 G7     C
%    nebudu mít,       nebudu mít ...
%
%
%
%
%
%
%
%
%
%
%
%

\begin{song}{Pecky v čokoládě}{Pavel Dobeš}

\begin{xverse}{1.~}
[G]{Čl}ověk jde [D]cestou, aby potkával [G]lidi,
a když je [D]potká, tak musí zas [G]jít,
a když je [D]pozná, tak dělá, že [G]nevidí,
neboť kdyby se [D]zastavil, tak nemoh' by [G]jít.
\end{xverse}

\begin{xverse}{R:~}
[G]Tak se to v životě [Ami]střídá - [D7]vzestup, úpadek, [G]bída,
[Ami]ohlédneš se za štěstím [D7]a vidíš jen jeho [G]záda,
tak se to v životě [Ami]střídá - [D7]vzestup, úpadek, [G]bída,
[Ami]tak se to v životě [D7]jak vrabci na plotě [G]hádá.
\end{xverse}

\begin{xverse}{*:~}
[G]Tak se [D]snaž nemyslet už [G]na ni, tak se [D]snaž, [G|]{}
tak to [D]smaž, proč to děláš [G]dlaní, tak to [D]smaž. [G|]{}
\end{xverse}

\begin{xverse}{2.~}
Také jsem dostal pusu od děvčete,
také jsem sháněl bonpari,
a dneska chodíme každý zvlášť po světě,
spolu jsme si ten svět malovali.
\end{xverse}

\begin{xverse}{R:~}
Tak se to v životě ...
\end{xverse}

\begin{xverse}{3.~}
Ta moje měla ďolík v bradě
a řasy téměř v ofině
a měla ráda pecky v čokoládě,
zbyly jen pecky, asi ne.
\end{xverse}

\begin{xverse}{R:~}
Tak se to v životě ...
\end{xverse}

\begin{xverse}{4.~}
= 1.
\end{xverse}

\begin{xverse}{R:~}
Tak se to v životě ...
\end{xverse}

\begin{xverse}{*:~}
Tak se snaž ...
\end{xverse}

\end{song}




\begin{song}{Píseň psaná na vodu}{Jaromír Nohavica}

\begin{xverse}{1.~}
[G]Nad lotosovým květem na vteřinu zastavil se [C]{čas},
[G]na břehu cop si plete [C]dívka, jež má velmi tichý [G]hlas,
je slyšet klapot větru [Emi]o koruny [Hmi]sekvojových [Cmaj7]dřev,
[C]na tisíc kilometrů [Ami7]slyším její [D7]velmi něžný [G]zpěv.
\end{xverse}

\begin{xverse}{2.~}
Mládenec útloboký odvazuje od břehu člun,
v rákosí u zátoky měsíc se vynořil z vln,
z nebe se na zem sypou stříbrňáky dynastie Čchen
a starý básník Li-Po báseň píše na pergamen.
\end{xverse}

\begin{xverse}{3.~}
V klobouku čaroděje ukrývá se tygr a hroch,
dívce se srdce směje, po hladině přichází hoch,
lehounce vlny plynou, básník má čarovnou moc,
rozprostře svoje ruce nad krajinou, a je tu noc, na na ...
\end{xverse}

\end{song}

\begin{song}{Setkání s A. S. Puškinem}{Jaromír Nohavica/Bulat Okudžava}

\begin{xverse}{1.~}
Co bylo, je [Ami]pryč jednou [E7]provždy, a stesky jsou [Ami]k ničemu,
každá epocha [C]má vlastní [G]kulisy, herce i [C]kus
/: a [A7]líto mi [Dmi]je, že už [G]nikdy se nepotkám s [Ami]Puškinem,
nezajdu si s [E7]ním na čaj do bistra U sedmi [Ami]hus. :/
\end{xverse}

\begin{xverse}{2.~}
Dnes už nechodíme, jak zastara bosáci, naboso,
řvou motory aut, my z oken sčítáme galaxie
/: a líto mi je, že už po Moskvě nejezdí drožkáři
a nebudou víc, nebudou, a to líto mi je. :/
\end{xverse}

\begin{xverse}{3.~}
Před tebou se skláním, má bezbřehá epocho poznání,
a vzdávám ti hold, lidský rozume nad rozumem,
/: a líto mi je, že jak dřív se i dnes bůžkům klaníme
a na kolenou bijem hlavami o tvrdou zem. :/
\end{xverse}

\begin{xverse}{4.~}
Podél našich cest dlouhé aleje vítězných praporů,
náš boj za to stál, my máme vše, za co šli jsme se bít
/: a líto mi je, že se stavějí pomníky z mramoru
a vyšší než my, vyšší než veškerá vítězství. :/
\end{xverse}

\begin{xverse}{5.~}
Co bylo, je pryč, vyjdu ven na špacír, koukám: noc je tu,
vtom zničehonic vedle arbatských vrat u vody
/: stojí drožkář a kůň, a pan Puškin jde po prázdném prospektu,
já krk za to dám, zítra že se něco přihodí ... :/
\end{xverse}

\end{song}


\begin{song}{Prší}{Karel Plíhal}

\begin{xverse}{R:~}
[C]Prší [C/H|]{} [Dmi]a hvězdy [G]na plakátech [C]blednou, [C/H|]{}
[Dmi]zpívám si [E]spolu s repro[Ami]bednou, [Ami/G|]{}
jak ta [F]láska deštěm [G]voní,
stejně [C]voněla i [G]loni, zkrátka
\end{xverse}

\begin{xverse}{1.~}
[C]Prší [C/H|]{} [Dmi]a soused [G]chodí sadem s [C]konví, [C/H|]{}
[Dmi]každej se [E]diví, jenom [Ami]on ví, [Ami/G|]{}
proč [F]místo toho [E]kropení si [A]nezaleze k [D]topení
a [G]nepřečte si [C]McBaina, proč [F]vozí mouku [E]do mlejna.
\end{xverse}

\begin{xverse}{R:~}
Prší a hvězdy na plakátech...
\end{xverse}

\begin{xverse}{2.~}
Prší a soused venku prádlo věší,
práce ho, jak je vidět, těší,
ač promáčen je na nitku, tak na co volat sanitku,
stejně na čísle blázince je věčně někdo na lince.
\end{xverse}

\end{song}


\begin{song}{Sbohem, galánečko}{Vlasta Redl}

\begin{xverse}{1.~}
[G]Sbohem, galá[Emi]nečko, [Ami]já už musím [D7]jí[G]ti, [A7|]{}
[D]sbohem, galán[Hmi]ečko, [Emi]já už musím [A7]jí[D]ti,
[Ami]kyselé ví[D]nečko, [G]kyselé vín[Ami]ečko [D7|]{}  [G]podalas' mně k [Ami D7|]pi[G]tí,
[Ami]kyselé ví[D]nečko, [G]kyselé ví[C]neč[D7]ko [G]poda[Emi]las' mně k [Ami D7|]pi[G]tí.
\end{xverse}

\begin{xverse}{2.~}
/: Sbohem, galánečko, rozlučme sa v pánu, :/
/: kyselé vínečko kyselé vínečko podalas' mně v džbánu. :/
\end{xverse}

\begin{xverse}{3.~}
/: Ač bylo kyselé, přecaj sem sa opil, :/
/: eště včil sa stydím, eště včil sa stydím, co sem všecko tropil. :/
\end{xverse}

\begin{xverse}{4.~}
/: Ale sa něhněvám, žes' mňa ošidila, :/
/: to ta moja žízeň, to ta moja žízeň, ta to zavinila. :/
\end{xverse}

\end{song}


\begin{song}{Složitě}{Pavel Dobeš}

\begin{xverse}{1.~}
Tak už jsme [G]opět na trávě,
neslavně [Ami]skončil ten náš let,
[C]letěli jsme bezhlavě
a dosáh[G]nout jsme chtěli hvězd.
\end{xverse}

\begin{xverse}{*:~}
[Hmi]Letěli jsme spolu [C]ku štěstí,
už [Hmi]vypršelo, už je [D]po dešti.
\end{xverse}

\begin{xverse}{R:~}
Složi[G]tě žijem, [C]{ži}jem, složi[D]tě, [D/C|]{} [D/H|]{} [D/A|]{}
složi[G]tě žijem, [C]{ži}jem, složi[D]tě. [D/C|]{} [D/H|]{} [D/A|]{}
\end{xverse}

\begin{xverse}{2.~}
V korytě řeky se zvedá kalný proud,
chtěla jsem létat, tak nenuťte mě plout,
vždyť všude čtu a pokud vím,
tak ženská, to je jenom slabej tvor.
\end{xverse}

\begin{xverse}{3.~}
Škola už také mizí pod vodou,
místa nejsou ani náhodou,
jen řeka stoupá z břehů
a povodeň se šíří jako mor.
\end{xverse}

\begin{xverse}{*~}
Dívčí představy, ty odplouvají do dálav,
a kluk, co jsem mu patřila, mi mlčky říká: plav.
\end{xverse}

\begin{xverse}{R:~}
Složitě...
\end{xverse}

\begin{xverse}{4.~}
Tak vidíš, holka, není to tak složitý,
dospěli jiní, teď chápeš to i ty,
a ještě včera bylas' hloupé rozmazlené dítě.
\end{xverse}

\begin{xverse}{5.~}
Teď plaveš řekou, vzal tě její proud,
souhlasíš s ohněm, co musel uhasnout,
až půjdou kolem rybáři, tak chytnou tě i do děravě sítě.
\end{xverse}

\begin{xverse}{*~}
Jeden z nich ozdobí ti šaty vzácným kamením,
schováš se za jeho příjmením,
\end{xverse}

\begin{xverse}{R:~}
Složitě...
\end{xverse}
\end{song}


\begin{song}{Těšínská}{Jaromír Nohavica}

\begin{xverse}{1.~}
[Ami]Kdybych se narodil [Dmi]před sto léty [F]v [E]tomhle [Ami]městě, [Dmi|]{} [F|]{} [E|]{} [Ami|]{}
u Larischů na zahradě [Dmi]trhal bych květy [F|{}] [E]své ne[Ami]věstě, [Dmi|]{} [F|]{} [E|]{} [Ami|]{}
[C]moje nevěsta by [Dmi]byla dcera ševcova
z [F]domu Kamiňskich [C]odněkud ze Lvova,
kochal bym ja i [Dmi]pieščil, [F]chy[E]ba lat [Ami]dwieščie. [Dmi|]{} [F|]{} [E|]{} [Ami|]{}
\end{xverse}

\begin{xverse}{2.~}
Bydleli bychom na Sachsenbergu v domě u žida Kohna,
nejhezčí ze všech těšínských šperků byla by ona,
mluvila by polsky a trochu česky,
pár slov německy, a smála by se hezky,
jednou za sto let zázrak se koná, zázrak se koná.
\end{xverse}

\begin{xverse}{3.~}
Kdybych se narodil před sto lety, byl bych vazačem knih,
u Prohazků dělal bych od pěti do pěti a sedm zlatek za to bral bych,
měl bych krásnou ženu a tři děti,
zdraví bych měl a bylo by mi kolem třiceti,
celý dlouhý život před sebou, celé krásné dvacáté století.
\end{xverse}

\begin{xverse}{4.~}
Kdybych se narodil před sto lety v jinačí době,
u Larischů na zahradě trhal bych květy, má lásko, tobě,
tramvaj by jezdila přes řeku nahoru,
slunce by zvedalo hraniční závoru
a z oken voněl by sváteční oběd.
\end{xverse}

\begin{xverse}{5.~}
Večer by zněla od Mojzese melodie dávnověká,
bylo by léto tisíc devět set deset, za domem by tekla řeka,
vidím to jako dnes: šťastného sebe,
ženu a děti a těšínské nebe,
ještě že člověk nikdy neví, co ho čeká...
\end{xverse}

\end{song}




% \begin{song}{Vandrování}{Spirituál kvintet}
%
% \begin{xverse}{1.~}
% Jsem [D]student v Praze [G]rád jsem studoval
% [D]krásnou [Emi]dívku jsem [D]přitom věrně miloval
% [A]jiný [Hmi]má ji teď, [A]mě už nechce znát
% proto [D]sám [Emi]šel jsem [Hmi]va[A]ndro[D]vat [G|]{} [D|]{}
% \end{xverse}
%
% \begin{xverse}{2.~}
% Je smutná tahle dlouhá cesta má
% přítel hlad, hůl a stará vesta odraná
% Bůh sám ví kde dnes budu nocovat
% přesto dál musím vandrovat
% \end{xverse}
%
% \begin{xverse}{R.~}
%         A  D              A  F#mi   Emi Hmi Emi H E A E
%    |: Já vím, že prchám marně, ať dojdu sebedál než
%      A                D             Hmi A    D  G     A  D G D
%      slunce zajde, mě vždycky najde můj průvod, stesk a žal    :|
%    \end{xverse}
%
% \begin{xverse}{3.~}
% Mně v žilách prudce proudí mladá krev
% místo lásky však v srdci zbyl jen hořký hněv
% vím, že víc nechci nikdy milovat
% jenom sám budu vandrovat
% \end{xverse}
%
% \begin{xverse}{4.~}
% Až přijde den, kdy dojdu na štaci
% odkud cesta jde, co se nikdy nevrací
% nemám strach, jenom stačím sbohem dát -
% budu sám, jak jsem zvyklý vandrovat
% \end{xverse}
%
% \begin{xverse}{R.~}
% Já vím, že ...
% \end{xverse}



\begin{song}{V bufetě}{Jaromír Nohavica}

\begin{xverse}{1.~}
V [C]bufetě na stojáka [F]jím párek s hořči[C]cí
hned vedle tramvajáka, [F]má modrou čepi[C]ci,
[Ami]má kruhy [G]pod oči[C]ma, [F]je rozdrn[G7]{čen} z kole[C]jí,
[F]jsem zase, [G7]já jsem zase [C]mezi svý[Ami]ma, mezi [Dmi]těma, kteří [G7]rychle [C]jí.
\end{xverse}

\begin{xverse}{2.~}
Utek' jsem z kanceláří, z teploučka do chladna,
hej, pane pochůzkáři, láska je bezvadná,
půjčte mi vysílačku, anebo raději klíč,
má milá dala si česnekačku, tož abych utekl pryč.
\end{xverse}

\begin{xverse}{R:~}
/: Nemáme [Fadd9]{čas}, ve stoje [C]jíme,
hluboko v [Fadd9]{nás} nepokoj [Emi(C)]dříme,
v polední [G]pauze pár minut [C]jen,
párek a [G]křen, párek a [C]křen. :/
\end{xverse}

\begin{xverse}{3.~}
Čekám tě ve Džbánu na Kuřím rynku
u sklenky džusu, krabice krekerů,
jestliže nepřijdeš, můj budulínku,
v pijácké pýše se patrně poperu,
poslední stokoruna je zelenou nadějí,
pojď, zajdem raději do Neptuna mezi ty, kteří rychle jí.
\end{xverse}

\begin{xverse}{R:~}
Nemáme čas ...
+ párek a [G]křen, párek a [C]křen ...
\end{xverse}

\end{song}



\begin{song}{V hospodě na rynku}{Jaromír Nohavica}

\begin{xverse}{1.~}
V [C]hospodě [Emi7]na rynku [Ami]dal jsem si [C]rum,
[Fmaj7]než jsem co [C]stačil říct, [Dmi7]vzali mě k [G]vojákům,
[C]svázali [Emi7]lanama, [Ami]mohlo se [C]jet,
[Fmaj7]kůň pletl [C]nohama - [G]pitomej [C]svět.
V [Fmaj7]ležení [Emi7]u hranic [Fmaj7]dali nám [C]kvér,
generál [Dmi7]Líbršvíc [G]ukázal směr:
[Fmaj7]Támhle só [Emi7]Francóze, [Fmaj7]proklatá [C]sběř,
neptej se, [Dmi7]nevzpouzej, [G]kušuj a [C]běž.
\end{xverse}

\begin{xverse}{2.~}
Kuličky burácí, která je má,
země se kymácí pod mýma nohama,
doma už posekli a budou sít
a já tu, v předpeklí, nemám kam jít.
Srdce mi svázali do kozelce,
proti mně poslali ostrostřelce,
a to vše proto jen, že jsem pil rum,
v hospodě sedl si k verbířům.
\end{xverse}

\begin{xverse}{3.~}
Nějaký zpěváček za dvěstě let
jistě si vzpomene, jak jsem se hloupě splet',
ale z cizího neteče a hvězdy lžou,
nesměj se, člověče, na viděnou.
Falešní hostinští nalejvaj rum,
panenky za okny mávají mládencům.
Tady, co ležím, teď vyroste strom,
/: prachbídně zahyne Napoleon. :/
\end{xverse}
\end{song}

\begin{song}{Vlaštovko, leť}{Jaromír Nohavica}

\begin{xverse}{1.~}
[C]Vlaštovko, leť [Ami]přes Čínskou zeď,
[F]přes písek pouště Gobi, [G|]{}
[C]oblétni zem, [Ami]přileť až sem,
[F]jen ať se císař zlo[G]bí.
[Emi]Dnes v noci zdál se mi [Ami]sen,
[F]{že} ti zrní nasypal [G]Ludwig van Beethoven,
[C]vlaštovko, leť, [Ami]nás, chudé, veď. [F|]{} [G|]{} [C|]{}
\end{xverse}

\begin{xverse}{2.~}
Zeptej se ryb, kde je jim líp,
zeptej se plameňáků,
kdo závidí, nic nevidí
z té krásy zpod oblaků.
Až spatříš nad sebou stín,
věz, že ti mává sám pan Jurij Gagarin,
vlaštovko, leť, nás, chudé, veď.
\end{xverse}

\begin{xverse}{3.~}
Vlaštovko, leť rychle a teď,
nesu tři zlaté groše,
první je můj, druhý je tvůj,
třetí pro světlonoše.
Až budeš unavená,
pírka ti pofouká Máří Magdaléna,
[C]vlaštovko, leť, [Ami]nás, chudé, veď, [F|]{} [G|]{} [C|]{}
vlaštovko, leť, [Ami]nás, chudé, veď,
[C]vlaštovko, leť ...
\end{xverse}

\end{song}







\begin{song}{Zestárli jsme, lásko}{Jaromír Nohavica}

\begin{xverse}{1.~}
[A]Náš syn je už veliký, [E/G#]do plínek už nedě[F#mi]lá, [E|]{}
[A]zestárli jsme, lásko, s [E/G#]ním doce[Hmi7]la,  [E7|]{}
[D]oči má po mně a [A]vlasy po tobě,
[D]padají mu do če[A]la,
[E]tak [A]nebuď z toho smutná, [E]buď radši [A]veselá.
\end{xverse}

\begin{xverse}{2.~}
Naše dcera je už veliká, když koupe se, je nesmělá,
zestárli jsme, lásko, s ní docela,
kluci koukají se po ní, jak koukali jsme po tobě,
a my zamykáme panelák,
tak nebuď z toho smutná, buď radši veselá.
\end{xverse}

\begin{xverse}{R:~}
Protože [D]jedna a jedna jsou [A]{čt}yři
a dvě [D]hrušky a dvě jabka, to je [A]osm třešní na ta[A]lí[F#mi]{ři},
[Hmi]i kdybys [E7]nechtě[A]la.
\end{xverse}

\begin{xverse}{3.~}
Náš starý kredenc z roku jedna-dvě už brzy dodělá,
zestárli jsme, lásko, s ním docela,
kávový svatební servis z Číny
dostal léty pěkně do těla,
tak nebuď z toho smutná, buď radši veselá.
\end{xverse}

\begin{xverse}{4.~}
Líbat se tak na ulici, to se v našem věku nedělá,
zestárli jsme, lásko, zestárli jsme docela,
v televizi běží film pro pamětníky,
pan Marvan dělá Anděla,
tak nebuď z toho smutná, buď radši veselá.
\end{xverse}

\begin{xverse}{R:~}
Protože jedna a jedna ...
\end{xverse}

\begin{xverse}{5.~}
I ta píseň, co jsem kdysi pro tě napsal, je už omšelá,
zestárli jsme, lásko, s ní docela,
ale včera, když jsi spala a já na tebe koukal,
napsal jsem ti novou docela,
a to je tahleta píseň, trošku smutná, trošku veselá,
a to je [A]tahleta píseň, trošku [E]smutná, [D]trošku [E]veselá [A]...
\end{xverse}

\end{song}



\begin{song}{6 \& 90}{Vlasta Redl}

\begin{xverse}{1.~}
Nemůžu uvěřit, [D]bráško,
že na tě vůbec už nevzpomí[A]ná,
kdesi nad drahou [D]flaškou,
když je nálada nepovin[A]ná,
málem jsem [C#mi]nadělal z kytary [D]dřívka,
že se mi o tobě už [E]nebude [F#mi]zdát,
chvilka [A]zaváhání před vraty [D]chlívka
a je tu rok šesta[E]devade[F#mi]sát,
a máme [D]rok šesta[E]devade[A]sát.
\end{xverse}

\begin{xverse}{2.~}
Fantys' vyplatil v Spartách
a ani nebyla tak rozpačitá,
vždyť co se na seně vyhraje v kartách,
stejně se na zemi už nepočítá,
cítím, jak na patách něco tě líže,
něco, co ví, že tu budete stát,
dokud se jeden z vás nepohne blíže,
někam do roku šestadevadesát,
aspoň do roku šestadevadesát.
\end{xverse}

\begin{xverse}{*:~}
Dokud se [A]někdo z nás nepohne [D]blíže,
někam do roku šesta[D]devade[F]sát. [G|]{} [A|]{}
\end{xverse}

\begin{xverse}{3.~}
Upřímně řečeno - láska
už dávno není to, co nedá mi spát,
dva blázni v plynových maskách
a všechno, co se nam nemůže stát,
dokud mi tvoje srdce hoří jak svíce,
takže i v tmách aspoň slova jdou psát,
vážně netuším, co si přát více
na konci roku šestadevadesát.
\end{xverse}

\end{song}